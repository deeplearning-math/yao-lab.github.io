\documentclass[twoside,twocolumn,journal]{IEEEtran}
\usepackage{amsmath}
\interdisplaylinepenalty=2500
\usepackage{amssymb}
%\usepackage{cite}
\usepackage{chicagoc}
\usepackage{mathrsfs}

\newtheorem{thm}{Theorem}[section]
\newtheorem{lem}[thm]{Lemma}
\newtheorem{prop}[thm]{Proposition}
\newtheorem{cor}[thm]{Corollary}
\newtheorem{rem}[thm]{Remark}

\newenvironment{mainthm}[1][Main Theorem]{\medskip \noindent {\bf #1.}\begin{em}}{\end{em}\medskip}
\newenvironment{pf}[1][Proof]{\medskip\noindent\hspace{1em}{\itshape #1: }}{\hspace*{\fill}~\QED\par\endtrivlist\medskip}

\newcommand{\DS}{\displaystyle}

\def\N{{\mathbb N}}
\def\Q{{\mathbb Q}}        % rationals
\def\Z{{\mathbb Z}}        % integers
\def\R{{\mathbb R}}        % reals
\def\Rn{{\R^{n}}}          % product of n copies of reals

\def\P{{\mathbb P}}        % probability
\def\E{{\mathbb E}}        % expectation
\def\1{{\mathbf 1}}        % indicator
\def\var{{\mathop{\mathbf Var}}}    % variance
\def\<{{\langle}}
\def\>{{\rangle}}

\def\L{{\mathscr L}}
\def\A{{\mathscr A}}
\def\C{{\mathscr C}}
\def\H{{\mathscr H}}
\def\K{{\mathscr K}}
\def\S{{\mathscr S}}
\def\T{{\mathscr T}}
\def\X{{\mathscr X}}
\def\Y{{\mathscr Y}}
\def\ZZ{{\mathscr Z}}
\def\y{{\mathbf y}}
\def\z{{\mathbf z}}
\def\Err{{\mathscr E}}
\def\Prob{{\bf Prob}}
\def\Proj{{\rm Proj}}
\def\grad{{\rm grad}}
\def\ess{{\rm ess}}
\def\supp{{\rm supp}}
\def\span{{\rm span}}
\def\O{{\mathscr O}}
\def\amax{{\mu_{\max}}}
\def\amin{{\mu_{\min}} }

\begin{document}

\title{On Complexity Issue of Online Learning Algorithms}

\author{Yuan~Yao
\thanks{Manuscript submitted March 28, 2005, with Confirmation Number 1630. This work was supported by NSF grant 0325113.}
\thanks{Yuan Yao is with the Department of Mathematics, University of California at Berkeley, Berkeley, CA 94720 (E-mail: yao@math.berkeley.edu). This work
was done while the author visited the Toyota Technological Institute at Chicago, 1427 East 60th Street, Chicago, IL 60637.}}

\maketitle

\markboth{Submitted to: IEEE Transactions on Information
Theory,~Vol.~xx, No.~y,~Month~2010}{Yao: On Complexity Issue of
Online Learning Algorithms}


\begin{abstract}
In this paper, some new probabilistic upper bounds are presented for the
online learning algorithm proposed in \cite{SmaYao04}, and more
generally for linear stochastic approximations in Hilbert spaces.
With these upper bounds not only does one recover almost sure
convergence, but also relaxes the square summable condition on
the step size appeared in our early work. We also give two probabilistic
upper bounds for an averaging process, both of which achieve the
same rate with respect to sample size as in ``batch learning'' algorithms.
\end{abstract}

\begin{keywords}
Online learning, regularization, stochastic approximation, averaging
process, reproducing kernel Hilbert Space.
\end{keywords}

%\setcounter{section}{-1}


%%%%%%%%%%%%%%%%%%%%%%%%%%%%%%%%%%%%%%%%%%%%%%%%%%%%%%%%%%%%%%%%%%%%%%%%%%%%%%%%%%%%%%%%%%%%

\section{Introduction}

%%%%%%%%%%%%%%%%%%%%%%%%%%%%%%%%%%%%%%%%%%%%%%%%%%%%%%%%%%%%%%%%%%%%%%%%%%%%%%%%%%%%%%%%%%%%

\PARstart{S}{upervised} learning, or learning from examples, is to
find a function in a hypothesis space $\H$, which associates an input $x\in \X$ to an output $y\in \Y$, by drawing examples
$(x_t,y_t)_{t\in \N}$ at random from a probability measure $\rho$ on $\X\times
\Y$. By ``online learning'', we mean a sequential decision process $(f_t)_{t\in \N}$ in the
hypothesis space, where each $f_{t+1}$ is decided by the
current observation and $f_{t}$ which only depends on previous
examples, i.e. $f_{t+1}=T_{z_t}(f_t)$ where $z_t=(x_t,y_t)$ (see, e.g. \cite{SmaYao04}, \cite{CesConGen04}). As a contrast, ``batch learning'' refers to a
decision utilizing the whole set of examples (see, e.g.,
\cite{CucSma02}, \cite{EvgPonPog99}).

In the scheme of regularization, one wants to
approximate a function $f^\ast_\lambda$ as a solution of the following
optimization problem (see, e.g., \cite{EvgPonPog99}, \cite{PogSma03}),
\begin{equation} \label{eq:rls}
 \min_{f\in \H} \int_{\X\times \Y} V(f(x),y) d \rho + \lambda \|f\|^2_\H, \ \ \lambda>0,
\end{equation}
where the hypothesis space $\H$ is associated with a norm $\|\cdot\|_\H$ and
$V:\H\times \X\times \Y \to \R$ is a loss function, which measures the prediction cost of $f$ at $x$ against $y$.
Among a variety of choices on $V$ and $\H$, it leads to a simple structure but deeper understanding by selecting the quadratic loss
$V(f(x),y)=(f(x)-y)^2$ and the hypothesis space $\H=\H_K$, the reproducing kernel Hilbert space (RKHS) associated with a
Mercer kernel $K$. In this setting there exists a unique minimizer $f^\ast_\lambda$, satisfying the following linear equation,
\begin{equation}  \label{eq:lineq}
(L_K + \lambda I)f = L_K f_\rho ,
\end{equation}
where $f_\rho(x) = \int_\Y y d \rho_{\Y|x}$, the conditional expectation of $y$, is called the \emph{regression
function}; and the integral operator $L_K:\L^2_\rho(\X) \to \L^2_\rho(\X)$ is defined by $ L_K(f)=\int_\X K(x,t) f(t) d \rho_\X$.
Since $L_K+\lambda I$ ($\lambda>0$) is invertible, we may write $f^\ast_\lambda = (L_K + \lambda I )^{-1} L_K f_\rho$.
Moreover, such a choice avoids the estimation of covering numbers of $\H$,
which is difficult in most cases \cite{CucSma02}, \cite{Zhou03}; it provides a simple estimate of optimal upper bounds asymptotically meeting
lower bounds \cite{SmaZho-ShannonIII}, \cite{CapDev05}; it bridges over the linear inverse problem toward other regularization schemes
\cite{DevRosCap04}, \cite{EngHanNeu00}; and more interestingly in this paper, it takes an especially simple form in online learning
algorithms \cite{SmaYao04}.

Given an independent and identically distributed random sequence $(x_t,y_t)_{t\in \N}$, the algorithm in \cite{SmaYao04} returns a sequence
$(f_t)_{t\in\N}\in \H_K$ to approximate $f^\ast_\lambda$,
\begin{equation}\label{eq:rkhs}
f_{t+1} = f_t - \gamma_t(( f_t(x_t)-y_t) K_{x_t} +\lambda f_t ),
\end{equation}
where $f_1\in \H_K$, e.g. $f_1=0$, and in this paper the step size $\gamma_t>0$ is chosen as $\gamma_t=O(t^{-\theta})$ for some $\theta \in [0,1)$.

The algorithm can be regarded as either the stochastic approximation of
the gradient descent method for (\ref{eq:rls}), or
the stochastic approximation of the linear equation (\ref{eq:lineq}),
which was originally proposed in \cite{RobMon51}, \cite{KieWol52}.
Traditional analysis on stochastic approximations has been
focusing on convergence and asymptotic rates. A convergence result
often used in applications, known as the Robbins-Siegmund Theorem
\cite{RobSie71}, imposes a condition on the step size that $\sum_t
\gamma_t = \infty$ and $\sum_t \gamma_t^2<\infty$, and leads to
the almost sure convergence (with probability one). For the step size
chosen in this paper, $\gamma_t = O(t^{-\theta})$, this requires
$\theta\in(1/2,1)$. In this setting, the asymptotic rate has been
shown as $O(\gamma_t^{1/2})=O(t^{-\theta/2})$. Note that the
condition $\sum_t \gamma_t = \infty$, is used to ``forget'' the
error caused by initial choices. However the square summable
condition, $\sum_t \gamma_t^2<\infty$, is not necessary for
the almost sure convergence. For example in \cite{Duflo97} (or see the
remarks in \cite{Benaim99}), to ensure the almost sure convergence
it is enough that for all $c>0$,
\[ \sum_{t} e^{-c/\gamma_t} < \infty. \]
This even justifies the use of
$\gamma_t = 1/\log^{1+\epsilon} t$ for some $\epsilon>0$, which is however not pursued in this paper. For more background on stochastic
approximations, see for example \cite{Duflo96}, \cite{KusYin03}, and references therein.

In learning theory a fundamental goal is to approximate the regression function $f_\rho$. For this purpose, it is not enough to apply traditional results
on convergence and asymptotic rates; since to approximate $f_\rho$ arbitrarily well, we need to tune the regularization parameter $\lambda$ arbitrarily
small as sample size goes large. The influence of $\lambda$ to convergence is hidden in the constants and thus we seek upper bounds to disclose it.

In \cite{SmaYao04}, we present a probabilistic upper bound based on Markov's Inequality, that the following holds
with probability at least $1-\delta$ ($\delta\in (0,1)$)
\[ \|f_t-f^\ast_\lambda\|_K \leq O(\lambda^{-\frac{\theta}{2(1-\theta)}} t^{-\theta/2} \delta^{-1/2}),\ \ \mbox{$\theta\in(1/2,1)$}. \]
This upper bound is tight in the asymptotic rate of $t$; however, it only implies that $f_t$ converges to
$f^\ast_\lambda$ in probability, weaker than the almost sure convergence.

In this paper, we present two new probabilistic upper bounds by using exponential probabilistic inequalities for martingales in Hilbert spaces
\cite{Pinelis94}, both of which lead to almost sure convergence and extend the rate of step size to $\theta\in [0,1)$, at the sacrifice of rates on
$\lambda$.

The first upper bound (as Theorem A) says that with probability at least $1-\delta$ ($\delta\in (0,1)$),
\[ \|f_t-f^\ast_\lambda\|_K \leq O(\lambda^{-1 - \frac{1}{2(1-\theta)}} t^{-\theta/2}\log^{1/2} 1/\delta), \ \ \mbox{$\theta\in [0,1)$}. \]
This upper bound implies almost sure convergence for all $\theta\in (0,1)$, by changing $1/\delta$ to $\log 1/\delta$. Note that
when $\theta=0$, algorithm (\ref{eq:rkhs}) is often called the \emph{Adaline} or \emph{Widrow-Hoff algoirthm}
(\cite{WidHof60}, or see Chapter 5 in \cite{CriSha00}),
which is not guaranteed to converge in this setting.

The second upper bound (as Theorem B) is given for the \emph{averaging process} proposed in \cite{Polyak90}, \cite{Ruppert88},
\begin{equation} \label{eq:averaging}
 \bar{f}_t = \frac{1}{t} \sum_{j=1}^t f_j = \bar{f}_{t-1} + \frac{1}{t} (f_t - \bar{f}_{t-1}), \ \bar{f}_1=f_1,
\end{equation}
that the following holds with probability at least $1-\delta$ ($\delta\in (0,1)$),
\[ \|\bar{f}_t - f^\ast_\lambda\|_K \leq O(\lambda^{-2}t^{-1/2} \log^{1/2} 1/\delta), \ \ \mbox{$\theta \in [0,1)$}. \]
In contrast to ``batch learning'' case with a rate $O(\lambda^{-1} t^{-1/2})$ \cite{SmaZho-ShannonIII}, this upper bound achieves the same fixed rate
in $t$ for all $\theta\in [0,1)$, while losing the rate in $\lambda$.

It is possible to improve the rate in $\lambda$ by turning back to Markov's Inequality. In fact, the reason of loss in $\lambda$ lies in the application
of the Hoeffding-style inequalities which, compared to Markov's Inequality, replace the variance by its uniform upper bounds.
Based on this observation, we obtain the following result (as Theorem B*) for the averaging process by using Markov's Inequality,
\[ \|\bar{f}_t - f^\ast_\lambda\|_K \leq O(\lambda^{-1}t^{-1/2} \delta^{-1/2}), \ \ \mbox{$\theta \in [0,1)$}, \]
which holds with probability at least $1-\delta$ ($\delta\in (0,1)$). However, an explicit upper bound on variance only leads to
$O(\lambda^{-3/2}t^{-1/2} \delta^{-1/2})$. It is not clear yet whether we can improve this by using other variance-based inequalities, such as
Bennet's or Bernstein's Inequality.

The organization of this paper is as follows. In Section II, we present our main results and discussions. In Section III, we study
a more general problem, linear stochastic approximation in Hilbert spaces, from which we derive Theorem A and B in a special case.
We propose in Section IV a martingale decomposition for remainders, which is crucial for later development. All the proofs for the theorems in Section III
are collected in Section V. In section VI we prove Theorem B* via a reverse martingale decomposition for remainders. Conclusion and open problems
are summarized in Section VII. The last section is an appendix collecting some crucial estimates used in this paper.

\section{Main Results}

Before presenting the main results, we need some definitions and remarks on notation.

In this paper, let $\X\subseteq \R^n$ be compact, $\Y=\R$, and $\ZZ=\X\times \Y$. Assume that there is a $M_\rho>0$ such that
$\supp (\rho) \subseteq \X\times [-M_\rho,M_\rho]$.
Define
\begin{equation} \label{eq:ck}
C_K:=\max_{x\in\X} \sqrt{K(x,x)} < \infty.
\end{equation}
and a constant only depending on $\theta\in [0,1)$,
\begin{equation} \label{eq:D}
D_\theta= 1+ 2^{\frac{\theta}{1-\theta}} \left(1+\Gamma \left(\frac{1}{1-\theta} \right) \right) \geq 1.
\end{equation}
Assume that the examples $(x_t,y_t)_{t\in \N}$ are independent and identically distributed (i.i.d.) according to $\rho$.


In this paragraph, we provide a short background on reproducing kernel Hilbert spaces (RKHS).
A function $K:X\times X\to \R$ is called a \emph{Mercer kernel}, if it is a
continuous symmetric real function which is \emph{positive
semi-definite} in the sense that $\sum_{i,j=1}^l c_i c_j K(x_i,
x_j)\geq 0$ for any $l\in \N$ and any choice of $x_i\in X$ and
$c_i \in \R$ ($i=1,\ldots,l$). Let $\H_K$ be the Reproducing Kernel Hilbert Space
associated with a Mercer kernel $K$. Recall the definition as
follows. Consider the vector space $V_K$ generated by $\{K_x:x\in
X\}$, i.e. all the finite linear combinations of $K_x$, where for
each $x\in X$, the function $K_x:X\to \R$ is defined by
$K_x(x')=K(x,x')$. A semi-definite inner product $\<\ ,\ \>_K$ on
this vector space can be defined as the unique linear extension of
$\<K_x,K_{x'}\>_K:=K(x,x')$. The induced semi-norm is $\|f\|_K =
\sqrt{\<f,f\>_K}$ for each $f\in V_K$. Notice that the zero set
$V_0=\{f\in V_K: \|f\|_K=0\}$ is a subspace. Then the
semi-definite inner product induces an inner product on the
quotient space $V_K/V_0$. Let $\H_K$ be the completion of this
inner product space $V_K/V_0$ with respect to $\|\cdot\|_K$. The most important
property of RKHS is the so called \emph{reproducing property}: for any $f\in
\H_K$, $f(x)=\<f,K_x\>_K$ ($x\in X$). RKHS can be regarded as a generalization of
real analytic functions (or band-limited functions), see for example \cite{Daubechies92}, \cite{SmaZho-ShannonI}.

Recall the definition of the \emph{incomplete gamma function} restricted on $[0,\infty)\times [0,\infty)$,
\[ \Gamma(a,x)=\int_x^\infty s^{a-1} e^{-s} d s, \ \ \ \ \ \ \mbox{where $a,x\geq 0$}. \]
The \emph{gamma function} is defined by $\Gamma(a) = \Gamma(a,0)$. Finally we make a remark on notation. When $n<m$, the product and summation,
$\prod_{i=m}^n x_i$ and $\sum_{i=m}^n x_i$, are understood to be $1$ and $0$, respectively. We use $\E_z$ and $\E_{z_1|z_2}$ to denote the
expectation and conditional expectation, respectively. Shorthand notation $\E$ is also used when its meaning is clear from the context.


The following are the main results in this paper.

%%%%%%%%%%%%%%%%%%%%%%%%%
%
%    Theorem  A
%
%%%%%%%%%%%%%%%%%%%%%%%%%

\subsection{Probabilistic Upper Bound with Almost Sure Convergence}

\begin{mainthm}[Theorem A]
Let $\lambda\leq \lambda_0$, $\gamma_t = t^{-\theta}/(C^2_K + \lambda)$ for some $\theta\in [0,1)$, and $f_1=0$.
Then for all $t\in \N$ there holds
\[ \| f_t - f^\ast_\lambda \|_K \leq \Err_{init}(t) + \Err_{samp}(t), \]
where
\[ \Err_{init}(t)\leq e^{\frac{\alpha}{1-\theta}(1-t^{1-\theta})} \|f^\ast_\lambda\|_K, \]
and with probability at least $1-\delta$ ($\delta\in (0,1)$),
\[ \Err_{samp}(t) \leq C_{\rho,\theta,K}  \left(\frac{1}{\lambda} \right)^{1+\frac{1}{2(1-\theta)}}
\left(\frac{1}{t}\right)^{\frac{\theta}{2}} \log^{1/2}\frac{2}{\delta}. \]
Here $C_{\rho,\theta,K} = 16\sqrt{D_\theta}C_K M_{\rho}(\lambda_0+C^2_K)^{1/2(1-\theta)}$.
\end{mainthm}

The proof of Theorem A will be given in Section III as a corollary of Theorem \ref{thm:robmon}.

\medskip

\begin{rem} The second inequality is equivalent to
\[ \Prob \{ \Err_{samp}(t) \geq \varepsilon  \} \leq 2e^{- c \varepsilon^2 t^{\theta} } \]
where $c= \lambda^{2+\frac{1}{1-\theta}}/C^{2}_{\rho,\theta,K}$ .
For each $\varepsilon>0$, denote by $A_t$ the event $\{\Err_{samp}(t)\geq \varepsilon\}$. Then
\[ \sum_{t\in \N} \Prob (A_t) \leq 2 \sum_{t\in \N} e^{- c\varepsilon t^\theta} < \infty. \]
By the Borel-Cantelli Lemma, we have $\Prob(A_t\mbox{ i.o.})=0$, i.e. it is of zero probability that $A_t$ happens for infinitely many values $t\in \N$,
whence $\Err_{samp}(t)\to 0$ almost surely (with probability one).
\end{rem}

\medskip

\begin{rem}
Note that when $\theta=0$, the \emph{Widrow-Hoff} algorithm \cite{WidHof60} can't ensure its convergence by this upper bound.
However, it can be combined with the averaging process to achieve a convergence rate of $O(t^{-1/2})$,
which will be discussed in the next subsection.
\end{rem}

\subsection{Averaging Process}
It is natural to consider the average of the ensemble
$\{f_1,\ldots,f_t\}$ up to time $t$, which might improve the
convergence rate since by intuition averaging may reduce variance.
In stochastic approximation, this acceleration by averaging was
firstly observed independently by \cite{Ruppert88} and
\cite{Polyak90} (or see \cite{PolJud92}) based on
asymptotic analysis; recently this phenomenon has also been
noticed in learning theory society (see, e.g.,
\cite{CesConGen04}). A recent result \cite{KonTsi04} studies
this averaging process in a more general framework of
two-time-scale linear stochastic approximations with asymptotic analysis. Below we show
a probabilistic upper bound with a fixed rate $O(t^{-1/2})$ for all $\theta\in [0,1)$.


%%%%%%%%%%%%%%%%%%%%%%%%%
%
%    Theorem  B
%
%%%%%%%%%%%%%%%%%%%%%%%%%

\begin{mainthm}[Theorem B]
Let $\lambda\leq \lambda_0$, $\gamma_t = t^{-\theta}/(C^2_K + \lambda)$ for some $\theta\in [0,1)$, and $f_1=0$.
Then for all $t\in \N$ there holds for (\ref{eq:averaging})
\[ \| \bar{f}_t - f^\ast_\lambda \|_K \leq \Err_{init}(t) + \Err_{samp}(t). \]
where
\[ \Err_{init}(t)\leq  C_1 \left( \frac{1}{\lambda t} \right), \]
and with probability at least $1-\delta$ ($\delta\in (0,1)$),
\[ \Err_{samp}(t) \leq C_2 \left(\frac{1}{\lambda} \right)^2 \sqrt{\frac{1}{t}} \log^{1/2}\frac{2}{\delta}. \]
Here $C_1=D_\theta(\lambda_0+C^2_K) \|f^\ast_\lambda\|_K$ and $C_2=8 D_\theta C_K M_\rho(\lambda_0+C^2_K)$.
\end{mainthm}

The proof of Theorem B will be given in Section III as a corollary of Theorem \ref{thm:average}.

\medskip

\begin{rem} Assume without loss of generality that $\lambda_0=C^2_K$. When $\theta=0$, $D_0=3$ and this gives the following bound
for combined \emph{Adaline-Averaging} algorithm
\[ \Err_{init}(t)\leq  6C^2_K \|f^\ast_\lambda\|_K \left( \frac{1}{\lambda t} \right),\]
and with probability at least $1-\delta$ ($\delta\in (0,1)$),
\[ \Err_{samp}(t) \leq 48 C^3_K M_\rho \left(\frac{1}{\lambda} \right)^2 \sqrt{\frac{1}{t}} \log^{1/2}\frac{2}{\delta}. \]
\end{rem}

\medskip

%%%%%%%%%%%%%%%%%%%%%%%%%
%
%    Theorem  B'
%
%%%%%%%%%%%%%%%%%%%%%%%%%

Let $\sigma^2_\lambda=\E[\|(y - f^\ast_\lambda(x)) - \lambda f^\ast_\lambda(x)\|^2$ for some $\sigma_\lambda \geq 0$. Then by Markov's Inequality we
improve the rate in $\lambda$ as in the following theorem, whose proof will be given in Section VI.

\begin{mainthm}[Theorem B*]
Let $\lambda\leq \lambda_0$, $\gamma_t = t^{-\theta}/(C^2_K + \lambda)$ for some $\theta\in [0,1)$, and $f_1=0$.
Then the following holds for all $t\in \N$,
\[ \| \bar{f}_t - f^\ast_\lambda \|_K \leq \Err_{init}(t) + \Err_{samp}(t), \]
where
\[ \Err_{init}(t)\leq D_\theta(\lambda_0+C^2_K) \|f^\ast_\lambda\|_K \left( \frac{1}{\lambda t} \right), \]
and with probability at least $1-\delta$ ($\delta\in (0,1)$),
\[ \Err_{samp}(t) \leq \frac{D_\theta \sigma_\lambda}{ \lambda \sqrt{\delta t}}. \]
\end{mainthm}

\begin{rem} It should be noticed that $\sigma_\lambda$ depends on $\lambda$, whence it is not clear yet if the rate $O(\lambda^{-1}t^{-1/2})$ is
achievable. A preliminary result by Proposition \ref{prop:varbnd}-2, gives an estimate $\sigma_\lambda=O(\lambda^{-1/2})$.
With this we have
\[ \|\bar{f}_t - f^\ast_\lambda\|_K \leq O\left(\left(\frac{1}{\lambda}\right)^{3/2}\sqrt{\frac{1}{\delta  t}}\right). \]
\end{rem}

\medskip

\begin{rem}
If $\sigma^2_\lambda=0$, we obtain the following upper bound in a deterministic setting,
\[ \| \bar{f}_t - f^\ast_\lambda \|_K \leq D_\theta(\lambda_0+C^2_K) \|f^\ast_\lambda\|_K \left( \frac{1}{\lambda t} \right). \]
\end{rem}

%%%%%%%%%%%%%%%%%%%%%%%%%%%%%%%%%%%%%%%%%%%%%%%%%%%%%%%%%%%%%%%%%%%%%%%%%%%%%%%%%%%%%%%%%%%

\subsection{Comparison with ``Batch Learning'' Results}

%%%%%%%%%%%%%%%%%%%%%%%%%%%%%%%%%%%%%%%%%%%%%%%%%%%%%%%%%%%%%%%%%%%%%%%%%%%%%%%%%%%%%%%%%%%%

Given a sample $\z=\{(x_i,y_i):i=1,\ldots,t\}$, ``batch learning'' means solving the following
\emph{regularized least square} problem (see, e.g., \cite{EvgPonPog99},
\cite{CucSma02})
\[ \min_{f\in \H_K} \frac{1}{t}\sum_{i=1}^t (f(x_i)-y_i)^2 + \lambda \|f\|^2_K, \ \ \lambda>0. \]
There exists a unique minimizer $f_{\lambda,\z}$ satisfying
\[ f_{\lambda,\z}(x)= \sum_{i=1}^t a_i K(x,x_i) \]
where $a=(a_1,\ldots,a_t)$ is the solution of the linear equation
\[ (\lambda t I + K_\z) a = \y, \]
with $t\times t$ identity matrix $I$, $t\times t$ matrix $K_\z$ whose $(i,j)$ entry is $K(x_i,x_j)$ and $ \y=(y_1,\ldots,y_t)\in \R^t$.

A probabilistic upper bound for
$\|f_{\lambda,\z}-f^\ast_\lambda\|_K$ is given in
\cite{CucSma02b}, and this has been substantially improved by
\cite{DevCapRos04} using also some ideas from \cite{BouEli02}.
Moreover, \cite{Zhang03} gives error bounds expressed in a different form.
A recent result (Theorem 1 in \cite{SmaZho-ShannonIII}) shows that,

\medskip

\begin{thm} With probability at least $1-\delta$ ($\delta\in (0,1)$) there holds
\[ \| f_{\lambda,\z} - f^\ast_\lambda \|_K \leq \frac{6C_K M_\rho\log(2/\delta)}{\lambda \sqrt{t}}.\]
\end{thm}

\medskip

\begin{rem}
A recent result \cite{CapDev05} shows that the rate $O(\lambda^{-1} t^{-1/2})$ is optimal in the sense that it leads to a
convergence rate asymptotically meeting the minimax lower bound.
Theorem B tells us that the averaging process achieves $O(\lambda^{-2} t^{-1/2})$,
which is optimal in $t$ but suboptimal in $\lambda$. Theorem B* improves
this to $O(\lambda^{-1} t^{-1/2})$, for uniformly bounded $\sigma^2_\lambda$; while for the estimate
provided in Proposition \ref{prop:varbnd}-2, one only achieves $O(\lambda^{-3/2} t^{-1/2})$. Yet it is an open
problem if we can achieve the optimal rate in both $\lambda$ and $t$.
\end{rem}

%%%%%%%%%%%%%%%%%%%%%%%%%%%%%%%%%%%%%%%%%%%%%%%%%%%%%%%%%%%%%%%%%%%%%%%%%%%%%%%%%%%%%%%%%%

\section{Linear Stochastic Approximation in Hilbert Spaces}

%%%%%%%%%%%%%%%%%%%%%%%%%%%%%%%%%%%%%%%%%%%%%%%%%%%%%%%%%%%%%%%%%%%%%%%%%%%%%%%%%%%%%%%%%%

In this section we study a more general problem,
stochastic approximation of linear equations in Hilbert
spaces. Some general upper bounds
are given and they lead to Theorem A and B in a special case.

Let $W$ be a Hilbert space, $A(z):W\to W$ a random
positive operator depending on $z\in \ZZ$ and $B(z)\in W$ a random
vector. Define $\hat{A}=\E_z[A(z)]$ and $\hat{B}=\E_z[B(z)]$.
Consider the following linear equation
\begin{equation} \label{eq:basiclin}
\hat{A}w = \hat{B},
\end{equation}
whose unique solution is $w^\ast= \hat{A}^{-1} \hat{B}$.

In the sequel, we assume that almost surely,

\medskip

\noindent {\bf Finiteness Condition.} A. $\amin I\leq  A(z) \leq \amax I$ ($0< \amin\leq \amax < \infty$) and let $\alpha=\amin/\amax\in(0,1]$; \\
B. $\|B(z)\|\leq \beta<\infty$; \\
C. $\E \|A(z) w^\ast - B(z) \|^2 = \sigma^2 <\infty$.

\medskip

Given an i.i.d. sequence $(z_t)_{t\in \N}$, define a sequence $\{w_t\}_{t\in W}$ as successive stochastic approximations of $w^\ast$,
\begin{equation}\label{eq:wt}
w_{t+1} = w_t - \gamma_t (A_t w_t - B_t), \ \ \ \ \ \mbox{$w_1\in W$}
\end{equation}
where $A_t=A(z_t)$, $B_t=B(z_t)\in W$ and $\gamma_t=1/\amax t^{\theta}$ for some $\theta\in [0,1)$.

Define a \emph{remainder} sequence
$(r_t)_{t\in \N}$ by
\[ r_t = w_t - w^\ast, \]
which measures the deviation between $w_t$ and $w^\ast$.
It can be seen that both $w_t$ and $r_t$ are $W$-valued random
variables depending on $z_1,\ldots,z_{t-1}$. In this note we
assume that $(z_t)_{t\in \N}$ is a i.i.d. sequence, but the method
we used here can be extended to more general cases.

The main results in this section are in the following.

\medskip

\begin{thm} \label{thm:robmon}
Let $\gamma_t = t^{-\theta}/\amax$ ($\theta\in [0,1)$) and $w_1=0$. Then for all $t\in \N$, there holds
\[ \| w_{t} - w^\ast \| \leq \Err_{init}(t) + \Err_{samp}(t), \]
where
\[ \Err_{init}(t)\leq e^{\frac{\alpha}{1-\theta}(1-t^{1-\theta})} \|r_1\|, \]
and with probability at least $1-\delta$,
\[ \Err_{samp}(t) \leq \frac{16\sqrt{D_\theta} \beta}{\amax} \left(\frac{1}{\alpha} \right)^{1+\frac{1}{2(1-\theta)}}
\left(\frac{1}{t}\right)^{\theta/2} \log^{1/2}\frac{2}{\delta}. \]
\end{thm}

\medskip

For the averaged sequence $\DS \bar{w}_t = \frac{1}{t} \sum_{j=1}^t w_j$, we have

\medskip

\begin{thm} \label{thm:average}
Let $\DS \gamma_t = t^{-\theta}/\amax$ ($\theta\in [0,1)$) and $w_1=0$. Then for all $t\in \N$ there holds
\[ \| \bar{w}_t -w^\ast \| \leq \Err_{init}(t) + \Err_{samp}(t), \]
where
\[ \Err_{init}(t)\leq  D_\theta \left(\frac{1}{\alpha t} \right)\|r_1\| , \]
and with probability at least $1-\delta$,
\[ \Err_{samp}(t) \leq  \frac{8 D_\theta \beta}{\amax} \left(\frac{1}{\alpha}\right)^2 \sqrt{\frac{1}{t}} \log^{1/2} \frac{2}{\delta}. \]
\end{thm}

\subsection{Proofs of Theorem A and B}

\begin{pf}[Proof of Theorem A]
We first show that the algorithm given by (\ref{eq:rkhs}) can be derived from equation (\ref{eq:wt}); then Theorem A follows from Theorem \ref{thm:robmon}.

Let $E_x:\H_K \to \R$ be the evaluation operator
such that $E_x(f)=f(x)$. Let $E_x^\ast:\R\to \H_K$ be the adjoint of $E_x$ defined by $\<y,E_x(f)\>_{\R} =\<E_x^\ast(y),f\>_{\H_K}$,
whence by reproducing property $f(x)=\<f,K_x\>$, we have $E_x^\ast(y)=y K_x$. Now take $W=\H_K$, define $A(z):\H_K \to \H_K$ by
$f\mapsto  E_x^\ast E_x(f) + \lambda $
and $B(z) = E_x^\ast(-y)$. Then $\hat{A}=L_K + \lambda I$ and $\hat{B}=-L_K f_\rho$.
By this substitution, equation (\ref{eq:wt}) becomes (\ref{eq:rkhs}).

Notice that $\amax= \lambda + C_K^2$, $\amin=\lambda$, and $\beta= C_K M_{\rho}$. Theorem A thus follows from
Theorem \ref{thm:robmon}.
\end{pf}

\begin{pf}[Proof of Theorem B]
In a similar way to the proof of Theorem A, Theorem B follows from Theorem \ref{thm:average}.
\end{pf}

\section{Martingale Decomposition of Remainders}

In this section, we decompose the remainder $r_t$ and its average $\bar{r}_t$ into the sum of two parts: one
is deterministic reflecting the error caused by initial choice, and the other is a martingale reflecting the
fluctuation caused by random sampling. Upper bounds for them will be given in the next section. Such a decomposition is somehow
close to the treatment in Robbins-Siegmund Theorem \cite{RobSie71}, where $\|r_t\|^2$ is transformed into a supermartingale.
But our problem benefits from the linear structure and get a direct decomposition on $r_t$.
We note that such a martingale decomposition can be extended to $\|r_t\|^2$ in nonlinear stochastic approximations, which however
is not pursued here.

First of all we introduce some short-hand notations. Define a
random positive operator on $W$,
\begin{equation} {\Pi}_k^t=
\left\{
\begin{array}{lr}
\DS \prod_{i=k}^t \left( I - \gamma_i A_i \right), & k\leq t; \\
I, & k>t.
\end{array}
\right.
\end{equation}
If we replace $A_i$ by $\hat{A}$, we obtain a deterministic positive operator, say $\hat{\Pi}_k^t$.
Define $Y_t = A_t w^\ast - B_t$, a $W$-valued random variable
depending on $z_t$. Clearly $\E_{z_t} Y_t = 0$ and by Finiteness
Condition-C, $\E\|Y_t\|^2 = \sigma^2$ for all $t$.

The following proposition gives a decomposition of $r_t$ into the sum of a deterministic part and a martingale.

\medskip

\begin{prop} For all $t\in \N$,
\begin{equation} \label{eq:total}
r_{t} = \hat{\Pi}_1^{t-1} r_1 -
\sum_{k=1}^{t-1}\gamma_k  \Hat{\Pi}_{k+1}^{t-1} \chi_k,
\end{equation}
where $\chi_k=(A_k - \hat{A}) w_k + \hat{B}-B_t$ ($1\leq k \leq t$).
\end{prop}

\begin{pf} By equation (\ref{eq:wt})
\begin{eqnarray*}
r_{t+1} & = & w_{t+1}-w^\ast = r_{t} - \gamma_{t} (A_{t} w_{t} - B_{t}) \\
& = & (I - \gamma_t \hat{A}) r_t  - \gamma_t ((A_t-\hat{A})r_t + Y_t ) \\
& = & (I - \gamma_t \hat{A}) r_t - \gamma_t \chi_t,
\end{eqnarray*}
where we can check that
\begin{eqnarray*}
\chi_t & = & (A_t - \hat{A})r_t + Y_t \\
&  = & (A_t - \hat{A}) w_t + \hat{B}- B_t.
\end{eqnarray*}
Then equation (\ref{eq:total}) follows from induction on $t$.
\end{pf}

Note that $\hat{\Pi}_{k+1}^t$ is deterministic, $r_k$ depends on
$z_1,\dots,z_{k-1}$, $A_k-\hat{A}$ and $Y_k$ are both of zero
means depending only on $z_k$. Recall that given a sequence of random variables $(\xi_k)_{k\in \N}$ such that $\xi_k$ depends on
random variables $\{z_i:1\leq i \leq k\}$, $(\xi_k)$ is called a \emph{martingale difference sequence}
if $\E_{z_k|z_1,\ldots,z_{k-1}}[\xi_k]=0$. The sum of a martingale difference sequence is called a \emph{martingale}. This motivates
the following definition of a martingale difference sequence,
\[
\xi_k = \left\{
\begin{array}{ll}
\gamma_k \Hat{\Pi}_{k+1}^{t-1} \chi_k, & 1\leq k\leq t; \\
\xi_k=0, & k> t.
\end{array}
\right.
\]
With this we write,
\begin{equation} \label{eq:total1}
r_{t} = \hat{\Pi}_1^{t-1} r_1 - \sum_{k=1}^{t-1}\xi_k.
\end{equation}

Now consider the averaging process. Define
\[ \bar{w}_{t} = \frac{1}{t}\sum_{i=1}^t w_i=\bar{w}_{t-1} + \frac{1}{t} (w_t-\bar{w}_{t-1}), \ \ \bar{w}_1=w_1, \]
and we study upper bounds for the \emph{averaged remainder} sequence
\[ \bar{r}_{t} = \bar{w}_{t} - w^\ast = \frac{1}{t} \sum_{i=1}^{t} (w_i - w^\ast) = \frac{1}{t} \sum_{i=1}^{t} r_i. \]

The following proposition gives a decomposition of $\bar{r}_t$.

\medskip

\begin{prop} For all $t\in \N$,
\begin{equation}
\bar{r}_{t} = \frac{1}{t}\left(\sum_{j=0}^{t-1} \hat{\Pi}_1^j \right)
r_1 - \frac{1}{t}\sum_{k=1}^{t-1}\gamma_k \left( \sum_{j=k}^{t-1} \hat{\Pi}_{k+1}^j \right)\chi_k,
\end{equation}
\end{prop}

\begin{pf} By equation (\ref{eq:total}),
\begin{eqnarray*}
\bar{r}_{t} & = & \frac{1}{t} \sum_{j=1}^{t} r_j \\
& = & \frac{1}{t}\left(1+\sum_{j=1}^{t-1} \hat{\Pi}_1^j \right) r_1 -
\frac{1}{t}\sum_{j=1}^{t-1} \sum_{k=1}^{j} \gamma_k \hat{\Pi}_{k+1}^j \chi_k \\
& = & \frac{1}{t}\left(\sum_{j=0}^{t-1} \hat{\Pi}_1^j \right) r_1 -
\frac{1}{t}\sum_{k=1}^{t-1}\gamma_k \left( \sum_{j=k}^{t-1} \hat{\Pi}_{k+1}^j \right)\chi_k
\end{eqnarray*}
which ends the proof.
\end{pf}

Let
\[ \eta_k =
\left\{
\begin{array}{ll}
\DS \frac{\gamma_k}{t} (\sum_{j=k}^{t-1} \hat{\Pi}_{k+1}^j)\chi_k, & 1\leq k \leq t; \\
0, & k>t.
\end{array}
\right.
\]
Then $(\eta_k)_{k\in \N}$ is a martingale difference sequence and its sum is a martingale. With this we have
\begin{equation} \label{eq:avgtotal1}
\bar{r}_{t} = \frac{1}{t}\left(\sum_{j=0}^{t-1} \hat{\Pi}_1^j \right) r_1 - \sum_{k=1}^{t-1}\eta_k.
\end{equation}


Now define an \emph{initial error} by $\Err_{init}(t) = \|\hat{\Pi}_1^{t-1} r_1 \|$
(or, $\Err_{init}(t) = \|\frac{1}{t}\left(\sum_{j=0}^{t-1} \hat{\Pi}_1^j \right)
r_1 \|$ in averaging process), which is deterministic and reflects the propagated effect of $r_1$;
and a \emph{sample error} by $\Err_{samp}(t) = \|\sum_{k=1}^{t-1} \xi_k\|$
(or, $\Err_{samp}(t) = \|\sum_{k=1}^{t-1}\eta_k\|$ in averaging process), which is random
and reflects the stochastic error caused by samples. The initial error can be bounded deterministically.
For the sample error, we can obtain probabilistic upper bounds by using the exponential inequalities for martingale
difference sequences in Hilbert spaces \cite{Pinelis94}. We will show this in the next section.

\section{Proofs of Theorem 3.1 and Theorem 3.2}

For simplicity, in this section we choose Hoeffding's inequality for martingale difference sequences in
Hilbert spaces \cite{Pinelis92}. We note here that by choosing Bennet-type inequalities \cite{Pinelis94},
one can get tighter bounds depending on variances $\E\|A_t - \hat{A}\|^2$ and $\E\|B_t-\hat{B}\|^2$, in the sense that when these
variances approach to zero, they lead to deterministic upper bounds.

Before presenting the proofs, we need some preliminary results. The first one is an extension of Hoeffiding's Inequality from real numbers to
Hilbert spaces, which is due to Iosif Pinelis \cite{Pinelis92} (see also Theorem 3.5 in \cite{Pinelis94}).

\medskip

\begin{lem}[Pinelis-Hoeffding] \label{lem:pinelis}
Let $(\xi_i)_{i\in \N}\in \H$ be a martingale difference sequence in a Hilbert space $\H$ such that for all $i$ almost surely
$\|\xi_i\|\leq c_i<\infty$. Then for all $t\in \N$,
\[ \Prob \left\{ \left\|\sum_{i=1}^t \xi_i \right\| \geq \epsilon \right\} \leq
2 \exp \left\{-\frac{\epsilon^2}{2\sum_{i=1}^t c_i^2} \right\}. \]
\end{lem}

\medskip

The following proposition collects some useful estimates.

\medskip

\begin{prop} \label{prop:bounds} Let $\alpha'=\alpha/(1-\theta)$. The following holds for all $t\in \N$,

1. $\|\Pi_k^t \| \leq e^{\alpha'[k^{1-\theta} - (t+1)^{1-\theta}]}$ when $k\leq t$, and the same holds for $\|\hat{\Pi}_k^t\|$;

2. $\left\|\sum_{j=k}^t \Pi_{k+1}^j \right\| \leq D_\theta \alpha^{-1} k^\theta$, and the same holds for $\|\sum_{j=k}^t\hat{\Pi}_{k+1}^t\|$;

3. $\| w^\ast  \| \leq \beta/\amin$;

4. $\|Y_t \| \leq 2 \beta/\alpha$;

5. $\|w_t\| \leq  e^{\alpha'(1-t^{1-\theta})} \|w_1\|+ 3\beta/\amin$;

6. $\|r_t\| \leq  e^{\alpha'(1-t^{1-\theta})} \|w_1\|+ 4\beta/\amin$;

7. $\| \chi_t\| \leq  2\amax e^{\alpha'(1-t^{1-\theta})} \|w_1\| + 8\beta/\alpha$.
\end{prop}

\begin{pf} 1. By Lemma \ref{lem:estimates}-1 with $p=1$,
\[ \|\Pi^t_k \|\leq \prod_{i=k}^t \left( 1 - \frac{\alpha}{i^\theta} \right) \leq e^{\alpha'[k^{1-\theta} - (t+1)^{1-\theta}]}. \]
Similar to $\|\hat{\Pi}_k^t\|$.

2. Note that
\[ \left\|\sum_{j=k}^t \Pi_{k+1}^j \right\| \leq \sum_{j=k}^t \prod_{i=k+1}^j \left( 1- \frac{\alpha}{i} \right), \]
whence the result follows from Lemma \ref{lem:estimates}-2.

3. $\|w^\ast\|\leq \|\hat{A}^{-1}\|\|\hat{B}\|\leq \beta/\amin$.

4. $\|Y_t\| = \| A_t w^\ast - B_t \| \leq \amax \beta /\amin + \beta \leq 2 \beta/\alpha$, since $\alpha=\amin/\amax$.

5. By equation (\ref{eq:wt})
\begin{eqnarray*}
w_{t+1} & = & w_t - \gamma_t (A_t w_t - B_t) \\
& = & (I - \gamma_t A_t) w_t + \gamma_t B_t \\
& = & \Pi_1^t w_1 + \sum_{k=1}^t \gamma_k \Pi_{k+1}^t B_k,
\end{eqnarray*}
whence
\begin{eqnarray*}
\|w_{t+1}\| & \leq & \|\Pi_1^t\| \|w_1\| + \beta \sum_{k=1}^{t-1} \gamma_k \|\Pi_{k+1}^t\| \\
& \leq & e^{\alpha'(1-(t+1)^{1-\theta})} \|w_1\|+ \frac{3\beta}{\amin},
\end{eqnarray*}
where the last step follows from part 1 and Lemma \ref{lem:estimates}-3.

6. Since $\| r_t \| \leq \|w_t \| + \|w^\ast\|$, using part 3 and 5 gives the result.

7. Since $\|\chi_t\|=\|(A_t - \hat{A})w_t + \hat{B}-B_t \| \leq 2 \amax\|w_t\| + 2\beta$, apply part 5 and notice that $6\beta/\alpha+2\beta\leq 8\beta/\alpha$,
which gives the result.
\end{pf}

Now we are ready to give the formal proofs of Theorem \ref{thm:robmon} and \ref{thm:average}.

%%%%%%%%%%%%%%%%%%%%%%%%%%%%%%%%%%%%%%%%%%%%%%%%%%%%%%%%%%%%%%%%%%%%%%%%%%%%%%%%

\subsection{Proof of Theorem 3.1}

%%%%%%%%%%%%%%%%%%%%%%%%%%%%%%%%%%%%%%%%%%%%%%%%%%%%%%%%%%%%%%%%%%%%%%%%%%%%%%

\begin{pf}[Proof of Theorem \ref{thm:robmon}] By equation (\ref{eq:total1}) we have
\begin{eqnarray*}
\|r_{t}\| & \leq & \|\hat{\Pi}_1^{t-1} r_1\| + \| \sum_{k=1}^{t-1}\xi_k \|\\
& = & \Err_{init}(t)+\Err_{samp}(t).
\end{eqnarray*}

The upper bound on $\Err_{init}(t)$ follows from Proposition \ref{prop:bounds}-1. For the upper bound on $\Err_{samp}(t)$,
by Proposition \ref{prop:bounds}-6 with $w_1=0$, $\|\chi_k\|\leq 8\beta/\alpha$, whence $\xi_k$ is bounded by
\begin{eqnarray*}
\|\xi_k\| & \leq  & \gamma_k \|\Pi_{k+1,t-1}\| \|\chi_k \| \\
& \leq & \frac{8\beta}{\amin} \left[\frac{1}{k^\theta} \prod_{i=k+1}^{t-1} \left( 1- \frac{\alpha}{i^\theta}\right)\right] = c_k.
\end{eqnarray*}
Applying Pinelis-Hoeffding inequality (Lemma \ref{lem:pinelis}), we obtain
\[ \Prob \left\{ \left\|\sum_{k=1}^{t-1} \xi_k \right\| \geq \epsilon \right\} \leq
2 \exp \left\{-\frac{\epsilon^2}{2\sum_{k=1}^{t-1} c_k^2} \right\}. \]
Let the right hand side equal $\delta$, then
\[
\epsilon^2 = 2 \left(\sum_{k=1}^{t-1} c_k^2\right) \log \frac{2}{\delta} \leq \frac{128\beta^2}{\amin^2} \psi^2_{\theta}(t,\alpha) \log \frac{2}{\delta},
\]
where
\[ \psi^2_\theta(t,\alpha)= \sum_{k=1}^{t-1} \frac{1}{k^{2\theta}} \prod_{i=k+1}^{t-1} \left(1-\frac{\alpha}{i^\theta}\right)^2.\]
We complete the proof by applying the upper bound for $\psi^2_{\theta}(t,\alpha)$ in Lemma \ref{lem:estimates}-4.
\end{pf}

\subsection{Proof of Theorem 3.2}

\begin{pf}[Proof of Theorem 3.2] By equation (\ref{eq:avgtotal1}) we have
\begin{eqnarray*}
\|\bar{r}_{t}\| & \leq & \frac{1}{t}\left\|\left(\sum_{j=0}^{t-1} \hat{\Pi}_1^j \right) r_1\right\| + \|\sum_{k=1}^{t-1}\eta_k\| \\
& = & \Err_{init}(t) + \Err_{samp}(t).
\end{eqnarray*}

By Proposition \ref{prop:bounds}-2 with $k=1$,
\[ \left\|\sum_{j=1}^{t-1} \hat{\Pi}_1^j \right\| \leq \frac{D_\theta}{\alpha}, \]
which gives the first bound.

As to the second bound, by Proposition \ref{prop:bounds}-2 and Proposition \ref{prop:bounds}-6 with $w_1=0$, we obtain
\begin{eqnarray}
\|\eta_k\| & \leq & \frac{\gamma_k}{t} \left\| \sum_{j=k}^{t} \hat{\Pi}_{k+1}^j \right\| \|\chi_k\| \nonumber \\
& \leq & \frac{8 \beta D_\theta \amax}{t \amin^2} = c_\eta. \label{eq:eta}
\end{eqnarray}
Applying Pinelis-Hoeffding inequality (Lemma \ref{lem:pinelis}),
\[ \Prob \left\{ \left\|\sum_{k=1}^{t-1} \eta_k \right\| \geq \epsilon \right\} \leq
2 \exp \left\{-\frac{\epsilon^2}{2\sum_{k=1}^{t-1} c_\eta^2} \right\},  \]
and setting the right hand to be $\delta$, we obtain
\[ \epsilon \leq  \sqrt{2t} c_\eta \log^{1/2} \frac{2}{\delta}.\]
The second bound follows from (\ref{eq:eta}).
\end{pf}

%%%%%%%%%%%%%%%%%%%%%%%%%%%%%%%%%%%%%%%%%%%%%%%%%%%%%%%%%%%%%%%%%%%%%%%%%%%%%%%%%%%%%%%%%%%%%%

\section{Reversed Martingale Decomposition and Proof of Theorem B*}

%%%%%%%%%%%%%%%%%%%%%%%%%%%%%%%%%%%%%%%%%%%%%%%%%%%%%%%%%%%%%%%%%%%%%%%%%%%%%%%%%%%%%%%%%%%%%%

In this section we give a proof of Theorem B*. Note that in the martingale decompostion in Section IV,
$\chi_t=(A_t - \hat{A})r_t + \hat{B}-B_t$ whose variance grows in proportion to $\|r_t\|^2$, whence there is no improvement
replacing Hoeffding's Inequality by Markov's inequality.
However, we may avoid this by turning to the remainder decomposition used in \cite{SmaYao04} where we directly deal with the variance,
$\sigma^2=\E\|Y_t\|^2$. Yet this approach leads to a reversed martingale decomposition for remainders, as we shall see soon.

The following lemma is taken from \cite{SmaYao04}, whose proof is included here for completeness.

\medskip

\begin{lem}\label{lem:oldrt} For all $t\in \N$,
\[ r_{t} = \Pi_1^{t-1} r_1 - \sum_{k=1}^{t-1}\gamma_k  \Pi_{k+1}^{t-1} Y_k. \]
\end{lem}

\begin{pf} Note that
\begin{eqnarray*}
r_{t+1}& = & w_{t+1}- w^\ast \\
& = & w_t - \gamma_t (A_t w_t + B_t) - (I-\gamma_t A_t) w^\ast - \gamma_t A_t w^\ast \\
& = & (I-\gamma_t A_t) r_t - \gamma_t Y_t.
\end{eqnarray*}
The result then follows from induction on $t\in \N$.
\end{pf}

It leads to the following decomposition for
the averaged remainder.

\medskip

\begin{lem} For all $t\in \N$,
\[ \bar{r}_{t} = \frac{1}{t} \left(\sum_{j=1}^{t-1} \Pi_1^j \right) r_1 - \sum_{k=1}^{t-1}\frac{\gamma_k}{t}  \left(\sum_{j=k}^{t-1}\Pi_{k+1}^{j}\right) Y_k. \]
\end{lem}

\medskip

For $k\in \Z$, define
\[ \eta_k =
\left\{
\begin{array}{ll}
\DS \frac{\gamma_k}{t} \left(\sum_{j=k}^{t-1} \Pi_{k+1}^j\right)Y_k, & 1\leq k \leq t; \\
0, & \mbox{otherwise}.
\end{array}
\right.
\]

Recall that a sequence of random variables $(x_k)$ is called a \emph{reversed martingale difference sequence}
if $(x_{-k})$ is a martingale difference sequence. Then $(\eta_k)$
is a reversed martingale difference sequence; since it depends on $\{z_k,\ldots,z_{t-1}\}$ and $\E_{z_k|z_{k+1},\ldots,z_{t-1}}[\eta_k]=0$,
which implies that $(\eta_{-k})$ is a martingale difference sequence.

We will use the following well-known Markov's Inequality.

\medskip

\begin{lem}[Markov] Let $X$ be a nonnegative random variable.
Then for any real number $\epsilon>0$, we have
\[ \Prob\{ X\geq \epsilon \} \leq \frac{\E[X]}{\epsilon}. \]
\end{lem}

\medskip

\begin{thm} \label{thm:B1}
Let $\gamma_t = t^{-\theta}/\amax$ ($\theta\in [0,1)$) and $w_1=0$. Define $\alpha=\amin/\amax \in (0,1]$.
Then the following holds for all $t\in \N$,
\[ \| \bar{w}_t -w^\ast \| \leq \Err_{init}(t) + \Err_{samp}(t). \]
Here
\[ \Err_{init}(t)\leq  D_\theta\left(\frac{1}{\alpha t} \right)\|r_1\| , \]
and with probability at least $1-\delta$ ($\delta\in (0,1)$),
\[ \Err_{samp}(t) \leq  \frac{D_\theta \sigma}{\sqrt{\delta} \amax} \left(\frac{1}{\alpha}\right) \sqrt{\frac{1}{t}}. \]
\end{thm}

\begin{pf}
The initial error bound follows from Proposition \ref{prop:bounds}-2. As to the sample error, note that
\begin{eqnarray*}
\E \|\sum_{k=1}^{t-1} \eta_k\|^2 & \leq & \sum_{k=1}^{t-1} \frac{\gamma^2_k}{t^2}  (\psi^0_\theta(t,k,\alpha))^2 \E \|Y_k\|^2 \\
& \leq & \frac{D_\theta^2 \sigma^2}{\amax^2\alpha^2}t^{-1}.
\end{eqnarray*}
where in the second last step,
\[ \psi^0_\theta(t,k,\alpha) = \sum_{j=k}^{t-1} \prod_{i=k+1}^j \left(1-\frac{\alpha}{i^\theta} \right), \]
and the last is due to Lemma \ref{lem:estimates}-2 and Finiteness Condition-C. The sample error bound then follows from Markov inequality
by regarding $X=\|\sum_{k=1}^{t-1} \eta_k\|^2$.
\end{pf}

\begin{pf}[Proof of Theorem B*]
Setting $\amax=\lambda+C^2_K$, $\amin=\lambda$, $\alpha=\lambda/(\lambda+C^2_K)$, and $\sigma=\sigma_\lambda$, the result follows from Theorem \ref{thm:B1}.
\end{pf}

The following proposition gives an estimate of $\sigma_\lambda$ via $\|f^\ast_\lambda\|_K$.

\medskip

\begin{prop} \label{prop:varbnd}
1. $ \|f^\ast_\lambda\|_K \leq M_\rho /\sqrt{\lambda}$;

2. $\sigma_\lambda \leq M_\rho(1+\sqrt{\lambda}+\lambda)/\sqrt{\lambda}$.
\end{prop}

\begin{pf}
1. Note that
\[ f^\ast_\lambda = \arg \min_{f\in \H_K} \|f - f_\rho\|_{\L^2_\rho}^2 + \lambda \|f\|^2_K. \]
Taking $f=0$, we have
\[ \|f^\ast_\lambda - f_\rho\|_{\L^2_\rho}^2 + \lambda \|f^\ast_\lambda\|^2_K \leq \|f_\rho\|_{\L^2_\rho}^2 \leq M_\rho^2, \]
which leads to the result.

2. By definition
\begin{eqnarray*}
\sigma_\lambda & = & \sqrt{\E\|(y-f^\ast_\lambda(x))-\lambda f^\ast_\lambda\|_K^2} \\
& \leq & \E \|y-f^\ast_\lambda(x)\|_K + \lambda \|f^\ast_\lambda\|_K \\
& \leq &  M_\rho+ \frac{M_\rho}{\sqrt{\lambda}} + M_\rho \sqrt{\lambda}
\end{eqnarray*}
which ends the proof.
\end{pf}

\section{Conclusion and Open Problems}

In this paper, we have shown by probabilistic upper bounds that a two-stage online learning algorithm,
the stochastic approximation of the gradient descent method followed by an averaging process, can achieve
the almost sure convergence with an optimal asymptotic rate with respect to the sample size, as good as ``batch learning''.
Moreover considering the regularization parameter and confidence, the best results obtained so far are,
$O(\lambda^{-2} t^{-1/2} \log^{1/2} 1/\delta)$ (Theorem B) or $O(\lambda^{-3/2} t^{-1/2} \delta^{-1/2})$ (Remark 2.4 for Theorem B*)
under the conditions in this paper, and $O(\lambda^{-1} t^{-1/2} \delta^{-1/2})$ (Theorem B*) under a strong condition on variance.

Thus it is still an open problem, \emph{if we can achieve $O(\lambda^{-1} t^{-1/2} \log^{1/2}1/\delta)$, the optimal
rate known in ``batch learning''}.

%%%%%%%%%%%%%%%%%%%%%%%%%%%%%%%%%%%%%%%%%%%%%%%%%%%%%%%%%%%%%%%%%%%%%%%%%%%%%%%%%

\section*{Appendix A: Some Estimates based on Gamma Function}\

%%%%%%%%%%%%%%%%%%%%%%%%%%%%%%%%%%%%%%%%%%%%%%%%%%%%%%%%%%%%%%%%%%%%%%%%%%%%%%%%%
\renewcommand{\thesection}{A}
%\setcounter{section}{8}
\setcounter{equation}{0}
\setcounter{thm}{0}
\renewcommand{\thethm}{A.\arabic{thm}}
\renewcommand{\theequation}{A-\arabic{equation}}

\begin{lem} \label{lem:key1} Let $\theta\in [0,1)$, $a>0$ and $t\geq 2$. Then for any $\tau\in \R$,
\[ e^{-a t^{1-\theta}} \int_1^t x^{-(\theta+\tau)} e^{a x^{1-\theta}} d x = O(t^{-\tau}). \]
In fact, if $\tau\geq 0$,
\[  A_{\theta,a} t^{-\tau} \leq e^{-a t^{1-\theta}} \int_1^t x^{-(\theta+\tau)} e^{a x^{1-\theta}} d x \leq A'_{\theta,\tau,a} t^{-\tau}, \]
and if $\tau<0$,
\[  B_{\theta,\tau,a} t^{-\tau} \leq e^{-a t^{1-\theta}} \int_1^t x^{-(\theta+\tau)} e^{a x^{1-\theta}} d x  \leq B'_{\theta,a}t^{-\tau}. \]
Here
\[ A_{\theta,a} =\frac{1-e^{-a(2^{1-\theta}-1)}}{a(1-\theta)} , \]
\[ A'_{\theta,\tau,a} =
\frac{2^{\tau/(1-\theta)}}{a(1-\theta)} \left(1+a^{-\tau/(1-\theta)} \Gamma \left( \frac{1+\tau-\theta}{1-\theta}\right)\right), \]
\[ B_{\theta,\tau,a} = \frac{2^{\tau/(1-\theta)}(1-e^{-a})}{a(1-\theta)},\ \ \mbox{and}\ \  B'_{\theta,a} = \frac{1}{  a(1-\theta)}. \]
\end{lem}

\begin{pf} Let $y=t^{1-\theta} - x^{1-\theta}$. Then
\begin{eqnarray}
& & e^{-at^{1-\theta}} \int_1^t x^{-(\theta+\tau)} e^{ax^{1-\theta}} d x  \label{eq:identity}\\
& = & \frac{1}{1-\theta} \int_0^{t^{1-\theta}-1} (t^{1-\theta}-y)^{-\tau/(1-\theta)} e^{-a y} d y \nonumber \\
& = & \frac{t^{-\tau}}{1-\theta} \int_0^{t^{1-\theta}-1} \left(1-\frac{y}{t^{1-\theta}}\right)^{-\tau/(1-\theta)} e^{-a y} d y. \nonumber
\end{eqnarray}

1. (\emph{For $A_{\theta,a}$})
For $0\leq y \leq t^{1-\theta}-1$, $\DS 1-\frac{y}{t^{1-\theta}}\leq 1$. Thus if $\tau\geq 0$, equation (\ref{eq:identity}) has
\begin{eqnarray*}
r.h.s. & \geq & \frac{t^{-\tau}}{1-\theta} \int_0^{t^{1-\theta}-1} e^{-a y} d y \\
& = & \frac{t^{-\tau}}{a(1-\theta)}(1-e^{-a(t^{1-\theta}-1)}) \\
& \geq & \frac{t^{-\tau}}{a(1-\theta)}(1-e^{-a(2^{1-\theta}-1)}), \ \ \ \ \ \mbox{$t\geq 2$}.
\end{eqnarray*}

2. (\emph{For $B'_{\theta,a}$}) Similarly if $\tau<0$, equation (\ref{eq:identity}) has
\begin{eqnarray*}
r.h.s. & \leq & \frac{t^{-\tau}}{1-\theta} \int_0^{t^{1-\theta}-1} e^{-a y} d y \\
&  = & \frac{t^{-\tau}}{a(1-\theta)}(1-e^{-a(t^{1-\theta}-1)}) \\
& \leq & \frac{t^{-\tau}}{a(1-\theta)}.
\end{eqnarray*}

3. (\emph{For $A'_{\theta,\tau,a}$})
Note that for $0\leq y \leq t^{1-\theta}-1$, $s=y/t^{1-\theta}\in (0,1)$, whence
\begin{eqnarray} \label{eq:B}
\frac{1}{1-s} & = & 1 + \sum_{n=1}^\infty s^n
= 1 + \frac{s}{1 - s} \nonumber \\
& \leq & 1 + \frac{y/t^{1-\theta}}{1 - (t^{1-\theta}-1)/t^{1-\theta}} \nonumber \\
& = & 1 + y.
\end{eqnarray}
Thus for $\tau>0$ the right hand side of (\ref{eq:identity}) is bounded by
\begin{eqnarray*}
r.h.s.
& \leq &  \frac{t^{-\tau}}{1-\theta}\int_0^{t^{1-\theta}-1} (1+y)^{\tau/(1-\theta)} e^{-a y} d y  \\
& \leq & \frac{2^{\tau/(1-\theta)}t^{-\tau}}{1-\theta}\int_0^1  e^{-a y} d y  \\
& & \ \ + \frac{2^{\tau/(1-\theta)}t^{-\tau}}{1-\theta} \int_1^{t^{1-\theta}-1} y^{\tau/(1-\theta)} e^{-a y} d y \\
& \leq & \frac{2^{\tau/(1-\theta)}}{1-\theta} t^{-\tau}  \left\{\int_0^\infty  e^{-a y} d y \right. \\
& & \ \ + \left.\int_0^\infty y^{(1+\tau-\theta)/(1-\theta)-1} e^{-a y} d y \right\} \\
& \leq & \frac{2^{\tau/(1-\theta)}}{a(1-\theta)} \left(1 \right.\\
& & \ \ \left. +a^{-\tau/(1-\theta)} \Gamma \left( \frac{1+\tau-\theta}{1-\theta}\right)\right) t^{-\tau}.
\end{eqnarray*}

4. (\emph{For $B_{\theta,\tau,a}$})
By equation (\ref{eq:B}), for $\tau<0$ the right hand side of (\ref{eq:identity}) is bounded by
\begin{eqnarray*}
r.h.s.
& \geq &  \frac{t^{-\tau}}{1-\theta}\int_0^{t^{1-\theta}-1} (1+y)^{\tau/(1-\theta)} e^{-a y} d y  \\
& \geq & \frac{2^{\tau/(1-\theta)}t^{-\tau}}{1-\theta}\int_0^1  e^{-a y} d y  \\
& & \ \ + \frac{2^{\tau/(1-\theta)}t^{-\tau}}{1-\theta} \int_0^{t^{1-\theta}-1} y^{\tau/(1-\theta)} e^{-a y} d y \\
& \geq & \frac{2^{\tau/(1-\theta)}t^{-\tau}}{1-\theta}\int_0^1  e^{-a y} d y \\
& = & \frac{2^{\tau/(1-\theta)}(1-e^{-a})}{a(1-\theta)} t^{-\tau}.
\end{eqnarray*}
This completes the proof.
\end{pf}


\begin{lem} \label{lem:key2} Let $\theta \in (0,1)$. Then
\[ C_\theta t^\theta \leq e^{t^{1-\theta}} \int_t^\infty e^{-x^{1-\theta}} d x \leq C'_\theta t^{\theta}, \]
where $C_\theta = 1/(1-\theta)$ and $C'_\theta = 2^{\theta/(1-\theta)} (1+\Gamma(1/(1-\theta)))$.
\end{lem}

\begin{pf} 1. \emph{Lower bound.} Consider the continuous function
\[ f(x) = x^{1-\theta}. \]
By the mean value theorem, when $x\geq t> 0$, there exists a $\zeta\in (t,x)$ such that
\begin{eqnarray*}
 f(t)-f(x) & = & f'(\zeta) (t-x) = (1-\theta)\zeta^{-\theta}(t-x) \\
 & \geq & -(1-\theta) x^{-\theta} (x-t),
\end{eqnarray*}
whence
\begin{eqnarray*}
& & e^{t^{1-\theta}} \int_t^\infty e^{- x^{1-\theta}} dx  \\
& \geq & \int_t^\infty e^{-(1-\theta) t^{-\theta} (x-t)}dx  \\
& = & e^{(1-\theta) t^{1-\theta}} \int_t^\infty e^{-(1-\theta) t^{-\theta} x}d x \\
& = & \frac{t^\theta}{1-\theta} .
\end{eqnarray*}

2. \emph{Upper bound.} It is enough to show that for $x\geq 1$ and
$a\geq 1$,
\begin{equation} \label{eq:lem2}
\Gamma(a,x) \leq  G_a e^{-x} x^{a-1}, \ \ \ \ \
G_a=2^{a-1}(1+\Gamma(a)).
\end{equation}
If this is true, the result follows from setting $a=1/(1-\theta)\geq 1$, $C'_\theta=G_{1/(1-\theta)}$, and replacing $x$ by $t^{1-\theta}$.

To show (\ref{eq:lem2}), by setting $s=x+\tau$,
\begin{eqnarray*}
\Gamma(a,x) & = & \int_x^\infty s^{a-1} e^{-s} d s  \\
& = &  x^{a-1}e^{-x} \int_0^\infty e^{-\tau} (1+\tau/x)^{a-1} d \tau, \\
& \leq & x^{a-1} e^{-x} \int_0^\infty e^{-\tau} (1+\tau)^{a-1} d \tau, \\
& & \ \ \ \ \ \mbox{(by $x\geq 1$ and $a\geq 1$),} \\
& \leq & x^{a-1} e^{-x} \left\{ 2^{a-1} \int_0^1 e^{-\tau}d\tau \right. \\
& & \ \ \left.+ 2^{a-1} \int_1^\infty e^{-\tau} \tau^{a-1} d \tau \right \} \\
& \leq & 2^{a-1}(1+\Gamma(a)) x^{a-1} e^{-x}.
\end{eqnarray*}
This completes the proof.
\end{pf}

%%%%%%%%%%%%%%%%%%%%%%%%%%%%%%%%%%%%%%%%%%%%%%%%%%%%%%%%%%%%%%%%%%%%%%%%%%%%
%
% Estimate Lemma
%
%%%%%%%%%%%%%%%%%%%%%%%%%%%%%%%%%%%%%%%%%%%%%%%%%%%%%%%%%%%%%%%%%%%%%%%%%%%%

\begin{lem} \label{lem:estimates}
1. For $\alpha\in (0,1]$, $p>0$, and $\theta\in [0,1]$,
\begin{eqnarray*}
& & \prod_{i=k}^{t-1} \left(1-\frac{\alpha}{i^\theta}\right)^p \\
& \leq & \DS
\left\{
\begin{array}{ll}
\DS \exp\left\{\frac{\alpha p}{1-\theta} (k^{1-\theta}-t^{1-\theta})\right\}, & \theta\in [0,1) \\
\DS \left(\frac{k}{t}\right)^{\alpha p}, & \theta = 1
\end{array}
\right.
\end{eqnarray*}

2. For $\alpha\in (0,1]$, $\theta\in [0,1)$, and all $t\in \N$,
\[ \psi^0_\theta(t,k,\alpha):=\sum_{j=k}^{t-1} \prod_{i=k+1}^j \left(1 - \frac{\alpha}{i^\theta} \right)
\leq \frac{D_\theta}{\alpha} k^{\theta};
\]

3. For $\alpha\in (0,1]$, $\theta\in [0,1]$, and all
$t\in \N$,
\[ \psi^1_\theta(t,\alpha):=\sum_{k=1}^{t-1} \frac{1}{k^\theta} \prod_{i=k+1}^{t-1} \left(1-\frac{\alpha}{i^\theta}\right) \leq \frac{2}{\alpha}; \]

4. For $\alpha\in (0,1]$, $\theta\in [0,1)$, and all $t\in \N$,
\begin{eqnarray*}
 \psi^2_\theta(t,\alpha) & := & \sum_{k=1}^{t-1} \frac{1}{k^{2\theta}} \prod_{i=k+1}^{t-1} \left(1-\frac{\alpha}{i^\theta}\right)^2 \\
 & \leq & 2 D_\theta \left(\frac{1}{\alpha} \right)^{\frac{1}{1-\theta}} \left(\frac{1}{t}\right)^{\theta}.
\end{eqnarray*}
\end{lem}

\begin{pf}
The following fact will be used repeatedly in the proof,
\begin{equation}\label{eq:logubd}
\ln (1+x) \leq x, \ \ \ \ \mbox{for all $x>-1$}.
\end{equation}

1. By the inequality (\ref{eq:logubd}), we have for $\theta \in [0,1]$,
\[ \ln \left(1 - \frac{\alpha}{i^\theta}\right)^p \leq -\frac{\alpha p}{i^\theta}. \]
Thus
\begin{eqnarray*}
\sum_{i=k}^{t-1} \ln \left( 1-\frac{\alpha}{i^\theta} \right)^p & \leq & - \alpha p \sum_{i=k}^{t-1} \frac{1}{i^\theta} \\
 & \leq & -\alpha p\int_{k}^{t} \frac{1}{x^\theta} d x
\end{eqnarray*}
which equals
\[\frac{\alpha p}{1-\theta} \left(k^{1-\theta}-t^{1-\theta} \right), \]
if $\theta\in [0,1)$, and
\[ \ln \left(\frac{k}{t}\right)^{\alpha p}, \]
if $\theta=1$. Taking the exponential gives the inequality.

2. Notice that
\[ \psi^0_\theta(t,\alpha)=1+ \sum_{j=k+1}^{t-1} \prod_{i=k+1}^j \left(1 - \frac{\alpha}{i^\theta} \right)
\leq \frac{D_\theta}{\alpha} k^{\theta}. \]
As $1\leq k^\theta/\alpha$, it suffices to prove the second term is bounded by $(D_{\theta}-1)k^\theta/\alpha$.

Let $\alpha'=\alpha/(1-\theta)$. Using part 1 with $p=1$, we obtain
\begin{eqnarray*}
\prod_{i=k+1}^j \left(1 - \frac{\alpha}{i^\theta} \right) & \leq &  e^{\alpha'[(k+1)^{1-\theta} - (j+1)^{1-\theta}]},
\end{eqnarray*}
whence
\begin{eqnarray*}
& & \sum_{j=k+1}^{t-1} \prod_{i=k+1}^j \left(1 - \frac{\alpha}{i^\theta} \right) \\
& \leq & \sum_{j=k+1}^{t-1} e^{\alpha'[(k+1)^{1-\theta} - (j+1)^{1-\theta}]} \\
& \leq & e^{\alpha'(k+1)^{1-\theta}}\int_{k+1}^\infty e^{- \alpha' x^{1-\theta}} d x \\
& \leq & (\alpha')^{-1/(1-\theta)} C_\theta' [(\alpha')^{1/(1-\theta)} (k+1)]^{\theta}, \\
& & \ \ \ \ \ \mbox{(by Lemma \ref{lem:key2})} \\
& \leq & 2^{\theta/(1-\theta)} \left\{ 1+\Gamma\left(\frac{1}{1-\theta}\right)\right\} \left(\frac{1}{\alpha}\right)k^\theta \\
& = & \frac{D_\theta-1}{\alpha} k^\theta,
\end{eqnarray*}
as desired.

3. Notice that
\[ \psi^1_\theta(t,\alpha)= \frac{1}{(t-1)^\theta} + \sum_{k=1}^{t-2} \frac{1}{k^\theta}\prod_{i=k+1}^{t-1} \left(1-\frac{\alpha}{i^\theta}\right). \]
The first term is bounded by $1/\alpha$ for $t>1$. It is sufficient to show the second term is bounded by $2/\alpha$. To see this, we consider seperately
two cases $\theta\in [0,1)$ and $\theta=1$.

If $\theta\in [0,1)$, from part 1 with $p=1$, we have
\begin{eqnarray*}
& & \sum_{k=1}^{t-2} \frac{1}{k^\theta}\prod_{i=k+1}^{t-1} \left(1-\frac{\alpha}{i^\theta}\right) \\
& \leq & e^{-\frac{\alpha}{1-\theta}t^{1-\theta}} \sum_{k=1}^{t-2} \frac{1}{k^\theta}e^{\frac{\alpha}{1-\theta} (k+1)^{1-\theta}}
\end{eqnarray*}
where
\begin{eqnarray*}
& & \sum_{k=1}^{t-2} \frac{1}{k^\theta}e^{\frac{\alpha}{1-\theta} (k+1)^{1-\theta}} \\ & \leq & 2^\theta \sum_{k=1}^{t-2}
\left(\frac{1}{k+1}\right)^\theta e^{\frac{\alpha}{1-\theta} (k+1)^{1-\theta}} \\
& \leq & 2 \int_1^{t}  e^{\frac{\alpha}{1-\theta} x^{1-\theta}} x^{-\theta} d x \\
& \leq & \frac{2}{\alpha} e^{\frac{\alpha}{1-\theta}t^{1-\theta}},
\end{eqnarray*}
as desired.


If $\theta=1$, from part 1 ($p=1$),
\begin{eqnarray*}
& & \sum_{k=1}^{t-2} \frac{1}{k} \prod_{i=k+1}^{t-1} \left( 1 - \frac{\alpha}{i} \right) \\
& \leq & \sum_{k=1}^{t-2} \frac{1}{k}\left(\frac{k+1}{t}\right)^\alpha \\
& \leq & \frac{2}{t^\alpha} \sum_{k=1}^{t-2} \frac{(k+1)^\alpha}{k+1} \\
& \leq  & \frac{2}{t^\alpha} \int_1^{t} x^{\alpha-1} d x,
\end{eqnarray*}
where if $\alpha=1$,
\[ \frac{2}{t^\alpha} \int_1^{t} x^{\alpha-1} d x = 2; \]
and if $0<\alpha<1$,
\[ \frac{2}{t^\alpha} \int_1^{t} x^{\alpha-1} d x = \frac{2}{\alpha}\left(\frac{t^\alpha-1}{t^\alpha}\right) \leq \frac{2}{\alpha}.\]

4. Notice that
\[ \psi^2_\theta(t,\alpha) = \frac{1}{(t-1)^{2\theta}} + \sum_{k=1}^{t-2} \prod_{i=k+1}^{t-1}\left(1-\frac{\alpha}{i^\theta}\right)^2. \]
The first term is bounded by
\[ \frac{1}{(t-1)^{2\theta}} \leq \frac{2^{2\theta}}{t^{2\theta}}. \]

Below we are going to give an upper bound on the second term. Let $\alpha'=\alpha/(1-\theta)$. By part 1 with $p=2$,
\[ \prod_{i=k+1}^{t-1} \left(1-\frac{\alpha}{i^\theta}\right)^2 \leq \exp \{ 2 \alpha' [(k+1)^{1-\theta} - t^{1-\theta} ] \}. \]
Then
\begin{eqnarray*}
& & \DS \sum_{k=1}^{t-2} \frac{1}{k^{2\theta}} \prod_{i=k+1}^{t-1} \left(1-\frac{\alpha}{i^\theta}\right)^2 \\
& \leq &\DS 2^{2\theta} \sum_{k=1}^{t-2} \frac{1}{(k+1)^{2\theta}} e^{2 \alpha' [(k+1)^{1-\theta} - t^{1-\theta} ]} \\
& \leq &\DS 2^{2\theta} e^{-2\alpha' t^{1-\theta}} \int_1^{t} x^{-2\theta} e^{2 \alpha' x^{1-\theta}} d x \\
& \leq &\DS \frac{2^{\theta/(1-\theta)+2\theta-1}}{\alpha'(1-\theta)} \left\{1 +(2\alpha')^{-\theta/(1-\theta)}\right. \\
& & \DS \ \ \left. \cdot\Gamma\left(\frac{1}{1-\theta}\right) \right\}t^{-\theta}  \\
& & \ \ \ \ \ \mbox{(by Lemma \ref{lem:key1} with $\tau=\theta$)}.
\end{eqnarray*}
where by
\[ (2\alpha')^{-\theta/(1-\theta)} \leq \left(\frac{1}{\alpha}\right)^{\theta/(1-\theta)} ,  \]
we obtain
\[
r.h.s. \leq 2^{\theta/(1-\theta)+2\theta-1}\left(1+ \Gamma\left(\frac{1}{1-\theta}\right) \right) \left(\frac{1}{\alpha}\right)^{1/(1-\theta)}.
\]

Combining two terms together, we obtain
\begin{eqnarray*}
& & \psi^2_\theta(t,\alpha) \\
& \leq & 2^{2\theta} t^{-2\theta} + 2^{\theta/(1-\theta)+2\theta-1} \left\{1 + \Gamma\left(\frac{1}{1-\theta}\right) \right\} \\
& & \ \  \cdot \left(\frac{1}{\alpha}\right)^{1/(1-\theta)} t^{-\theta}\\
& \leq & 2^{\theta}\left\{ \alpha^{1/(1-\theta)}(2/t)^{\theta} + 2^{\theta/(1-\theta)} \left[1 + \Gamma\left(\frac{1}{1-\theta}\right)\right] \right\} \\
& & \ \  \cdot \left(\frac{1}{\alpha}\right)^{1/(1-\theta)} t^{-\theta}\\
& \leq & 2D_\theta \left(\frac{1}{\alpha}\right)^{1/(1-\theta)} t^{-\theta},
\end{eqnarray*}
for $t\geq 2$, as desired. For $t=1$, we complete the proof by noting that $\psi^2_\theta(1,\alpha)=0$.
\end{pf}

\section*{Acknowledgement}

The author would like to acknowledge Jia Yu for her suggestion on using the gamma function which eventually develops into Lemma \ref{lem:key1} and \ref{lem:key2};
Pierre Tarres for pointing out recent convergence results on Robbins-Monro procedure and many helpful suggestions on improving early drafts; Peter Bartlett,
Andrea Caponnetto, Adam Klai, Ha Quang Minh, Tommy Poggio, Lorenzo Rosasco, Ding-Xuan Zhou for many helpful discussions; and especially Steve Smale,
without whom this paper never comes
into reality.


\bibliographystyle{IEEEtran}
\bibliography{yao}

\begin{thebibliography}{10}
\providecommand{\url}[1]{#1}
\csname url@rmstyle\endcsname
\providecommand{\newblock}{\relax}
\providecommand{\bibinfo}[2]{#2}
\providecommand\BIBentrySTDinterwordspacing{\spaceskip=0pt\relax}
\providecommand\BIBentryALTinterwordstretchfactor{4}
\providecommand\BIBentryALTinterwordspacing{\spaceskip=\fontdimen2\font plus
\BIBentryALTinterwordstretchfactor\fontdimen3\font minus
  \fontdimen4\font\relax}
\providecommand\BIBforeignlanguage[2]{{%
\expandafter\ifx\csname l@#1\endcsname\relax
\typeout{** WARNING: IEEEtran.bst: No hyphenation pattern has been}%
\typeout{** loaded for the language `#1'. Using the pattern for}%
\typeout{** the default language instead.}%
\else
\language=\csname l@#1\endcsname
\fi
#2}}

\bibitem{SmaYao04}
S.~Smale and Y.~Yao, ``Online learning algorithms,'' \emph{Foundation of
  Computational Mathematics}, 2004, submitted.

\bibitem{CesConGen04}
N.~Cesa-Bianchi, A.~Conconi, and C.~Gentile, ``On the generalization ability of
  on-line learning algorithms,'' \emph{IEEE Transactions on Information
  Theory}, vol.~50, no.~9, pp. 2050--2057, 2004.

\bibitem{CucSma02}
F.~Cucker and S.~Smale, ``On the mathematical foundations of learning,''
  \emph{Bull. of the Amer. Math. Soc.}, vol.~29, no.~1, pp. 1--49, 2002.

\bibitem{EvgPonPog99}
T.~Evgeniou, M.~Pontil, and T.~Poggio, ``Regularization networks and support
  vector machines,'' \emph{Advances of Computational Mathematics}, vol.~13,
  no.~1, pp. 1--50, 1999.

\bibitem{PogSma03}
T.~Poggio and S.~Smale, ``The mathematics of learning: Dealing with data,''
  \emph{Notices of the AMS}, vol.~50, no.~5, pp. 537--544, 2003.

\bibitem{Zhou03}
D.-X. Zhou, ``Capacity of reproducing kernel spaces in learning theory,''
  \emph{IEEE Transactions on Information Theory}, vol.~49, no.~7, pp.
  1743--1752, 2003.

\bibitem{SmaZho-ShannonIII}
S.~Smale and D.-X. Zhou, ``Learning theory estimates via integral operators and
  their approximations,'' \emph{to appear}, 2005.

\bibitem{CapDev05}
A.~Caponnetto and E.~D. Vito, ``Fast rates for regularized least squares
  algorithm,'' \emph{CBCL Paper/AI Memo}, 2005, preprint.

\bibitem{DevRosCap04}
E.~De~Vito, L.~Rosasco, A.~Caponnetto, U.~D. Giovannini, and F.~Odone,
  ``Learning from examples as an inverse problem,'' \emph{Journal of Machine
  Learning Research}, 2004, to appear.

\bibitem{EngHanNeu00}
H.~W. Engl, M.~Hanke, and A.~Neubauer, \emph{Regularization of Inverse
  Problems}.\hskip 1em plus 0.5em minus 0.4em\relax Kluwer Academic Publishers,
  2000.

\bibitem{RobMon51}
H.~Robbins and S.~Monro, ``A stochastic approximation method,'' \emph{The
  Annals of Mathematical Statistics}, vol.~22, no.~3, pp. 400--407, 1951.

\bibitem{KieWol52}
J.~Kiefer and J.~Wolfowitz, ``Stochastic estimation of the maximum of a
  regression function,'' \emph{The Annals of Mathematical Statistics}, vol.~23,
  pp. 462--466, 1952.

\bibitem{RobSie71}
H.~Robbins and D.~Siegmund, ``A convergence theorem for nonnegative almost
  supermartingales and some applications,'' in \emph{Optimizing Methods in
  Statistics}, J.~S. Rustagi, Ed.\hskip 1em plus 0.5em minus 0.4em\relax
  Academic Press, New York, 1971, pp. 233--257.

\bibitem{Duflo97}
M.~Duflo, ``Cibles atteignables avec une probabilit\'{e} positive d'apr\'{e}s
  m. benaim,'' \emph{Unpublished manuscript}, 1997.

\bibitem{Benaim99}
M.~Bena\"{\i}m, ``Dynamics of stochastic approximations,'' in \emph{Le
  Seminaire de Probabilites, Lectures Notes in Mathematics, Vol 1709}.\hskip
  1em plus 0.5em minus 0.4em\relax Springer-Verlag, 1999, pp. 1--68.

\bibitem{Duflo96}
M.~Duflo, \emph{Algorithmes Stochastiques}.\hskip 1em plus 0.5em minus
  0.4em\relax Berlin, Heidelberg: Springer-Verlag, 1996.

\bibitem{KusYin03}
H.~J. Kushner and G.~G. Yin, \emph{Stochastic Approximations and Recursive
  Algorithms and Applications}.\hskip 1em plus 0.5em minus 0.4em\relax Berlin,
  Heidelberg: Springer-Verlag, 2003.

\bibitem{Pinelis94}
I.~Pinelis, ``Optimum bounds for the distributions of martingales in banach
  spaces,'' \emph{The Annals of Probability}, vol.~22, no.~4, pp. 1679--1706,
  1994.

\bibitem{WidHof60}
B.~Widrow and M.~Hoff, ``Adaptive switching circuits,'' \emph{IRE WESCON
  Convention Record}, no.~4, pp. 96--104, 1960.

\bibitem{CriSha00}
N.~Cristianini and J.~Shawe-Taylor, \emph{An Introduction to Support Vector
  Machines and Other Kernel-based Learning Methods}.\hskip 1em plus 0.5em minus
  0.4em\relax Cambridge Unversity Press, 2000.

\bibitem{Polyak90}
B.~T. Polyak, ``New method of stochastic approximation type,'' \emph{Automation
  and Remote Control}, vol.~51, pp. 937--946, 1990.

\bibitem{Ruppert88}
D.~Ruppert, ``Efficient estimators from a slowly convergent robbins-monro
  procedure,'' Technical Report 781, School of Operations Research and
  Industrial Engineering, Cornell University, Tech. Rep., 1988.

\bibitem{Daubechies92}
I.~Daubechies, \emph{Ten Lectures on Wavelets}.\hskip 1em plus 0.5em minus
  0.4em\relax Philadelphia, PA: Society for Industrial and Applied Mathematics
  (SIAM), 1992.

\bibitem{SmaZho-ShannonI}
S.~Smale and D.-X. Zhou, ``Shannon sampling and function reconstruction from
  point values,'' \emph{Bull. of the Amer. Math. Soc.}, vol.~41, no.~3, pp.
  279--305, 2004.

\bibitem{PolJud92}
B.~T. Polyak and A.~B. Juditsky, ``Acceleration of stochastic approximation by
  averaging,'' \emph{SIAM Journal of Control and Optimization}, vol.~30, no.~4,
  pp. 835--855, 1992.

\bibitem{KonTsi04}
V.~R. Konda and J.~N. Tsitsiklis, ``Convergence rate of linear two-time-scale
  stochastic approximation,'' \emph{The Annals of Applied Probability},
  vol.~14, no.~2, pp. 796--819, 2004.

\bibitem{CucSma02b}
F.~Cucker and S.~Smale, ``Best choices for regularization parameters in
  learning theory,'' \emph{Foundations Comput. Math.}, vol.~2, no.~4, pp.
  413--428, 2002.

\bibitem{DevCapRos04}
E.~De~Vito, A.~Caponnetto, and L.~Rosasco, ``Model selection for regularized
  least-squares algorithm in learning theory,'' \emph{Foundations of
  Computational Mathematics}, 2004, to appear.

\bibitem{BouEli02}
O.~Bousquet and A.~Elisseeff, ``Stability and generalization,'' \emph{Journal
  of Machine Learning Research}, no.~2, pp. 499--526, 2002.

\bibitem{Zhang03}
T.~Zhang, ``Leave-one-out bounds for kernel methods,'' \emph{Neural
  Computation}, vol.~15, pp. 1397--1437, 2003.

\bibitem{Pinelis92}
I.~Pinelis, ``An approach to inequalities for the distributions of
  infinite-dimensional martingales,'' in \emph{Probability in Banach Spaces, 8:
  Proceedings of the Eighth International Conference}, R.~M. Dudley, M.~G.
  Hahn, and J.~Kuelbs, Eds., 1992, pp. 128--134.

\end{thebibliography}


\end{document}
