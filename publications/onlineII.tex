\documentclass[twoside,11pt]{amsart}
\usepackage{amsmath}
%\usepackage{amsart}
\usepackage{amsthm}
\usepackage{amssymb}
\usepackage{mathrsfs}

%\usepackage{theorem}
\usepackage{chicagoc}
%\include{mynote}

\newenvironment{mthm}[1][Main Theorem]{\medskip \noindent {\bf #1.}
\begin{em}}{\end{em}\medskip}

\theoremstyle{theorem}
\newtheorem*{mainthm}{Main Theorem}
\newtheorem*{thma}{Theorem A}
\newtheorem*{thmb}{Theorem B}
\newtheorem*{thmc}{Theorem C}

\newtheorem{thm}{Theorem}[section]
\newtheorem{conj}{Conjecture}[section]
\newtheorem{lem}[thm]{Lemma}
\newtheorem{lema}{Lemma A}
\newtheorem{lemb}{Lemma B}
\newtheorem{thmaa}{Theorem A}
\newtheorem{thmbb}{Theorem B}
\newtheorem{prop}[thm]{Proposition}
\newtheorem{cor}[thm]{Corollary}
\newtheorem{cora}[thmaa]{Corollary A}
\newtheorem{corb}[thmbb]{Corollary B}

\theoremstyle{definition}
\newtheorem*{defn}{Definition}
\newtheorem*{defa}{Definition A}
\newtheorem{exmp}{Example}[section]

\theoremstyle{remark}
\newtheorem{rem}[thm]{Remark}
\newtheorem*{rem*}{Remark}
\newtheorem*{note}{Note}
\newtheorem*{exer}{Exercise}

\setlength{\oddsidemargin}{0.25 in}
\setlength{\evensidemargin}{-0.25 in}
\setlength{\topmargin}{-0.6 in}
\setlength{\textwidth}{6.5 in}
\setlength{\textheight}{8.5 in}
\setlength{\headsep}{0.75 in}
%\setlength{\parindent}{0 in}
\setlength{\parskip}{0.1 in}

\newcommand{\DS}{\displaystyle}

\def\Q{{\mathbb Q}}        % rationals
\def\Z{{\mathbb Z}}        % integers
\def\R{{\mathbb R}}        % reals
\def\Rn{{\R^{n}}}          % product of n copies of reals
\def\<{{\langle}}
\def\>{{\rangle}}
\def\P{{\mathbb P}}        % probability
\def\E{{\mathbb E}}        % expectation
\def\1{{\mathbf 1}}        % indicator
\def\var{{\mathop{\mathbf Var}}}    % variance

\def\L{{\mathscr L}}
\def\L2{{\mathscr L}^2_{\rho_\X}}
\def\A{\hat{A}}
\def\b{\hat{b}}
\def\w{\hat{w}}
\def\M{M_\rho}

\def\C{{\mathscr C}}
\def\H{{\mathscr H}}
\def\W{{\mathscr W}}
\def\P{{\mathscr P}}
\def\X{{\mathscr X}}
\def\Y{{\mathscr Y}}
\def\Z{{\mathscr Z}}
\def\F{{\mathcal F}}
\def\f{f^\prime}
\def\PPi{{\hat{\Pi}}}
\def\T{{\mathscr T}}
\def\Err{{\mathscr E}}
\def\N{{\mathbb N}}
\def\z{{\mathbf z}}
\def\Prob{{\bf Prob}}
\def\Proj{{\rm Proj}}
\def\grad{{\rm grad}}
\def\ess{{\rm ess}}
\def\supp{{\rm supp}}
\def\span{{\rm span}}
\def\O{{\mathscr O}}
\def\amax{{\overline{\alpha}}}
\def\amin{{\underline{\alpha}} }
\def\i{{\bf i}}
\def\j{{\bf j}}
\def\t{t_0}

\def\al{\alpha}
\def\be{\beta}
\def\la{\lambda}
\def\De{\Delta}
\def\de{\delta}
\def\ka{\kappa}
\def\ga{\gamma}
\def\ze{\zeta}
\def\eps{\epsilon}

\def\lf{\lfloor}
\def\rf{\rfloor}
\def\lc{\lceil}
\def\rc{\rceil}

\def\st{\star}

\newcommand{\Es}{\mathbb{E}}

\begin{document}

\title{Online Learning as Stochastic Approximations of Regularization Paths}
\thanks{This work was supported by NSF grant 0325113.}

\author{Pierre Tarr\`es}
\address{Pierre Tarr\`es, Mathematical Institute, University of Oxford, 24-29 St Giles', Oxford OX1 3LB,  U.K.}
\email{tarres@maths.ox.ac.uk}

\author{Yuan Yao}
\address{Yuan Yao, Department of Mathematics, University of California at Berkeley, Berkeley, CA 94720.}
\curraddr{Toyota Technological Institute at Chicago, 1427 East
60th Street, Chicago, IL 60637.}
\email{yao@math.berkeley.edu}

\keywords{Online Learning, Stochastic Approximations, Regularization Path, Reproducing Kernel Hilbert Space}

\date{November 12, 2005.}

\subjclass[2000]{62L20, 68Q32, 68T05}

\maketitle

\begin{abstract}
In this paper, we study an online learning algorithm as stochastic approximations of a regularization path convergent to the regression function. 
Some probabilistic upper bounds are given for the convergence rates of the algorithm, under certain regularity assumption on the regression function. 
In the case of a strong convergence (convergence in reproducing kernel Hilbert spaces), 
the convergence rate obtained is the same as the best known rate in batch learning; and in the case of a weak convergence (convergence in mean square distance),
the convergence rate is optimal in the sense that it reaches the minimax and individual lower rate. 
\end{abstract}

\bigskip

%\tableofcontents

%\setcounter{section}{-1}


%%%%%%%%%%%%%%%%%%%%%%%%%%%%%%%%%%%%%%%%%%%%%%%%%%%%%%%%%%%%%%%%%%%%%%%%%%%%%%%%%%%%%%%%%%%%

\section{Introduction}

%%%%%%%%%%%%%%%%%%%%%%%%%%%%%%%%%%%%%%%%%%%%%%%%%%%%%%%%%%%%%%%%%%%%%%%%%%%%%%%%%%%%%%%%%%%%

Consider the following problem of learning from examples: given a sequence of random examples 
$(z_t=(x_t,y_t))_{t\in \N}$ drawn from a probability measure $\rho$ on $\X\times \Y$, one wants to approximate
the \emph{regression function}, $f_\rho(x):=\int_\Y y d \rho_{\Y|x}$, \emph{i.e.} the conditional expectation of $y$ given $x$. 
In an optimization view, $f_\rho$ minimizes the following quadratic functional 
\begin{equation}  \label{eq:ls}
\Err(f) = \int_{\X\times \Y} (f(x) - y)^2 d \rho.
\end{equation}
However, minimizing (\ref{eq:ls}) without constraints may put $f_\rho$ in a too large space to search. So one typically turns to 
some regularization methods to solve (\ref{eq:ls}). In this paper, we focus on the following regularized least square problem
\begin{equation}  \label{eq:reg}
f_\la = \arg \min_{f\in \H} \Err(f) + \la \|f\|_\H^2, 
\end{equation}
where $\H$ is some Hilbert space and $\la>0$ is the regularization parameter. This scheme is well-known as ridge regression in statistics
and is also called Tikhonov regularization in inverse problems \cite{EngHanNeu00}. One may choose a suitable $\H$, such that when $\la \to 0$, $f_\la \to f_\rho$. 
The map $f_\la: \R_+ \to \H$ defined by $\la\mapsto f_\la$, characterizes a function path in $\H$, which may start from some $f_{\la_0}$ and 
go toward $f_\rho$ as $\la\to 0$. Therefore it is called here a \emph{regularization path} of $f_\rho$ in $\H$. 

Our purpose is to seek a sequence of functions $(f_t)_{t\in\N}\in \H$ with certain dependence on the examples up to time $t$, 
and a sequence of regularization parameters $(\la_t)$ tending to $0$, such that $f_t$ follows $f_{\la_t}$ closely and converges to $f_\rho$. 
It is crucial to restrict the permissible form of the dependence of $f_t$ on historical examples. For example, $f_t$ may depend explicitly on 
all the available examples at time $t$, $(z_i)_{i=1}^t$, which is often called as \emph{batch learning}. In this setting existing results
[e.g. \citeNP{CucSma02}; \citeNP{SmaZho-ShannonIII}] already give such a sequence $f_t$, which is the minimizer of an empirical counterpart of (\ref{eq:reg})
by replacing the integral with a sum over the sample set at time $t$. 


In this paper, our departure from those batch learning results lies on the following recursive structure imposed on $(f_t)$: for each $t$, 
$f_t$ depends on the example $z_t$ and $f_{t-1}$ which only depends on previous examples $z_1,\ldots, z_{t-1}$, i.e. $f_t=T_t(f_{t-1},z_t)$ 
for some map $T_t:\H\times \X \to \H$. Such a sequence $(f_t)$ is called an \emph{online learning} sequence, as a constrast to the \emph{batch learning}. 
In particular, in online learning, the sample size $t$ is changing over time, thus one can not choose a fixed regularization parameter based on 
\emph{a priori} knowledge on a fixed sample size as in batch learning. This feature forces online learning to track the entire regularization path, which increases the technical 
difficulty in the treatment of online learning vs. batch learning. We note that recent works on support vector machines \cite{HasRosTibZhu04} enables one to construct 
batch learning algorithms to follow the entire regularization path with the same amount of computational cost as a single fixed regularization.

An attractive feature of this online learning scheme, lies in that its computational complexity to follow the entire regularization path
may be as small as linear $O(t)$ (the algorithm in this paper, however, requires $O(t^2)$ in the worst case). 
In a contrast, the batch learning scheme typically involves inverting a matrix, which is $O(t^3)$. 

The construction of such an online learning sequence $(f_t)$, as we shall see soon, is a stochastic approximation of the gradient descent method to 
solve (\ref{eq:reg}), for each fixed $\la_t$. In this way, our algorithms can be regarded as stochastic approximations of the regularization path $f_\la$.
For fixed regularization parameters, $\la_t = \la$, this kind of online learning algorithms in a similar setting has been studied in 
\cite{SmaYao05}, with improved upper bounds in \cite{Yao05}. References [\citeNP{Duflo96}; \citeNP{KusYin03}] provide more background on 
stochastic approximations.  

As in previous results, in this paper we choose $\H$ a reproducing kernel Hilbert spaces (RKHS), $\H_K$. RKHS enables us to develop results in a 
coordinate-free way, where the gradiend descent method takes an especially simple form [e.g. \citeNP{KivSmoWil04}]. 
Moreover, RKHS provides an unified framework to include several important settings, e.g. 
(1) generalized smooth spline functions in Sobelev spaces \cite{Wahba90}, (2) real analytic functions with bounded bandwidth \cite{Daubechies92} 
and their generalizations \cite{SmaZho-ShannonI}, (3) an unified framework for stochastic processes [\citeNP{Loeve48}; \citeNP{Parzen61}]. 
By choosing suitable kernels, $\H_K$ can be used to approximate any functions in $\L2$, the square integrable functions with respect to 
the marginal probability measure $\rho_\X$. For a wider background on RKHS, see for example \cite{BerTho04}. 

In this paper, we present some probabilistic upper bounds for the convergence of $(f_t)_{t\in \N}$ to $f_\rho$, in $\H_K$ or $\L2$, 
under the assumption of $f_\rho\in \H_K$ with additional regularity. The convergence rate in $\H_K$ is shown the same as the best known rate 
in batch learning \cite{SmaZho-ShannonIII}; and the convergence rate in $\L2$, is optimal in the sense that it reaches the minimax and individual 
lower rate.

Our treatment starts from more general Hilbert spaces, with stochastic approximations of the solutions for a sequence of linear operator equations, which
gives the main results in this paper when specialized to the setting in $\H_K$. Two structural decomposition theorems, namely the \emph{reversed martingale 
decomposition} and \emph{the martingale decomposition}, play an important role in the proof of the main results, where the former is suitable for the 
strong convergence in $\H_K$ and the latter, by exploiting the spectral decomposition, is suitable for the weak convergence in $\L2$. A crucial estimate 
is about the drift along the regularization path, $\|f_{\la_t}-f_{\la_{t-1}}\|$, 
which has the same order as the approximation error $\|f_{\la_t}- f_\rho\|$. This key observation leads to the same
rates for online learning as in batch learning. 

The organization of this paper is as follows. Section \ref{sec:main} collects the main results in this paper. Section \ref{sec:general} studies
stochastic approximations of regularization paths for linear operator equations in general Hilbert spaces, where two crucial structural decompositions
are presented. Section \ref{sec:drifts} collects some estimates on the drifts along the regularization path, 
$\|f_{\la} - f_{\mu}\|$ ($\la,\mu>0$), which is the basis for later developments.
Section \ref{sec:hk} studies the upper bounds for convergence in $\H_K$ and
Section \ref{sec:l2} studies the upper bounds for convergence in $\L2$. Appendix A derives
a probabilistic inequality from the Pinelis-Bernstein inequality, which is used to derive the probabilistic upper bounds in this paper. 
Appendix B collects some basic estimates. Appendix C gives some estimates on the gap between the online learning sequence and the regularization path.
Appendix D gives the proof of Theorem \ref{thm:convergence}.  

\section{Main Results} \label{sec:main}

\subsection{Notations and Assumptions} 
Let $\X\subseteq \R^n$ be closed, $\Y=\R$ and $\Z=\X\times \Y$. 
Let $\rho$ be a probability measure on $\Z$, $\rho_{\X}$ be the induced marginal probability measure on $\X$,
and $\rho_{\Y|x}$ be the conditional probability measure on $\Y$ with respect to $x\in
X$. Define $f_\rho:\X\to \Y$ by $f_\rho (x) = \int_\Y y d \rho_{\Y|x}$, 
the \emph{regression function of $\rho$}. In the sequel, we denote by $\E[\ ]$ the expectation. 

Let $\L2$ be the Hilbert space of
square integrable functions with respect to $\rho_\X$. In the sequel $\|\ \|_\rho$
denotes the norm in $\L2$.

Let $K:\X\times \X\to \R$ be a \emph{Mercer kernel}, i.e. a
continuous symmetric real function which is \emph{positive
semi-definite} in the sense that $\sum_{i,j=1}^m c_i c_j K(x_i,
x_j)\geq 0$ for any $m\in \N$ and any choice of $x_i\in X$ and
$c_i \in \R$ ($i=1,\ldots,m$). A Mercer kernel $K$ induces
a function $K_x : \X\to \R$ ($x\in \X$) defined by $ K_x(x^\prime) = K(x,x^\prime)$. 
Let $\H_K$ be the \emph{reproducing kernel Hilbert space} (RKHS)
associated with a Mercer kernel $K$, i.e. the completion of the $\span\{K_x: x\in \X\}$
with respect to the following inner product: the unique linear extension of the
bilinear form $\<K_x, K_{x^\prime}\>_K = K(x,x^\prime)$ ($x,x^\prime \in \X$). The norm
of $\H_K$ is denoted by $\|\ \|_K$. The most important property of RKHS is the \emph{reproducing property}:
for all $f\in \H_K$ and $x\in X$, $f(x)=\<f,K_x\>_K$. 

Throughout this paper, assume that

\noindent {\bf Finiteness Condition.} 
\noindent (A) There exists a constant $\ka\geq 0$ such taht
\begin{equation*} 
\kappa:=\sup_{x\in \X} \sqrt{K(x,x)} < \infty.
\end{equation*}
\noindent (B) There exists a constant $\M \geq 0$ such that
\[ \supp(\rho)\subseteq \X\times [-\M, \M]. \]

Define a linear map
$L_K:\L2\to \H_K$ by $L_K(f)(x)= \int_X K(x,t) f(t) d
\rho_X$. Together with the inclusion $J:\H_K \to \L2$, $J\circ L_K:\L2 \to \L2$
is a compact operator on $\L2$ [e.g. \citeNP{CarDevToi05} or \citeNP{HalSun78}]. The operator $L_K|_{\H_K} + \lambda I:\H_K \to \H_K$ is invertible if $\lambda>0$,
where $L_K|_{\H_K}:\H_K \to \H_K$ is the restriction of $L_K:\L2
\to \H_K$. All the three operators, by abusing the notation, are all denoted by $L_K$ in the sequel. 
It can be shown [e.g. \citeNP{CucSma02}] that for any $\la\in \R_+$,
\begin{equation} 
f_{\la} = (L_K + \la I)^{-1} L_K f_\rho \in \H_K. 
\end{equation}
Next we define the operator $L_K^r$, which will be used to represent the regularity of $f_\rho$
in this paper. The compactness of $L_K:\L2\to \L2$ implies the existence of an orthonormal eigensystem $(\mu_\al,\phi_\al)_{\al\in \N}$ of $L_K$, whence
we define $L_K^r:\L2\to\L2$ by
\begin{equation} \label{eq:LKr}
\begin{array}{rcl}
L_K^r: & \L2 & \to \L2 \\ 
& \DS \sum_{\al} a_\al \phi_\al & \DS \mapsto \sum_\al a_\al \mu_\al^r \phi_\al 
\end{array}
\end{equation}
Note that $L_K^{1/2}:\L2\to \H_K$ is an isometrical isomorphism between the quotient space $\L2/\ker(L_K)$ and $\H_K$. For simplicity in this paper
we assume that $\ker(L_K)=\{0\}$.
Finally we define an operator $L_K^x:\H_K \to \H_K$ by $L_K^x(f) = \<f,K_x\>_K K_x=f(x)K_x$, whose expectation with $\rho_X$ is $\E[L_K^x]=L_K$. 

\subsection{Stochastic Gradient Algorithms}

Recall that in the introduction, a sequence $(f_t)_{t\in \N}$ is called an \emph{online learning sequence}, if for
each $t$, $f_t = T_t ( f_{t-1}, z_t)$, where $(z_t)$ is a sequence of random examples and the map $T_t: \H_K \times \Z  \to \H_K$ depends on $t$. 

Define an online learning sequence $(f_t)_{t\in\N}$ as follows,
\begin{equation}\label{eq:ft}
f_t = f_{t-1} - \gamma_t [(f_{t-1}(x_t)-y_t)K_{x_t} +\lambda_t f_{t-1}], \ \ \ \ \ \mbox{for some $f_0\in \H_K$, e.g. $f_0=0$}
\end{equation}
where \\
(A) for each $t$, $(x_t,y_t)$ is independent and identically distributed (i.i.d.) according to $\rho$; \\
(B) the step size $\gamma_t>0$; \\
(C) the regularization parameter $\lambda_t>0$. 

\begin{rem}
The computational cost of this algorithm typically is $O(t^2)$. As each step $t$, the main computational cost is due to the evaluation $f_{t-1}(x_t)$ which
needs to access all $K_{x_i}$ ($1\leq i\leq t$) in $O(t)$ steps. Thus the total cost is of $O(t^2)$ at time $t$. In the cases that one can store and access
the values $f_t(x)$ for all $x$, e.g. on a grid of $\X$, the computational cost is only linear $O(t)$ at the requirement of large memory and fast memory 
access.
\end{rem}

By reproducing property, we can see that the gradient map of 
$$V_z(f)= \frac{1}{2}[(f(x)-y)^2 + \la \|f\|_K^2], \ \ \ \ z=(x,y)\in \Z$$
is given by $\grad V_z(f)=(f(x)-y)K_{x} +\lambda f$ \cite{SmaYao05}, as a random variable depending on $z$. 
Since the expectation $\E[V_z(f)]=2(\Err(f)+\la\|f\|_K^2)$, algorithm (\ref{eq:ft}) can thus be regarded as 
stochastic approximations of gradient descent method to solve (\ref{eq:reg}), for each $\la=\la_t$. 

\subsection{Main Theorems}

In the sequel we consider the approximation of $f_\rho \in \H_K$ by online learning sequence $(f_t)$. 
The following theorems are the main results of this paper. 
Theorem A gives a sufficient condition that $(f_t)$ follows the regularization path $f_\la$. 
Theorem B and C provides probabilistic upper bounds for the convergence rate of $f_t\to f_\rho$ in $\H_K$ and $\L2$, respectively. 

\medskip

First we need a definition for that $(f_t)$ follows the regularization path $f_\la$.
\begin{defn}
An online learning sequence $(f_t)$ is said to \emph{follow the regularization path} $f_\la:\R_+\to \H_K$, if there exists a sequence $(\la_t)\downarrow 0$, such that
$$\lim_{t\to \infty} \E \|f_t - f_{\la_t} \|_K = 0.$$ 
\end{defn}

The following theorem gives some sufficient conditions for $(f_t)$ following the regularization path. 

\begin{thma} [Path Following Condition]
The online learning sequence $(f_t)$ defined by equation (\ref{eq:ft}) follows the regularization path $f_\la$, 
if the following conditions are satisfied:

\noindent (A) $\DS \sum_{t\to \infty} \gamma_t \la_t = \infty$.

\noindent (B) $\DS \lim_{t\to \infty} \frac{\gamma_t}{\la_t} =0$,

\noindent (C) $\DS \lim_{t\to \infty} \frac{\|f_{\la_t} - f_{\la_{t-1}}\|_K}{\la_t \gamma_t } =0$,
\end{thma}

This theorem will be proved in Section \ref{sec:general}, following from Theorem \ref{thm:convergence} in the setting of general Hilbert spaces.  

\begin{rem}
Although $\la_t \to 0$, condition (A) puts a restriction that $\ga_t\la_t$ can not drop too fast, in fact this is necessary to ``forget'' the error 
caused by the initial guess $f_0$. Condition (B) says that the step size $\ga_t \to 0$, and it has to drop faster than the regularization parameter $\la_t$. 
Such a condition is to attenuate the random fluctuation caused by sampling. 
Condition (C) implies that the drifts of the regularization path $(f_{\la_t})$ converges to zero, at a speed faster than $\ga_t \la_t$. This condition
says that in the long run, the drifts along the regularization path should be slow enough for the algorithm to follow the path. 
\end{rem}

The next two theorems present some probabilistic upper bounds which characterize the convergence rates in $\H_K$ and $\L2$, under certain regularity assumptions
on the regression function $f_\rho$. Below by $L_K^{-r} f_\rho \in \L2$ ($r>0$), we mean that $f_\rho$ lies in the image of the mapping $L_K^r:\L2\to \L2$. 
Due to the isometry $L_K^{1/2}:\L2 \to \H_K$, for $r\geq 1/2$ this implies $f_\rho \in \H_K$, with additional regularity if $r>1/2$. Using the spectral decomposition of $L_K$,
$L_K^{r}$ can be regarded as a low-pass filter as given in (\ref{eq:LKr}), whence
\[ L_K^{-r} f_\rho \in \L2 \Leftrightarrow f_\rho=\sum_{\al} a_\al \mu_\al^r \phi_\al,\ \ \  \mbox{for some $\sum_\al a_\al^2 <\infty$}.\]


\begin{thmb} [Convergence Rates in $\H_K$] 
Assume that $L_K^{-r} f_\rho\in \L2$ for some $r\in (1/2,3/2]$. Let $\DS \t= (\ka+1)^{(2r+1)/r} \max(1,\log \frac{2}{\delta})^{(2r+1)/2r}$ 
for some $\delta\in (0,1)$.
Then there is a choice of $(\ga_t)$ and $(\la_t)$ such that with probability at least $1-\delta$, 
the following holds for all $t\in \N$, 
\[ \|f_t - f_\rho \|_K \leq \frac{C_1}{t} + ( C_2 \log^{1/2}\frac{2}{\delta} + C_3 \| L^{-r}_K f_\rho \|_\rho ) 
\left(\frac{1}{t}\right)^{\frac{2r-1}{4r+2}}, \]
where 
\[ C_1= 2\t^{\frac{4r+3}{4r+2}} M_\rho, \ \ \ C_2 = 11 (\kappa+1) M_\rho, \ \ \  C_3 = \frac{20r-2}{(2r-1)(2r+3)}. \]
One such choice is $\ga_t = (t+\t)^{-2r/(2r+1)}$ and $\la_t = (t+\t)^{-1/(2r+1)}$.
\end{thmb}

Its proof is given in Section \ref{sec:hk}. 

\begin{rem}
The asymptotic rate $O(t^{-(2r-1)/(4r+2)})$ is the same as the batch learning algorithms [Theorem 2, in \citeNP{SmaZho-ShannonIII}]. 
\end{rem}

\begin{rem}
Note that the upper bound consists of three parts. 
The first term at a rate $O(t^{-1})$, captures the influence of the initial choice $f_0=0$, which is much faster than the remaining terms.
The second term at a rate $O(t^{-(2r-1)/(4r+2)})$, reflects the error caused by random fluctuations by the i.i.d. sampling. 
The third term at a rate $O( \| L^{-r}_K f_\rho \|_\rho t^{-(2r-1)/(4r+2)})$, collects contributions from both drifts along the 
regularization path $f_{\la_t}-f_{\la_{t-1}}$ and the approximation error $f_{\la_t}-f_\rho$, since they share the same rate upto different constants. 
\end{rem}


\begin{thmc}[Convergence Rates in $\L2$]
Assume that $L_K^{-r} f_\rho\in \L2$ for some $r\in [1/2,1]$. Let $\DS \t=(\kappa^2+1)^{(2r+1)/r}$.
Then there is a choice of $(\ga_t)$ and $(\la_t)$ such that with probability at least $1-\delta$ ($\delta\in (0,1)$), 
the following holds for all $t\in \N$, 
\[ \|f_t - f_\rho \|_\rho \leq D_1 t^{-1} + (D_2 \sqrt{\frac{1}{\delta}} + D_3 \| L^{-r}_K f_\rho\|_\rho 
+D_4 \sqrt{\frac{1}{\delta}} \| L^{-r}_K f_\rho\|_\rho) t^{-r/(2r+1)}, \]
where 
\[ D_1 = 2(\t+1) M_\rho, \ \ \  D_2 = \sqrt{6} \ka \M (\sqrt{3}  + \ka \sqrt{4\ka^2+1} ),\ \ \  D_3=\frac{5r+1}{r(r+1)}, \ \ \ D_4=2\sqrt{2}\ka. \]
One such choice is $\ga_t = (t+\t)^{-2r/(2r+1)}$ and $\la_t = (t+\t)^{-1/(2r+1)}$.
\end{thmc}

Its proof will be given in Section \ref{sec:l2}.

\begin{rem}
A special case is $r=1/2$, which is equivalent to say $f_\rho \in \H_K$. In this case $\ga_t = \la_t = (t+\t)^{-1/2}$, whence it
does not satisfy the Path Following Condition (B) in Theorem A. But Theorem C suggests a weaker notion that $f_t$ follows the regularization path,
\emph{i.e.} $\lim_{t\to \infty} \E[\|f_t - f_{\la_t}\|_\rho] = 0$, which in fact converges at a rate of $O(t^{-1/4})$ uniformly for all $f_\rho \in \H_K$. 
\end{rem}

\begin{rem}
It is still open whether the upper bound above can be improved by replacing $1/\delta$ with $\log 1/\delta$. One way to achieve this is to prove
that $\|f_t\|_\rho \leq O(1/\sqrt{\la_t})$, which is open yet. For details, see more discussions in Remark \ref{rem:open}, 
on the problem of using Bernstein's type inequality here.
\end{rem}

Note that for any $f\in \L2$, the generalization error $\Err(f) - \Err(f_\rho)=\|f - f_\rho \|_\rho^2$ [e.g. see \citeNP{CucSma02}], 
which is often used to evaluate the performance of learning algorithms in literature. We have the following corollary of Theorem C.

\begin{cor} \label{cor:C}
Under the same condition of Theorem C, there holds with probability at least $1-\delta$ ($\delta\in (0,1)$), for all $t\in \N$, 
\[ \Err(f_t) - \Err(f_\rho) \leq 2D_1 t^{-2} + 2(D_2 \sqrt{\frac{1}{\delta}} + D_3 \| L^{-r}_K f_\rho\|_\rho++D_4 \sqrt{\frac{1}{\delta}} \| L^{-r}_K f_\rho\|_\rho)^2 t^{-2r/(2r+1)}, \]
where $D_1,\ldots,D_4$ are the same constants in Theorem C. 
\end{cor}

\begin{rem}
For $r\in (1/2,1]$, the asymptotic rate $O(t^{-2r/(2r+1)})$ has been shown to be optimal in the sense that it reaches the minimax and individual 
lower rate [\citeNP{CapDeV05}]. To be precise, let $\P(b,r)$ ($b>1$ and $r\in (1/2,1]$) 
be the set of probability measure $\rho$ on $\X\times \Y$, such that: (A) almost surely $|y|\leq \M$;
(B) $L_K^{-r} f_\rho \in \L2$; (C) the eigenvalues $(\mu_n)_{n\in\N}$ of $L_K:\L2\to\L2$, arranged in a nonincreasing order, are subject to the decay $\mu_n= O(n^{-b})$.
Then the following minimax lower rate was given as Theorem 2 in \cite{CapDeV05}, 
\[ \liminf_{t\to \infty} \inf_{(z_i)_1^t \mapsto f_t } \sup_{\rho\in \P(b,r)} \Prob \left\{(z_i)_1^t \in \Z^t: \Err(f_t) - \Err(f_\rho) > 
C t^{-\frac{2rb}{2rb+1}} \right\} = 1 \]
for some constant $C>0$ independent on $t$, where the infimum in the middle is taken over all algorithms as a map 
$\Z^t \ni(z_i)_1^t \mapsto f_t\in \H_K$. Moreover, for every $B>b$, the following individual lower rate was given as Theorem 3 in \cite{CapDeV05},
\[\inf_{((z_i)_1^t\mapsto f_t)_{t\in \N}} \sup_{\rho \in \P(b,r)} \limsup_{t\to \infty} \frac{\E [\Err(f_t)] - \Err(f_\rho)}{t^{-\frac{2rB}{2rB+1}}} > 0, \]
where the infimum is taken over arbitrary sequences of functions $f_t:\Z^t\to \H_K$. Note that in both lower rates the permissible $f_t$ 
is beyond online learning sequences. In the minimax lower rate, the probability measure may change
for each fixed $t$; while in the individual lower rate, the probability measure is fixed for all large enough $t$, which is more suitable to 
the nature of learning. For more background on these lower rates, see for example \cite{GKKW02} and references therein. 

Now we compare these lower rates to our upper bound. 
Since $L_K:\L2\to\L2$ is a trace-class operator, its eigenvalues are summable. Therefore by taking $b=B=1$, one may obtain an 
eigenvalue-independent lower rate $O(t^{-2r/(2r+1)})$ for all possible $L_K$. In this way, the upper bound in Corollary \ref{cor:C} reaches
both the minimax and the individual lower rates. 
\end{rem}


\section{Sequential Stochastic Approximations in Hilbert Spaces} \label{sec:general}

In this section, we consider a more general setting: stochastic approximations of solutions for a sequence of linear operator equations 
in general Hilbert spaces. The sequence of linear operator equations are constructed in a spirit of regularization. 
A theorem for the convergence is given, which leads to Theorem A when specialized to the setting in this paper.

Let $\W$ be a Hilbert space and $\A:\W\to \W$ be a positive operator and $\b\in \W$. Consider the following linear equation
\begin{equation} \label{eq:hatlin}
\A w = \b.
\end{equation}
where $\A$ has an \emph{unbounded} inverse. As in the standard setting of Robbins-Monro procedure \cite{RobMon51}, 
we assume that $\A$ and $\b$ are the expectations of some random operators and vectors, respectively.
However due to the unboundedness of $\A^{-1}$, the complexity analysis fails for the standard algorithm \cite{SmaYao05} . 

To solve this ill-posed problem with unbounded $\A^{-1}$, one may construct a sequence $\A_t\to \A$ and $\b_t\to \b$, 
where each $\A_t$ has bounded inverse. 
Then one has a sequence $\w_t=\A_t^{-1} \b_t$ converging to the solution of (\ref{eq:hatlin}) and $\w_t$ is continuous with respect
to $\A_t$ and $\b_t$. Such a sequence $(\w_t)$, will be called a \emph{regularization path} of the solution of equation (\ref{eq:hatlin}). 

It remains to approximate $(\w_t)$, since $\A_t$ and $\b_t$ are the means with respect to some unknown probability measure, as $\A$ and $\b$. 
Here we define a sequence of stochastic approximations
\[ w_{t}= w_{t-1} - \gamma_{t} ( A_t w_{t-1} - b_t ), \]
where $A_t=A_t(z_t)$ and $b_t=b_t(z_t)$ are random variables depending on the sample $z_t$, such that $\E[A_t(z_t)]=\A_t$ and $\E[b_t(z_t)]=\b_t$. 

In this section our purpose is to study the evolutions of $(w_t)$ and $(\w_t)$, and give conditions of their gap converging to zero.

First of all we summarize the assumptions taken in this section. 

\noindent {\bf Generalized Finiteness Condition.} For each $t\in \N$, almost surely there holds \\
\noindent (A) $\A$ is invertible with \emph{unbounded} inverse; \\
\noindent (B) $A_t$, $\A_t$ and $\A$ have operator norms bounded by $\amax\in(0,\infty)$; \\
\noindent (C) $\A_t$ is invertible with \emph{bounded} inverse $\|\A^{-1}_t\|\leq 1/\amin_t$, with $\amin_t \to 0$; \\
(D) $b_t$ and $\b_t$ have norms bounded by $\beta\in(0,\infty)$;


\subsection{Two Structural Decomposition Theorems}

In this subsection we presents two structural decompositions of the \emph{remainder}, 
\begin{equation}
r_t:=w_t - \w_t. 
\end{equation}
Both ways decomposes $r_t$ into three parts: 
one depending on $r_0$; one depending on the following defined \emph{drifts} along the regularization path $(\w_t)$,
\begin{equation}
\Delta_j:=\w_t - \w_{t-1};
\end{equation}
and one random variable of zero mean, either as a reversed martingale or
as a martingale. Both decompositions are found useful, where the reversed martingale decomposition is enough to study the convergence in $\H_K$ 
and the martingale decomposition is crucial to obtain sharp convergence rates in $\L2$.

\begin{thm}[Reversed Martingale Decomposition] \label{thm:rmart} Define a random operator on $\W$,
\begin{equation*} \Pi_j^t(z_j,\ldots,z_t)=
\left\{
\begin{array}{lr}
\DS \prod_{i=j}^t \left( I - \gamma_i A_{i} \right), & j\leq t; \\
I, & j>t.
\end{array}
\right.
\end{equation*}
Then for all $t\in \N$ and $t>t_0$,
\begin{equation}  \label{eq:rmart}
r_{t} = \Pi_{t_0+1}^t r_{t_0} - \sum_{j=t_0+1}^t \gamma_j \Pi_{j+1}^t (A_j \w_j - b_j) - \sum_{j=t_0+1}^t \Pi_{j}^t \Delta_j
\end{equation}
\end{thm}

\begin{rem}
Note that $\Pi_{j+1}^t$ is a random operator depending on $z_{j+1},\ldots z_t$, and $A_j \w_j - b_j$ is a zero mean random variable depending on $z_j$.
By independence of $(z_t)_{t\in \N}$, the conditional expectation $\E[\gamma_j \Pi_{j+1}^t (A_j \w_j - b_j)|z_{j+1},\ldots,z_{t}]=0$, whence 
for each $t$, $\ga_j \Pi_{j+1}^t(A_j\w_j - b_j)$ is a \emph{reversed martingale difference sequence} whose sum is a \emph{reversed martingale sequence} with zero mean. 
For more background on reversed martingale, see for example \cite{Neveu75}. This decomposition will be used to derive Theorem A and Theorem B. 
\end{rem}

\begin{proof}[Proof of Theorem \ref{thm:rmart}] By definition,
\begin{eqnarray*}
r_{t} & = & w_{t}-\w_{t} \\
& = & w_{t-1} - \gamma_t (A_t w_{t-1} - b_t) - \w_{t} \\
& = & (I - \gamma_t A_t)(w_{t-1} - \w_{t-1}) - \gamma_t (A_t \w_t -b_t) - (I-\gamma_t A_t) (\w_t-\w_{t-1})
\end{eqnarray*}
which gives
\begin{equation}
 r_{t} = (I-\gamma_t A_t) r_{t-1} - \gamma_t (A_t \w_t - b_t) - (I-\gamma_t A_t)\Delta_t .
\end{equation}
The result then follows from induction on $t\in \N$. 
\end{proof}

\begin{thm}[Martingale Decomposition] \label{thm:mart} Let $\chi_t=(\A_t - A_t) w_{t-1} +(b_t - \b_t)$, and
\begin{equation*} 
\PPi_j^t=
\left\{
\begin{array}{lr}
\DS \prod_{i=j}^t \left( I - \gamma_i  \A_i  \right), & j\leq t; \\
I, & j>t.
\end{array}
\right.
\end{equation*}
Then for all $t\in \N$ and $t>t_0$,
\begin{equation} \label{eq:mart}
r_{t} = \PPi_{\t+1}^t r_0 + \sum_{j=\t+1}^t \gamma_j \PPi_{j+1}^t \chi_j - \sum_{j=\t+1}^t \PPi_{j}^t \Delta_j 
\end{equation}
\end{thm}

\begin{rem}
This decomposition was proposed in \cite{Yao05}. 
Note that in this decomposition only the second term is random. The operator $\PPi_{j+1}^t$ is deterministic and $\chi_j$
is a zero mean random variable depending on $z_1,\ldots, z_j$. Therefore the conditional expectation
$\E[\gamma_j \PPi_{j+1}^t \chi_j|z_1,\ldots,z_{j-1}]=0 $, whence for each $t$, $\gamma_j \PPi_{j+1}^t \chi_j$ is a 
\emph{martingale difference sequence} for all $t\in \N$, whose sum is a \emph{martingale sequence} of zero mean. 
Note that this martingale property holds even for dependent sampling $z_t(z_1,\ldots,z_{t-1})$. 
\end{rem}

\begin{rem}
An importance feature of this decomposition used in this paper, lies in that the operator $\PPi_j^t$ is deterministic and when taking $\A_i = L_K+\la_i$, it
has a spectral decomposition by the eigenfunctions of $L_K:\L2\to \L2$. This feature plays a key role in the proof of Theorem C. 
But a disadvantage is that the term $\chi_t$ depends on $w_{t-1}$, which increases the difficulty to bound $\chi_t$. 
In fact, the open problem how to improve Theorem C by replacing $1/\delta$ to $\log1/\delta$, depends on how to get a tighter
bound on $\|\chi_t\|$, see Remark \ref{rem:open} for details. 
\end{rem}

\begin{proof}[Proof of Theorem \ref{thm:mart}] By definition,
\begin{eqnarray*}
r_{t} & = & w_{t}-\w_{t} \\
& = & w_{t-1} - \gamma_t (A_t w_{t-1} - b_t) - \w_{t} \\
& = & (I - \gamma_t \A_t)(w_{t-1} - \w_{t-1}) + \gamma_t [(\A_t-A_t) w_{t-1} + (b_t - \A_t \w_t  )] - (I-\gamma_t \A_t) [\w_t - \w_{t-1}].
\end{eqnarray*}
Using $\A_t \w_t = \b_t$ for all $t\in \N$, we obtain
\begin{equation}
r_{t} = (I-\gamma_t \A_t) r_{t-1} + \gamma_t [(\A_t - A_t) w_{t-1} +(b_t - \b_t)] - (I-\gamma_t \A_t) (\w_{t}-\w_{t-1}). 
\end{equation}
The result then follows from induction on $t\in \N$. 
\end{proof}

\subsection{Convergence of Remainder and The Proof of Theorem A}

Here we give an application of the reversed martingale decomposition, Theorem \ref{thm:rmart}, to derive Theorem A. A general theorem is first
given to study $\E \|r_t\|\to 0$, which leads to Theorem A when specialized to $r_t = f_t - f_{\la_t}$. 

\begin{thm}\label{thm:convergence}
Suppose that the variance $\E \|A_t \w_t - b_t \|^2$ is uniformly bounded for all $t\in \N$.
Then
\[ \E \|r_t \| \to 0, \]
if the following conditions hold: 

\noindent (A) $\DS \sum_{t} \gamma_t \amin_t = \infty$.

\noindent (B) $\DS \lim_{t\to \infty} \frac{\gamma_t}{\amin_t} =0$,

\noindent (C) $\DS \lim_{t\to \infty} \frac{\|\Delta_t\|}{\amin_t \gamma_t } =0$, 

\end{thm}

Its proof was included in Appendix D. Equipped with Theorem \ref{thm:convergence}, we are in a position to prove Theorem A. 

\begin{proof}[Proof of Theorem A]
Let $\W=\H_K$ and $w_t=f_t$. Define $A_t = L_t + \lambda_t I$ ($L_t:=L_K^{x_t}=\<\ ,K_{x_t}\>_K K_{x_t}$), $\A_t = L_K + \lambda_t I$, $b_t = y_t K_{x_t} $, and $\b_t =L_K f_\rho$. 
Therefore $\amin_t = \la_t$ and $\w_t = f_{\la_t}$. With these replacement, it suffices to check the uniform boundedness of 
$$\E \|A_t \w_t - b_t \|^2=\E\|(f_{\la_t}(x_t) - y_t)K_{x_t} + \la_t f_{\la_t}\|_K^2. $$
But this can be seen by Lemma \ref{lem:xi}(B) which gives $\E\|(f_{\la_t}(x_t) - y_t)K_{x_t} + \la_t f_{\la_t}\|_K^2\leq 10\ka^2 \M^2$. 
Theorem A then follows from Theorem \ref{thm:convergence}.
\end{proof}

%%%%%%%%%%%%%%%%%%%%%%%%%%%%%%%%%%%%%%%%%%%%%%%%%%%%%%%%%%%%%%%%%%%%%%%%%%%%%%%%%%%%%%%%%%%%%%%%%%%%%%

\section{Estimates for The Drifts} \label{sec:drifts}

%%%%%%%%%%%%%%%%%%%%%%%%%%%%%%%%%%%%%%%%%%%%%%%%%%%%%%%%%%%%%%%%%%%%%%%%%%%%%%%%%%%%%%%%%%%%%%%%%%%%%%

In this section, we provide some estimates on the drift $\|f_\la - f_\mu\|$ ($\la,\mu>0$), in $\H_K$-norm or $\L2$-norm.
When specialized to $\mu=0$, these estimates give the approximation errors $\|f_\la-f_\rho\|$. 
Note that we extend the range of $r$ to $(0,\infty)$ (or $(1/2,\infty)$), from $(0,1]$ (or $(1/2,1]$), 
usually considered in literature [e.g. see \citeNP{SmaZho-ShannonIII}]. Such an extension was firstly pointed to us by H.Q. Minh. 
Note that for large enough $r$ ($r>1$ in $\L2$ norm or $r>3/2$ in $\H_K$ norm), the rates in the upper bounds can not be improved. 
This phenomenon is related to the \emph{saturation} problem of regularizations \cite{EngHanNeu00}. 

\begin{thm} \label{thm:pie1} Let $\lambda>\mu\geq 0$. If $\mu=0$ we define $f_\mu=f_\rho$. Assume that $L_K^{-r}f_\rho\in \L2$ for
some $r>0$. 

(A) If $r\in (0,1]$, then
\[ \|f_\lambda-f_\mu\|_\rho \leq  |\lambda^r - \mu^r| \frac{\|L_K^{-r} f_\rho\|_\rho}{r}; \]

(B) If $r\geq 1$, then for any $1\leq s \leq r$, 
\[ \|f_\lambda-f_\mu\|_\rho \leq  \kappa^{2(s-1)} |\lambda - \mu| \|L_K^{-s} f_\rho\|_\rho; \]

(C) If $r\geq 1/2$, then
\[ \|f_\lambda - f_\mu \|_K \leq \frac{|\la - \mu|}{\la} \|f_\rho\|_K; \]

(D) If $r\in (1/2,3/2]$, then 
\[ \|f_\lambda-f_\mu\|_K \leq  |\lambda^{r-1/2} - \mu^{r-1/2}| \frac{\|L_K^{-r} f_\rho\|_\rho}{r-\frac{1}{2}}; \]

(E) If $r\geq 3/2$, then for any $3/2\leq s \leq r$,
\[ \|f_\lambda-f_\mu\|_K \leq  \kappa^{2(s-3/2)} |\lambda - \mu| \|L_K^{-s} f_\rho\|_\rho. \]
\end{thm}

\begin{rem}
From (A) and (B) (or (D) and (E)) we can see that $\|f_\la - f_\mu\|_\rho\leq O(|\la^{\min(r,1)}-\mu^{\min(r,1)}|)$ 
(or $\|f_\la - f_\mu\|_K\leq O(|\la^{\min(r-1/2,1)}-\mu^{\min(r-1/2,1)}|)$). In this way the upper bounds `saturate' in the rates
when $f_\rho$ has large enough regularity indexed by $r>1$ (or $r>3/2$). 
\end{rem}

\begin{proof}
Assume that $\lambda\geq \mu$ for simplicity. By definition,
$$(L_K+\la I)f_\la=L_Kf_\rho,~(L_K+\mu I)f_\mu=L_K f_\rho,$$
which yields
\begin{equation} \label{eq:pierre}
f_\la-f_\mu=(\mu-\lambda)(L_K+\la I)^{-1}(L_K+\mu I)^{-1}L_Kf_\rho.
\end{equation}

(A) If $r\in (0,1]$,
\begin{eqnarray*}
\|f_{\lambda}- f_{\mu}\|_\rho & \leq & |\mu-\lambda|\|(L_K+\la I)^{-1} f_\rho \|_\rho \leq  |\mu-\lambda|\|(L_K+\la I)^{-1} L_K^r \| \|L_K^{-r}f_\rho \|_\rho  \\
& \leq & |\mu-\lambda|\|(L_K+\la I)^{r-1} \| \|L_K^{-r}f_\rho \|_\rho  = \Lambda(\mu) |\mu^r - \lambda^r|  \|J\| \|L_K^{-r} f_\rho\|_\rho
\end{eqnarray*}
where
\[ \Lambda(\mu)=\frac{1-\frac{\mu}{\lambda}}{1-\left(\frac{\mu}{\lambda}\right)^r}, \ \ \ \ \ \mbox{and}\ \ \ \  J= \lambda^{r-1} (L_K+\lambda I)^{r-1}. \]
Now $\|J\|\leq 1$ and 
\[ \Lambda(\mu) \leq \frac{1}{r}, \]
where we use, for $u:=1-\mu/\la$, that
$u\leq(1-(1-u)^r)/r$, since $u\mapsto(1-(1-u)^r)/r$
(defined on $(-\infty,1]$) is convex and remains above the tangent
line at $0$. In particular, $\Lambda(0)=1$. This completes the proof of (A).

(B) For any $s\leq r$, $L_K^{-r}f_\rho\in \L2$ implies $L_K^{-s}f_\rho \in \L2$. If $s\geq 1$, by equation (\ref{eq:pierre}),
\begin{eqnarray*}
\|f_{\lambda}- f_{\mu}\|_\rho & \leq & |\mu-\lambda|\|(L_K+\la I)^{-1} f_\rho \|_\rho \leq  |\mu-\lambda|\|(L_K+\la I)^{-1} L_K^s \| \|L_K^{-s}f_\rho \|_\rho  \\
& \leq & |\mu-\lambda|\|L_K^{s-1} \| \|L_K^{-s}f_\rho \|_\rho  \leq \kappa^{2(s-1)} |\mu-\lambda| \|L_K^{-s}f_\rho \|_\rho 
\end{eqnarray*}

(C) In particular if $r\geq 1/2$, this implies $f_\rho \in \H_K$, whence by (\ref{eq:pierre})
\[ \|f_\la-f_\mu\|_K\leq |\mu-\lambda|\|(L_K+\la I)^{-1}\| \|(L_K+\mu I)^{-1}L_K\| \|f_\rho\|_K \leq \frac{|\mu-\lambda|}{\la}\|f_\rho\|_K. \]

(D) If $r\in (1/2,3/2]$, then similar to (A), 
\begin{eqnarray*}
\|f_\lambda- f_\mu\|_K = \|L^{-1/2}_K (f^\ast_{\lambda}- f^\ast_{\mu})\|_\rho & \leq &
\Lambda(\mu) |\mu^{r-1/2} - \lambda^{r-1/2}|
\|J\| \|L_K^{-r} f_\rho\|_\rho
\end{eqnarray*}
where
\[ \Lambda(\mu)=\frac{1-\frac{\mu}{\la}}{1-\left(\frac{\mu}{\la}\right)^{r-1/2}}, \ \ \ \ \ \mbox{and}\ \ \ \ \ \ J= \la^{3/2-r} (L_K + \la I )^{r-3/2}. \]
We complete the proof by replacing $r$ with $r-1/2$ in (A).

(E) It follows from (B) by replacing $s$ with $s-1/2$. 
\end{proof}

%%%%%%%%%%%%%%%%%%%%%%%%%%%%%%%%%%%%%%%%%%%%%%%%%%%%%%%%%%%%%%%%%%%%%%%%%%%%%%%%%%%%%%%%%%%%%%%%%%%%%%

\section{Upper Bounds for Convergence in $\H_K$} \label{sec:hk}

%%%%%%%%%%%%%%%%%%%%%%%%%%%%%%%%%%%%%%%%%%%%%%%%%%%%%%%%%%%%%%%%%%%%%%%%%%%%%%%%%%%%%%%%%%%%%%%%%%%%%%

In this section we are going to give a probabilistic upper bound for
\[ \|f_t - f_\rho \|_K \]
as a proof for Theorem B. Throughout this section, we assume 
that $L_K^{-r} f_\rho\in \L2$ for some $r\in (1/2,3/2]$, which implies $f_\rho \in \H_K$ with additional regularity. 

The idea in the proof starts from the triangle inequality 
\[\|f_t - f_\rho\|_K\leq \|f_t - f_{\la_t}\|_K+\|f_{\la_t} - f_\rho\|_K. \] 
The first term can be further decomposed into the reversed martingale decomposition as Theorem \ref{thm:rmart}, which can be rewritten as
\begin{equation}  \label{eq:rmart1}
 r_t = \Pi_{1}^t r_0 - \sum_{j=1}^t \gamma_j \Pi_{j+1}^t (A_j \w_j - b_j) - \sum_{j=1}^t \Pi_{j}^t \Delta_j
\end{equation}
where $r_t = f_{t} - f_{\la_t}$, $A_t=L_t + \la_t I$ ($L_t:=L_K^{x_t}=\<\ ,K_{x_t}\>_K K_{x_t}$), $b_t = y_t K_{x_t}$, $\w_t = f_{\la_t}$, $\Delta_t = f_{\la_t} - f_{\la_{t-1}}$, and 
\begin{equation*} \Pi_j^t(x_j,\ldots,x_t)=
\left\{
\begin{array}{lr}
\DS \prod_{i=j}^t \left( I - \gamma_i (L_i + \la_i I) \right), & j\leq t; \\
I, & j>t,
\end{array}
\right.
\end{equation*}
with the choice that 
\[ \gamma_t = \frac{1}{(t+\t)^{\theta}}, \ \ \ \ \ \lambda_t = \frac{1}{(t+\t)^{1-\theta}}, \ \ \mbox{for some $\theta\in[0,1]$} \]

For convenience, we make the following definitions.

{\bf [Definitions of Errors]}

\noindent (A) \emph{Initial Error}: $\Err_{init}(t)=\|\Pi_1^t r_0\|$, which reflects the propagation error by the initial choice $f_0$; \\
\noindent (B) \emph{Sample Error}: $\Err_{samp}(t)=\|\sum_{j=1}^t \gamma_j \Pi_{j+1}^t (A_j \w_j - b_j)\|$, where $\xi_j =\gamma_j \Pi_{j+1}^t (A_j \w_j - b_j)$ 
is a reversed martingale difference sequence, reflecting the random fluctuation caused by sampling; \\
\noindent (C) \emph{Drift Error}: $\Err_{drift}(t)= \|\sum_{j=1}^t \Pi_{j}^t \Delta_j\|$, which measures the error caused by drifts from 
$f_{\la_{t-1}}$ to $f_{\la_t}$ along the regularization path; \\
\noindent (D) \emph{Approximation Error}: $\Err_{approx}(t)=\|f_{\la_t} - f_\rho\|_K$, which measures the distance between the regression function and
the regularization path at time $t$.

In this way we may bound
\[ \|f_t - f_\rho\|_K \leq \Err_{init}(t) + \Err_{samp}(t) + \Err_{drift}(t) + \Err_{approx}(t). \]
In the remaining of this section, we are going to provide upper bounds for each of the four errors, which, roughly speaking, are
\begin{eqnarray*} 
\Err_{approx}(t)  & \leq & O( t^{-(r-1/2)(1-\theta)} ) \\
 \Err_{drift}(t) & \leq & O( t^{-(r-1/2)(1-\theta)}) \\
 \Err_{init}(t) & \leq & O( t^{-1})  \\
 \Err_{samp}(t) & \leq & O( t^{\frac{1}{2} - \theta} )
\end{eqnarray*}

It is not surprising that the approximation error and drift error have the same rate, as both of them come from
the estimates on drifts in Theorem \ref{thm:pie1}. Theorem B then follows from these bounds by setting $\theta=2r/(2r+1)$.


\subsection{Approximation Error}

The approximation error is derived from Theorem \ref{thm:pie1}(D) by setting $\la=\la_t$ and $\mu=0$. 
\begin{thm}[Approximation Error] \label{err:approx} For $r\in (1/2,3/2]$,
\[ \|f_{\la_t} - f_\rho \|_K \leq C_1 (t+\t)^{-(r-1/2)(1-\theta)}, \]
where $C_1=(r-1/2)^{-1}\| L^{-r}_K f_\rho \|_\rho$.
\end{thm}


\subsection{Drift Error}

\begin{thm}[Drift Error] \label{err:drift} Assume that $a=1$ and $\t^\theta \geq \kappa^2+1$.
\[ \Err_{drift}(t)\leq C_2 (t+\t)^{-(r-1/2)(1-\theta)} \]
where $\DS C_2=\frac{4(1-\theta)}{1-(r-1/2)(1-\theta)}\|L_K^{-r} f_\rho\|_\rho$.
\end{thm}

\begin{proof} By Theorem \ref{thm:pie1}(D), it follows that
\[ \|\Delta_t\| = \|f_{\lambda_t}-f_{\lambda_{t-1}}\|_K \leq 4(1-\theta) (t+\t)^{-(r-1/2)(1-\theta)-1} \|L_K^{-r} f_\rho\|_\rho, \]
where we use
\begin{eqnarray}
|\lambda_{t}^{r-1/2} - \lambda_{t-1}^{r-1/2} | & = & | (t+\t)^{-(r-1/2)(1-\theta)} - (t+\t-1)^{-(r-1/2)(1-\theta)}| \nonumber \\
& \leq & (r-1/2)(1-\theta) (t+\t-1)^{-(r-1/2)(1-\theta)-1} \nonumber \\
& \leq & 4 (r-1/2)(1-\theta) (t+\t)^{-(r-1/2)(1-\theta)-1}, \ \ \ \mbox{$\DS \frac{a+1}{b+1}\geq \frac{a}{b}$ if $b>a>0$}, \label{eq:meanvaluethm}
\end{eqnarray}
where the second last step is due to the Mean Value Theorem with $h(x)=x^{-r(1-\theta)}$ and $h^\prime(x) = -r(1-\theta) x^{-r(1-\theta)-1}$, such that
\[ |h(t+\t)-h(t+\t-1)|=|h^\prime(\eta)|\leq |h^\prime(t+\t-1)|, \ \ \ \ \ \mbox{for some $\eta\in(t+\t-1,t+\t)$}. \]
By Lemma \ref{lem:onebnd}(B),
\[ \|\Pi_{j}^t\|\leq \frac{j+\t}{t+\t+1}, \]
whence
\begin{eqnarray*}
\Err_{drift}(t) &=& \| \sum_{j=1}^t \Pi_j^t \Delta_j\|_K \leq \frac{4(1-\theta)\|L_K^{-r} f_\rho\|_\rho}{t+\t+1}\sum_{j=1}^t 
(j+\t)^{-(r-1/2)(1-\theta)} \\
&\leq & \frac{4(1-\theta)\|L_K^{-r} f_\rho\|_\rho}{1-(r-1/2)(1-\theta)} (t+\t)^{-(r-1/2)(1-\theta)}
\end{eqnarray*}
since
\begin{eqnarray*}
\sum_{j=1}^t (j+\t)^{-(r-1/2)(1-\theta)} & \leq &   \int_0^t (x+\t)^{-(r-1/2)(1-\theta)} dx \leq \frac{(t+\t)^{1-(r-1/2)(1-\theta)}}{1-(r-1/2)(1-\theta)} 
\end{eqnarray*}
\end{proof}

 
\subsection{Initial Error}
\begin{thm}[Initial Error] \label{err:init} Let $\t\geq (\ka^2+1)^{1/\theta}$. Then for all $t\in \N$,
\[ \Err_{init}(t) \leq C_3 (t+\t)^{-1}, \]
where $C_3 = (\t+1)\|r_0\|$.
\end{thm}

\begin{proof}
By Lemma \ref{lem:onebnd}(B) with $j=1$,
\[ \|\Pi_1^t \| \leq \frac{\t+1}{t+\t+1} \leq \frac{\t+1}{t+\t}. \]
This gives $\Err_{init}(t) \leq (\t+1)\|r_0\| (t+\t)^{-1}$. 
\end{proof}

\subsection{Sample Error}

\begin{thm} [Sample Error] \label{err:samp}
Let $a=1$, $\t^\theta=(\kappa+1)^2\max(1,\log \frac{2}{\delta})$ and $\gamma_0 = \t^{-\theta}$. The following holds with probability
at least $1-\delta$ ($\delta\in (0,1)$) in the space $Z^t$,
\[ \Err_{samp} \leq C_4(t+\t)^{1/2-\theta}  \]
where $C_4=11 (\kappa+1) M_\rho \log^{1/2}\frac{2}{\delta}$.
\end{thm}

Before the formal presentation of the proof, we need some auxilary estimates.

\begin{lem} \label{lem:xi} Let $A_t \w_t - b_t=(f_{\lambda_t}(x_t) - y_t) K_{x_t} + \lambda_t f_{\lambda_t}$.

(A) $\|A_t \w_t - b_t\|_K\leq (\ka +1)^2 M_\rho/\sqrt{\lambda_t}$;

(B) $\E[\|A_t \w_t - b_t\|_K^2] \leq 10 \kappa^2 M_\rho^2$.
\end{lem}

\begin{proof}
Recall that
\[ A_t \w_t - b_t=(f_{\lambda_t}(x_t) - y_t) K_{x_t} + \lambda_t f_{\lambda_t}. \]
Then

(A) Using $\|f_\lambda\|_K \leq M_\rho/\sqrt{\lambda}$ in Lemma \ref{lem:fla}(A), 
\[ \|A_t \w_t - b_t\| \leq \|f_{\lambda_t}(x_t) K_{x_t}\|_K + |y_t| \| K_{x_t}\|_K + \lambda_t \|f_{\la_t}\|_K \leq M_\rho \kappa^2 /\sqrt{\lambda_t} + M_\rho \kappa + M_\rho \sqrt{\lambda_t} \]
since $\|f_{\lambda_t}(x_t) K_{x_t}\|_K = |\<f_{\lambda_t}, K_{x_t}\>|\| K_{x_t}\|_K  \leq \|f_{\lambda_t}\|_K \|K_{x_t}\|_K^2 
\leq M_\rho \kappa^2 /\sqrt{\lambda_t}$.
It remains to see
\begin{eqnarray*}
M_\rho \kappa^2 /\sqrt{\lambda_t} + M_\rho \kappa + M_\rho \sqrt{\lambda_t} & \leq & (\kappa^2+\ka +1) M_\rho/\sqrt{\lambda_t} \leq (\ka +1)^2 M_\rho/\sqrt{\lambda_t}
\end{eqnarray*}

(B) Using $\lambda_t f_{\la} = L_K f_\rho - L_K f_{\la}$ we obtain
\[ (f_{\lambda_t}(x_t) - y_t) K_{x_t} + \lambda_t f_{\la_t} = (L_t - L_K)f_{\la_t} +L_K f_\rho- y_t K_{x_t}.   \]
\begin{eqnarray*}
\E[\|A_t \w_t - b_t\|^2] & = &\E \|(L_t - L_K)f_{\la_t} +L_K f_\rho- y_t K_{x_t}\|_K^2 \\
& \leq & 2\E[\|(L_t - L_K)f_{\la_t}\|_K^2 + \|L_K f_\rho- y_t K_{x_t}\|_K^2] \\
& \leq & 2\E[\|L_t f_{\la_t}\|_K^2 +  \|y_t K_{x_t}\|_K^2] \leq 2 \kappa^2 (\|f_{\la_t}\|_\rho^2 + M^2_\rho) = 10 \kappa^2 M^2_\rho 
\end{eqnarray*}
since $\E[L_t] = L_K$, $\E[y_t K_{x_t}] = L_K f_\rho$ and $\|f_\la\|_\rho \leq 2 \M$ by Lemma \ref{lem:fla}(B). 
\end{proof}

Now we are ready to give the proof of the sample error bounds, Theorem \ref{err:samp}. 

\begin{proof}[Proof of Theorem \ref{err:samp}] Denote $X_j=\gamma_j \Pi_{j+1}^t ( A_j \w_j - b_j)=\gamma_j \Pi_{j+1}^t\xi_j$, which is a reversed martingale difference sequence.
It suffices to prove that
\begin{equation}\label{eq:xjbd}
\|X_j\|\leq 2 \ga_t \la_t^{-1/2}(\kappa+1)^2 M_\rho 
\end{equation}
\begin{equation}\label{eq:xjvar}
\E\|X_j\|^2 \leq 40\ga_t^2 \kappa^2 M_\rho^2
\end{equation}
Then by Proposition \ref{prop:bernstein}, a varied form of Pinelis-Bernstein inequality,
\begin{eqnarray*}
\Err_{samp}(t) & = & \|\sum_{j=1}^t X_j\|  \leq 2 \ga_t \la_t^{-1/2} (\kappa+1)^2 M_\rho \log \frac{2}{\delta} + \sqrt{80t \ga_t^2 \kappa^2 M_\rho^2 \log \frac{2}{\delta}} \\
& = & 2(\kappa+1) M_\rho (t+\t)^{1/2-\theta} \log^{1/2} \frac{2}{\delta} \left( (t+\t)^{-\theta/2} (\ka+1) \log^{1/2} \frac{2}{\delta} + \frac{\sqrt{20}\ka }{\kappa+1} \right) 
\end{eqnarray*}
where $\sqrt{20}\ka /(\kappa+1)\leq 2\sqrt{5}$ and $(t+\t)^{-\theta/2} (\ka+1) \log^{1/2} 2/\delta\leq 1$ for $\t^\theta \geq (\ka+1)^2 \log 2/\delta$, whence
\begin{eqnarray*}
r.h.s. & \leq & (2+4\sqrt{5}) (\kappa+1) M_\rho (t+\t)^{1/2-\theta} \log^{1/2} \frac{2}{\delta}.
\end{eqnarray*}
Note that $2+4\sqrt{5} \leq 2 + \sqrt{81}= 11$, which gives the result.

To see (\ref{eq:xjbd}) and (\ref{eq:xjvar}), note that for $\t^\theta\geq (\ka+1)^2  \geq \kappa^2+1$, 
$\|\Pi_j^t\| \leq \prod_{i=j}^t ( 1- \ga_i \la_i)$, whence
\begin{eqnarray*} 
\|\gamma_j \Pi_{j+1}^t\| & \leq & \frac{1}{\la_j} \ga_j \la_j \prod_{i=j+1}^t ( 1- \ga_i \la_i)  = \frac{1}{\la_j} \left[ \frac{1}{j+\t} \prod_{i=j+1}^t (1 - \frac{1}{i+\t})\right] , \ \ \ \ga_t \la_t = \frac{1}{t+\t} \nonumber \\
& \leq & \frac{1}{\la_j} \left[ \frac{1}{j+\t} \cdot  \frac{j+\t+1}{t+\t+1}\right], \ \ \ \ \ \mbox{by Lemma \ref{lem:pibnd}} \nonumber \\
& \leq & \frac{2}{\la_j (t+\t+1)}, \ \ \ \ \ \ \ \ \ \ \ \  \ \ \mbox{$\DS \frac{a+1}{b+1}\geq \frac{a}{b}$ for $b\geq a >0$}. \label{eq:samp1}
\end{eqnarray*}
Then it follows from Lemma \ref{lem:xi},
\[
\|X_j\|  \leq \|\ga_j \Pi_{j+1}^t\| \|\xi_j\| \leq \frac{2(\kappa+1)^2 M_\rho}{\la_j^{3/2} (t+\t+1)} \leq  
\frac{2(\kappa+1)^2 M_\rho}{\la_t^{3/2} (t+\t)} = \frac{2 \ga_t(\kappa+1)^2 M_\rho}{\sqrt{\la_t}}
\]
and
\begin{eqnarray*}
\E\|X_j\|^2 & \leq &  \frac{4}{\la_j^2 (t+\t+1)^2} \E\|\xi_j\|_K^2 \leq  \frac{40 \kappa^2 M_\rho^2}{\la_j^2 (t+\t+1)^2}   \leq \frac{40\kappa^2 M_\rho^2}{\la_t^2(t+\t)^2} 
= 40\ga_t^2 \kappa^2 M_\rho^2,
\end{eqnarray*}
as desired. 
\end{proof}



\subsection{Total Error and Proof of Theorem B}

Finally we derive Theorem B by combining the four error bounds obtained above.

\begin{proof}[Proof of Theorem B] By
\[ \|f_t - f_\rho\|_K \leq \Err_{approx}(t) + \Err_{drift}(t)+ \Err_{init}(t)+\Err_{samp}(t) \]
with Theorem \ref{err:approx}, \ref{err:drift}, \ref{err:init}, and \ref{err:samp}, we obtain
\[ \|f_t - f_\rho\|_K \leq (C_1 +C_2)(t+\t)^{-(r-1/2)(1-\theta)}+ C_3 (t+\t)^{-1} + C_4 (t+\t)^{1/2-\theta}. \]
Setting $\theta = 2r/(2r+1)$, we obtain 
\begin{equation} \label{eq:hk}
\|f_t - f_\rho \|_K \leq  (C_1+C_2 + C_4) t^{-(2r-1)/(4r+2)}+  C_3 t^{-1}. 
\end{equation}

Finally,
\[ C_1 + C_2 =  \left(\frac{2}{2r-1}+\frac{8}{2r+3}  \right)\| L^{-r}_K f_\rho \|_\rho=\frac{20r-2}{(2r-1)(2r+3)}\| L^{-r}_K f_\rho \|_\rho. \]
By Lemma \ref{lem:xi}(A) with $f_0=0$
\[ \|r_0\|=\|f_{\la_0}\|\leq \frac{M_\rho}{\sqrt{\la_0}} = \t^{\frac{1}{4r+2}} M_\rho \]
whence $C_3 = (\t+1)\|f_{\la_0}\|\leq 2\t^{\frac{4r+3}{4r+2}} M_\rho$. We end the proof by plugging these constants into (\ref{eq:hk}). 
\end{proof}


%%%%%%%%%%%%%%%%%%%%%%%%%%%%%%%%%%%%%%%%%%%%%%%%%%%%%%%%%%%%%%%%%%%%%%%%%%%%%%%%%%%%%%%%%%%

\section{Upper Bounds for Convergence in $\L2$} \label{sec:l2}

%%%%%%%%%%%%%%%%%%%%%%%%%%%%%%%%%%%%%%%%%%%%%%%%%%%%%%%%%%%%%%%%%%%%%%%%%%%%%%%%%%%%%%%%%%%
%%%%%%%%%%%%%%%%%%%%%%%%%%%%%%%%%%%%%%%%%%%%%%%%%%%%%%%%%%%%%%%%%%%%%%%%%%%%%%%%%%%%%%%%%%%%%%%%%%%%%%

In this section we are going to give a probabilistic upper bound for
\[ \|f_t - f_\rho \|_\rho \]
as a proof for Theorem C. Throughout this section, we assume 
that $L_K^{-r} f_\rho\in \L2$ for some $r\in [1/2,3/2]$, implying $f_\rho \in \H_K$ with additional regularity. Note that the case $r=1/2$ is not included
in the study of convergence rates for $\|f_t - f_\rho \|_K$. 

Similar to Section \ref{sec:hk}, the idea in the proof starts from the triangle inequality 
\[\|f_t - f_\rho\|_\rho \leq \|f_t - f_{\la_t}\|_\rho+\|f_{\la_t} - f_\rho\|_\rho. \] 
But for the first term, instead of using the reversed martingale decomposition, here we need the martingale decomposition in Theorem \ref{thm:mart}, 
which can be rewritten as
\[ r_{t} = \PPi_1^t r_0 + \sum_{j=1}^t \gamma_j \PPi_{j+1}^t \chi_j - \sum_{j=1}^t \PPi_{j}^t \Delta_j \]
where $\chi_t=( L_K-L_t ) f_{t-1} +( y_t K_{x_t}-L_K f_\rho )$ ($L_t:=L_K^{x_t}=\<\ ,K_{x_t}\>_K K_{x_t}$), $\Delta_t = f_{\la_t} - f_{\la_{t-1}}$, and
\begin{equation} 
\PPi_j^t=
\left\{
\begin{array}{lr}
\DS \prod_{i=j}^t \left( I - \gamma_i  (L_K + \la_i I)  \right), & j\leq t; \\
I, & j>t.
\end{array}
\right.
\end{equation}
with the choice that
\[ \gamma_t = \frac{1}{(t+\t)^{\theta}}, \ \ \ \ \ \lambda_t = \frac{1}{(t+\t)^{1-\theta}} .\]
The reason of using such a decomposition, is that due to the isometry $L_K^{1/2}:\L2 \to \H_K$ such that $\|r_t\|_\rho=\|L_K^{1/2} r_t \|_K$, 
one can benefit from the spectral decomposition of $L_K^{1/2}\PPi_j^t$ to get a tighter estimate. However such a nice feature is lost in the reversed 
martingale decomposition in that $L_K^{1/2} \Pi_j^t$ does not have an obvious spectral decomposition. On the other hand, due to $\chi_t$ depends on 
$f_{t-1}$, which increases the difficulty to estimate $\|\chi_t\|_\rho$. For this reason, we can not directly apply Pinelis-Bernstein's inequality to improve 
Theorem C by replacing $1/\delta$ with $\log 1/\delta$. 

As in Section \ref{sec:hk}, we make the following definitions for convenience.

{\bf [Definitions of Errors]}

\noindent (A) \emph{Initial Error}: $\Err_{init}(t)=\|\PPi_1^t r_0\|_\rho $, which reflects the propagation error by the initial choice $f_0$; \\
\noindent (B) \emph{Sample Error}: $\Err_{samp}(t)=\|\sum_{j=1}^t \gamma_j \PPi_{j+1}^t \chi_j \|_\rho $, where $\chi_j$ 
is a martingale difference sequence, reflecting the random fluctuation caused by sampling; \\
\noindent (C) \emph{Drift Error}: $\Err_{drift}(t)= \|\sum_{j=1}^t \PPi_{j}^t \Delta_j\|_\rho $, which measures the error caused by drifts from 
$f_{\la_{j-1}}$ to $f_{\la_j}$ along the regularization path; \\
\noindent (D) \emph{Approximation Error}: $\Err_{approx}(t)=\|f_{\la_t} - f_\rho\|_\rho$, which measures the distance between the regression function and
the regularization path at time $t$.

In this way we may bound
\[ \|f_t - f_\rho\|_\rho \leq \Err_{init}(t) + \Err_{samp}(t) + \Err_{drift}(t) + \Err_{approx}(t). \]
In the remaining of this section, we are going to provide upper bounds for each of the four errors, which, roughly speaking, are
\begin{eqnarray*} 
\Err_{approx}(t)  & \leq & O( t^{-r(1-\theta)} ) \\
 \Err_{drift}(t) & \leq & O( t^{-r(1-\theta)}) \\
 \Err_{init}(t) & \leq & O( t^{-1})  \\
 \Err_{samp}(t) & \leq & O( t^{-\theta/2} )
\end{eqnarray*}

Theorem C then follows from these bounds by setting $\theta=2r/(2r+1)$.

\subsection{Approximation Error}

\begin{thm}[Approximation Error]\label{err1:approx} For $r\in (0,1]$,
\[ \Err_{approx}(t)\leq C_5 (t+\t)^{-r(1-\theta)}, \]
where $C_5=r^{-1}\| L^{-r}_K f_\rho \|_\rho$.
\end{thm}
\begin{proof}
It follows from Theorem \ref{thm:pie1}(A) with $\la=\la_t$ and $\mu=0$. 
\end{proof}

\subsection{Drift Error}

\begin{thm}[Drift Error]\label{err1:drift} Assume $\t^\theta \geq \kappa^2+1$. For $r\in (0,1]$ and $L_K^{-r}f_\rho \in \L2$, 
\[ \Err_{drift}(t)\leq C_6 (t+\t)^{-r(1-\theta)} \]
where $\DS C_6=\frac{4(1-\theta)}{1-r(1-\theta)}\|L_K^{-r} f_\rho\|_\rho$.
\end{thm}

\begin{proof} By Theorem \ref{thm:pie1}(A), it follows that
\[ \|\Delta_t\| = \|f_{\lambda_t}-f_{\lambda_{t-1}}\|_\rho \leq 4(1-\theta) (t+\t)^{-r(1-\theta)-1} \|L_K^{-r} f_\rho\|_\rho, \]
use inequality (\ref{eq:meanvaluethm}) by replacing $r-1/2$ with $r$.
By Lemma \ref{lem:pibnd},
\[ \|\Pi_{j}^t\|\leq \frac{j+\t}{t+\t+1}, \]
whence
\begin{eqnarray*}
\Err_{drift}(t) &=& \| \sum_{j=1}^t \Pi_j^t \Delta_j\|_K \leq \frac{4(1-\theta)\|L_K^{-r} f_\rho\|_\rho }{t+\t+1}\sum_{j=1}^t 
 (j+\t)^{-r(1-\theta)}\\
 & \leq & \frac{4(1-\theta)\|L_K^{-r} f_\rho\|_\rho }{1-r(1-\theta)}  (t+\t)^{-r(1-\theta)}
\end{eqnarray*}
since
\[ \sum_{j=1}^t 
 (j+\t)^{-r(1-\theta)} \leq   \int_0^t (x+\t)^{-r(1-\theta)} dx \leq
\frac{1}{1-r(1-\theta)} (t+\t)^{1-r(1-\theta)} 
\]
\end{proof}

\subsection{Initial Error}

\begin{thm}[Initial Error]\label{err1:init} Let $a=1$ and $\t\geq (\ka^2+1)^{1/\theta}$. Then for all $t\in \N$,
\[ \Err_{init}(t) \leq C_7 (t+\t)^{-1}, \]
where $C_7 = 2\M (\t+1)$.
\end{thm}
\begin{proof}
Similar to Theorem \ref{err:init}, by Lemma \ref{lem:onebnd}(D) with $j=1$ we obtain 
\[ \|\PPi_1^t\| \leq \frac{\t+1}{t+\t}, \]
which gives
\[ \Err_{init}(t) \leq (\t+1)\|r_0\| (t+\t)^{-1}. \]
For $f_0=0$, using Lemma \ref{lem:fla}(B), $\|r_0 \|_\rho = \|f_{\la_0}\|_\rho \leq 2 \M$. 
\end{proof}


\subsection{Sample Error}

\begin{thm}[Sample Error]\label{err1:samp} Assume that $L_K^{-r} f_\rho\in \L2$ for some $r\in [1/2,1]$ and $\t^\theta \geq \ka^2 +1$.
Then with probability at least $1-\delta$ ($\delta\in (0,1)$), there holds for all $t\in \N$,
\[ \Err_{samp}(t)\leq  C_8 t^{-\theta/2} \]
where 
\[ C_8=\sqrt{\frac{2}{\delta}} \ka (3 \M + 2\|L_K^{-r} f_\rho \|_\rho + \sqrt{3 (4\ka^2+1)} \ka \M ) \]
\end{thm}

Before presenting the formal proof, we need an auxillary estimate.

\begin{lem} \label{lem:chivar} Assume that $L_K^{-r} f_\rho\in \L2$ for some $r\in [1/2,1]$ and $\t^\theta \geq \ka^2 +1$. Then for all $t\in \N$, there holds 
\[ \E \|\chi_t\|_K^2 \leq C_{9} . \]
where
\[ C_{9} = 2\ka^2 (9 \M^2 + 4\|L_K^{-r} f_\rho \|_\rho^2 + 3 (4\ka^2+1) \ka^2 \M^2 ) \]
\end{lem}

\begin{proof} By definition 
\[ \chi_t = (\A_t - A_t)w_{t-1} + b_t - \b_t = (L_K - L_t)f_{t-1} + y_t K_{x_t} - L_K f_\rho \]
where $L_t := \<\ , K_{x_t}\>K_{x_t} $. Then
\[ \E\|\chi_t\|_K^2 \leq 2 \E\|(L_K - L_t)f_{t-1} \|_K^2  +2 \E \|y_t K_{x_t} - L_K f_\rho \|_K^2 \leq 2 \E\| L_t f_{t-1}\|_K^2  +2 \E \|y_t K_{x_t} \|_K^2 \]
using for $\E[X]=\mu$, $\E\<X-\mu,X -\mu\> = \E\|X\|^2 - \|\mu\|^2 \leq \E\|X\|^2$, with the replacement that $X=L_t$ and $\mu=L_K$, or that 
$X=y_t K_{x_t}$ and $\mu=L_K f_\rho$, respectively. 

Note that the second term $\E\|y_t K_{x_t}\|_K^2 \leq \ka^2 \M^2$. It remains to bound the first term,
$$\E\| L_t f_{t-1}\|_K^2 = \E \|f_{t-1}(x_t) K_{x_t}\|_K^2 \leq \ka^2 \E \|f_{t-1}\|_\rho^2 \leq 
2 \ka^2 (\E \|f_{t-1} - f_{\la_{t-1}}\|_\rho^2 + \|f_{\la_{t-1}}\|_\rho^2), $$ 
where using Lemma \ref{lem:fla}, $\|f_{\la}\|_\rho \leq 2 \M$, and Corollary \ref{cor:rvarbnd} for the bound on $\E\|f_{t-1}-f_{\la_{t-1}}\|_\rho^2$, 
\[ r.h.s. \leq 2 \ka^2(8 \M^2 + 4\|L_K^{-r} f_\rho \|_\rho^2 + 3 (4\ka^2+1) \ka^2 \M^2 ) . \]
Putting two terms together gives the result.
\end{proof}

Now we are ready to prove the bound for the sample error. 

\begin{proof}[Proof of Theorem \ref{err1:samp}] Denote $X_j = \gamma_j \PPi_{j+1}^t \chi_j$, which is a martingale difference sequence. 
It suffices to show
\begin{equation} \label{eq:xjrhovar}
\E[ \|\sum_{j=1}^t X_j\|_\rho^2] \leq C_{9} (t+\t)^{-\theta}. 
\end{equation}
Then it follows from the Markov inequality 
\[ \Prob \left\{(z_i)_1^t\in \Z^t: \|\sum_{j=1}^t X_j\|_\rho\geq \eps \right\}\leq \frac{\E[ \|\sum_{j=1}^t X_j\|_\rho^2]}{\eps^2} \leq 
\frac{C_{9}}{\eps^2 }t^{-\theta}. \] 
Setting the right hand side to be $\delta$, and noticing that
$$\sqrt{\frac{C_{9}}{\delta}}= \sqrt{\frac{2}{\delta}} \ka  (9 \M^2 + 4\|L_K^{-r} f_\rho \|_\rho^2 + 3 (4\ka^2+1) \ka^2 \M^2 )^{1/2} \leq  
\sqrt{\frac{2}{\delta}} \ka (3 \M + 2\|L_K^{-r} f_\rho \|_\rho + \sqrt{3 (4\ka^2+1)} \ka \M )$$
using $(a^2+b^2+c^2)^{1/2} \leq a + b + c$ for $a,b,c>0$, we obtain the result. 

It remains to prove (\ref{eq:xjrhovar}). By isometry $L_K^{1/2}:\L2\to \H_K$, 
\begin{eqnarray*} 
\E[ \|\sum_{j=1}^t X_j\|_\rho^2 ]& = & \E \| L_K^{1/2} \sum_{j=1}^t X_j\|_K^2  = 
\sum_{j=1}^t \gamma_j^2 \E \| L_K^{1/2}  \PPi_{j+1}^t \chi_j\|_K^2 \\
& \leq & \sum_{j=1}^t \gamma_j^2 \| \PPi_{j+1}^t L_K \PPi_{j+1}^t\|\cdot \E\| \chi_j\|_K^2 
\end{eqnarray*}
Using Lemma \ref{lem:chivar}, we have $\E\|\chi_j\|_K^2 \leq  C_{9}$. 
To estimate $\sum_{j=1}^t \gamma_j^2 \| \PPi_{j+1}^t L_K \PPi_{j+1}^t\|$, we use the spectral decomposition of $L_K:\L2\to \L2$. 
Let $(\mu_\al, \phi_\al)_{\al\in \N}$ be an orthonormal eigen-system of $L_K$, by Mercer's Theorem. For simplicity, denote $a_i = \ga_i \la_i + \ga_i \mu_\al$, then
\begin{eqnarray*}
&& \sum_{j=1}^t \gamma_j^2 \| \PPi_{j+1}^t L_K \PPi_{j+1}^t\| \leq \sup_{\mu_\al} \sum_{j=1}^t \ga_j^2 \mu_\al \prod_{i=j+1}^t (1-a_i)^2 \\
& = & \sup_{\mu_\al} \sum_{j=1}^t \left [\ga_j  \prod_{i=j+1}^t (1-a_i) \right]  \cdot \left[\ga_j \mu_\al \prod_{i=j+1}^t (1-a_i)\right] \\
& \leq & \sup_{\mu_\al} \left\{\left [ \sup_{j}\ga_j  \prod_{i=j+1}^t (1-a_i) \right] \cdot
\left[ \sum_{j=1}^t \ga_j \mu_\al \prod_{i=j+1}^t (1-a_i)\right] \right\}
\end{eqnarray*}
where for large enough $\t$, 
\begin{equation}  \label{eq:sup}
\sup_j \ga_j  \prod_{i=j+1}^t (1-a_i)\leq  \sup_j \ga_j  \prod_{i=j+1}^t (1- \ga_i \la_i) \leq \sup_j \frac{1}{(j+\t)^\theta}\cdot \frac{j+\t}{t+\t+1} 
\leq (t+\t)^{-\theta},
\end{equation}
and 
\[ \sum_{j=1}^t \ga_j \mu_\al \prod_{i=j+1}^t (1-a_i) \leq  \sum_{j=1}^t (1 - (1-\ga_j \mu_\al))\prod_{i=j+1}^t (1-\ga_i \mu_\al ) =
1 - \prod_{i=1}^t(1-\ga_i \mu_\al ) \leq 1,\]
which gives (\ref{eq:xjrhovar}).
\end{proof}

\begin{rem} \label{rem:open}
It is still an open problem, if we can improve this bound to replace $1/\de$ with $\log 1/\de$, 
by using the Pinelis-Bernstein inequality for the martingale difference sequence. 
The difficulty seems that, the Pinelis-Bernstein inequality needs a uniform bound on $\|\gamma_j \PPi_{j+1}^t \chi_j\|_\rho$,
which so far is only $O(t^{-(1-2\theta)})$, by Lemma \ref{lem:ftk} such that $\|\chi_t\|_K \leq O(\|f_{t-1}\|_K) \leq O(1/\la_t)$ 
and $\|\ga_j L_K^{1/2} \PPi_{j+1}^t\|\leq O(t^{-\theta})$ as in the proof above. 
Then using the Pinelis-Bernstein inequality in Proposition \ref{prop:bernstein}, we have
\[ \Err_{samp}(t) \leq O(t^{-(1-2\theta)}) + O(t^{-\theta/2}), \]
where the first term has a decreasing rate slower than $O(t^{-\theta/2})$ when $\theta=2r/(2r+1)$ for $r\in [1/2,1]$. 
The successful application of Pinelis-Bernstein, may rely on an improved estimate $\|f_t\|_\rho \leq O(1/\sqrt{\la_t})$, 
which is still open at this moment.  
\end{rem}

\subsection{Total Error and the Proof of Theorem C}


Choosing $\theta\in [0,1]$ properly, we obtain the Theorem C. 

\begin{proof}[Proof of Theorem C] By triangle inequality,
\[ \|f_t - f_\rho \|_\rho \leq \Err_{approx}(t) + \Err_{drift}(t) + \Err_{init}(t) + \Err_{samp}(t)  \]
Combining Theorem \ref{err1:approx}, \ref{err1:drift}, \ref{err1:init}, and \ref{err1:samp}, and setting $\theta = 2r/(2r+1)$, we obtain
\[ \|f_t - f_\rho\|_\rho \leq (C_5 + C_6 +C_8) (t+\t)^{-r/(2r+1)} + C_7 (t+\t)^{-1} \]
where
\[ C_5+ C_6 = \left(\frac{1}{r} + \frac{4}{r+1}\right)\|L_K^{-r} f_\rho\|_\rho = \frac{5r+1}{r(r+1)}\|L_K^{-r} f_\rho\|_\rho \]
whence 
\begin{eqnarray*} 
C_5+C_6+C_8 & = & \sqrt{\frac{6}{\delta}} \ka \M (\sqrt{3}  + \ka \sqrt{4\ka^2+1} )+ \left(2\sqrt{2}\ka\sqrt{\frac{1}{\delta}}+\frac{5r+1}{r(r+1)}\right)\|L_K^{-r} f_\rho \|_\rho,
\end{eqnarray*}
which ends the proof.
\end{proof}


\section*{Appendix A: A Probabilistic Inequality}
\renewcommand{\thesection}{A}
\setcounter{equation}{0} \setcounter{thm}{0}
\renewcommand{\thethm}{A.\arabic{thm}}
\renewcommand{\theequation}{A-\arabic{equation}}

The following result is quoted from [Theorem 3.4 in \citeNP{Pinelis94}].

\begin{lem}[Pinelis-Bennett]
Let $\xi_i$ be a martingale difference sequence in a Hilbert space. Suppose that almost surely $\|\xi_i\|\leq M$ and $\E \|\xi_i\|^2 \leq \sigma^2$.
Then
\[ \Prob \left\{ \left\|\sum_{i=1}^t \xi_i \right\| \geq \epsilon \right\} \leq
2 \exp \left\{-\frac{\sigma^2}{M^2} g \left(\frac{M\epsilon}{\sigma^2} \right) \right\}, \]
where $g(x) = (1+x)\log(1+x)-x$ for $x>0$.
\end{lem}

Taking $g(x)\geq \frac{x^2}{2(1+x/3)}$ and replacing $\epsilon$ by $t\epsilon$, gives the following generalized Bernstein's inequality.

\begin{cor}[Pinelis-Bernstein]
Let $\xi_i$ be a martingale difference sequence in a Hilbert space. Suppose that almost surely $\|\xi_i\|\leq M$ and $\E \|\xi_i\|^2 \leq \sigma^2$.
Then
\begin{equation} \label{eq:bernstein}
\Prob \left\{ \left\|\frac{1}{t}\sum_{i=1}^t \xi_i \right\| \geq \epsilon \right\} \leq
2 \exp \left\{-\frac{t\epsilon^2}{2(\sigma^2+M\epsilon/3)} \right\}.
\end{equation}
\end{cor}

The following result will be used as a basic probabilistic inequality to derive various bounds. 

\begin{prop} \label{prop:bernstein}
Let $\xi_i$ be a martingale difference sequence in a Hilbert space. Suppose that almost surely $\|\xi_i\|\leq M$ and $\E \|\xi_i\|^2 \leq \sigma^2$.
Then the following holds with probability at least $1-\delta$ ($\delta\in (0,1)$),
\[ \left\|\sum_{i=1}^t \xi_i \right\| \leq M \log\frac{2}{\delta} + \sqrt{2t\sigma^2 \log\frac{2}{\delta}}.   \]
\end{prop}

\begin{proof}
Taking the right hand side of (\ref{eq:bernstein}) to be $\delta$, then we arrive at the following quadratic equation for $\epsilon$,
\[ \epsilon^2 - \frac{2 M }{3t}\epsilon \log\frac{2}{\delta} - \frac{2 \sigma^2}{t} \log\frac{2}{\delta} = 0. \]
Note that $\epsilon>0$, then
\begin{eqnarray*}
\epsilon & = & \frac{1}{2} \left\{ \frac{2M}{3t}   \log\frac{2}{\delta} + \sqrt{\frac{4M^2}{9t^2}   \log^2\frac{2}{\delta} + \frac{8 \sigma^2}{t} \log\frac{2}{\delta}} \right\} \\
& = & \frac{M}{3t}   \log\frac{2}{\delta} + \sqrt{\left(\frac{M}{3 t}\right)^2   \log^2\frac{2}{\delta} + \frac{2 \sigma^2}{t} \log\frac{2}{\delta}} \\
& \leq & \frac{2M}{3 t}   \log\frac{2}{\delta}+ \sqrt{ \frac{2 \sigma^2}{t} \log\frac{2}{\delta}},
\end{eqnarray*}
where the second last step is due to $\sqrt{a^2 + b^2}\leq a + b$ ($a,b>0$) with 
$$a= \frac{M}{3 t}\log\frac{2}{\delta},\ \ \ \mbox{and} \ \ \  b=\sqrt{\frac{2\sigma^2}{t} \log\frac{2}{\delta}}. $$
This gives
\[ \left\|\frac{1}{t}\sum_{i=1}^t \xi_i \right\| \leq \frac{2M}{3t}\log\frac{2}{\delta} + \sqrt{\frac{2\sigma^2}{t} \log\frac{2}{\delta}}.   \]
Multiplying both sides by $t$ and relaxing $2M/3$ by $M$, we obtain the result.  
\end{proof}



%%%%%%%%%%%%%%%%%%%%%%%%%%%%%%%%%%%%%%%%%%%%%%%%%%%%%%%%%%%%%%%%%%%%%%%%%%%%%%%%%%%%%%%%%

\section*{Appendix B: Basic Estimates}
\renewcommand{\thesection}{B}
\setcounter{equation}{0} \setcounter{thm}{0}
\renewcommand{\thethm}{B.\arabic{thm}}
\renewcommand{\theequation}{B-\arabic{equation}}


\begin{lem} \label{lem:pibnd} For all $\t>0$ and $a>0$,
\[ \prod_{i=j}^t \left(1 - \frac{a}{i+\t} \right) \leq \left(\frac{j+\t}{t+\t+1} \right)^a \]
\end{lem}

\begin{proof}
\begin{eqnarray*}
\prod_{i=j}^t \left(1 - \frac{a}{i+\t} \right)  & \leq & \exp\{ - a \sum_{i=j}^t (i+\t)^{-1} \} \\
& \leq & \exp\{-a\int_{j+\t}^{t+\t+1} x^{-1} d x\} \\
& = & \exp \left\{\ln\left(\frac{j+\t}{t+\t+1} \right)^a
\right\} =\left(\frac{j+\t}{t+\t+1} \right)^a
\end{eqnarray*}
\end{proof}

\begin{lem} \label{lem:onebnd} If $\t^\theta \geq \kappa^2+1$, then the following holds for all $t\in \N$,

(A) $\DS \|I - \gamma_t A_t \|\leq 1 - \frac{1}{t+\t}$;

(B) $\DS \|\Pi_j^t\| \leq \frac{j+\t}{t+\t+1} $;

(C) $\DS \|I - \gamma_t \A_t \|\leq 1 - \frac{1}{t+\t}$;

(D) $\DS \|\PPi_j^t\| \leq \frac{j+\t}{t+\t+1} $.
\end{lem}

\begin{proof}
(A) First we show that $\gamma_t (\kappa^2 + \lambda_t)< 1$. In fact, for $t\in \N$,
$$ \gamma_t \lambda_t + \gamma_t \kappa^2 = \frac{1 }{t+\t}+\frac{\kappa^2}{(t+\t)^{\theta}} 
\leq  \frac{1+\kappa^2}{\t^\theta} \leq 1. $$
Therefore
\[ \| I - \ga_t A_t \|= \| I - \ga_t L_t - \ga_t \la_t \| \leq 1-\ga_t \la_t = 1 - \frac{1}{t+\t} , \]
since $\|L_t\|\leq \ka^2$. 

(B) To show this, 
\[  \|\Pi_j^t\| = \| \prod_{i=j}^t (I-\ga_i A_i) \| \leq \prod_{i=j}^t (1-\ga_i \la_i) = \prod_{i=j}^t (1 - \frac{1}{i+\t})\leq \frac{j+\t}{t+\t+1} \]
where the last step is due to Lemma \ref{lem:pibnd}. 

(C) and (D) are similar to (A) and (B), respectively.
\end{proof}


\begin{lem} \label{lem:fla}For any $\lambda>0$,

(A) $\|f_\lambda\|_K \leq M_\rho/\sqrt{\lambda}$;

(B) $\|f_\lambda\|_\rho \leq 2M_\rho$.
\end{lem}

\begin{proof}
(A) Note that
\[ f_\lambda = \arg \min_{f\in \H_K} \|f - f_\rho\|_\rho^2 + \lambda \|f\|^2_K. \]
Taking $f=0$, we have
\begin{equation} \label{eq:varbnd1}
\|f_\lambda - f_\rho\|_\rho^2 + \lambda \|f_\lambda\|^2_K \leq \|f_\rho\|_\rho^2 \leq M_\rho^2,
\end{equation}
which leads to the result.

(B) From equation (\ref{eq:varbnd1}), we obtain $\|f_\lambda - f_\rho\|_\rho \leq M_\rho$. The result then follows from
\begin{eqnarray*}
\|f_\lambda\|_\rho & \leq & \| f_\lambda - f_\rho \|_\rho + \|f_\rho \|_\rho \leq 2M_\rho.
\end{eqnarray*}
\end{proof}

\begin{lem} \label{lem:ftk} If $f_0=0$, then for all $t\in \N$, 
\[ \|f_t\|_K \leq \frac{\ka \M}{\la_t} \]
\end{lem}

\begin{proof}
Since
\[ f_t = f_{t-1} - \ga_t ((f_{t-1}(x_t)-y_t)K_{x_t} + \la_t f_{t-1})=(1 -\ga_t \la_t - \ga_t L_K^{x_t}) f_{t-1} + \ga_t y_t K_{x_t} \]
then for $\t \geq [(1+\ka^2)]^{(2r+1)/2r}$, $\ga_t \ka^2 + \ga_t \la_t \leq 1$, whence
\[ \|f_t \|_K \leq \|1 -\ga_t \la_t - \ga_t L_K^{x_t}\| \|f_{t-1}\|_K + \ga_t \|y_t K_{x_t}\|_K \leq (1-\ga_t \la_t) \|f_{t-1}\|_K + \ga_t \ka \M. \]
By induction on $t$, we have
\[ \|f_t \|_K \leq \prod_{i=1}^t (1-\ga_i \la_i) \|f_0\|_K + \ka \M \sum_{j=1}^t \ga_j \prod_{i=j+1}^t (1 - \ga_i \la_i). \]

The first term is $0$ since $f_0=0$. In the second term
\[\sum_{j=1}^t \ga_j \prod_{i=j+1}^t (1 - \ga_i \la_i ) = \max_{1\leq j\leq t}(\frac{1}{\la_j}) \sum_{j=1}^t \ga_j\la_j  \prod_{i=j+1}^t (1 - \ga_i \la_i )\leq \frac{1}{\la_t} \]
since
$$\sum_{j=1}^t \ga_j\la_j  \prod_{i=j+1}^t (1 - \ga_i \la_i ) = 1 - \prod_{i=1}^t (1 - \ga_i \la_i ). $$
This gives the bound. 
\end{proof}


\section*{Appendix C: Estimates for the Path Gap} 
\renewcommand{\thesection}{C}
\setcounter{equation}{0} \setcounter{thm}{0}
\renewcommand{\thethm}{C.\arabic{thm}}
\renewcommand{\theequation}{C-\arabic{equation}}

In this section we derive some direct estimates for the remainder variance $\E \|f_t - f_{\la_t}\|^2$, in $\H_K$ norm or $\L2$ norm. The result
in Lemma \ref{lem:expan} will be used to derive a constant upper bound for $\E \|f_t - f_{\la_t}\|_\rho^2$ when $L_K^{-r} f_\rho\in \L2$ with $r\in [1/2,1]$,
i.e. Corollary \ref{cor:rvarbnd}. Another application of Lemma \ref{lem:expan}, which is not pursued in this paper, is that it can be used to derive Theorem A directly. 


\begin{lem} \label{lem:pie2}
Let $f\in \H_K$, and $\ga,\la\in \R_+$. For all $z=(x,y)\in X\times \R$, let
\[ f^z:= f - \ga ((f(x)-y)K_x + \la f) = f - \ga ((L_K^x+\la f) - g^z), \ \ \ \mbox{where $g^z:=y K_x$}. \]
Then
\[ \E\|f^z-f_\lambda\|_\st^2 \leq (1 - \ga \la )^2 \|f-f_\la\|_\st^2  - 2 \ga (1 - \ga \la - 2\ga \ka^2) \|L_K^{1/2}(f-f_\la)\|_\st^2 + \ga^2 C_\star . \]
where $\st$ stands for either $\rho$ or $K$, and 
\[C_\st = \left\{\begin{array}{rl}
15 \ka^2 \M^2, & \st = K \\
3 (4\ka^2+1) \ka^2 \M^2 , & \st=\rho
\end{array}
\right.
\]
\end{lem}

\begin{proof}
Using the expression $\la f_\la = L_K f_\rho - L_K f_\la$ from $(L_K + \la I ) f_\la = L_K f_\rho$, 
\begin{eqnarray*}
f^z-f_\la & = & f - f_\la -\ga ((L_K^x + \la) f - g^z) \\
& = & [ I - \ga (L_K + \la I) ](f-f_\la) + \ga(L_K +\la I) (f - f_\la) - \ga ((L_K^x + \la) f - g^z) \\
& = & [ I - \ga (L_K + \la I) ](f-f_\la) + \ga(L_K -L_K^x) f  - \ga (L_K f_\rho - g^z)
\end{eqnarray*}

Therefore
\begin{equation}  \label{eq:pierre1}
\E\|f^z - f_\la\|_\st^2 = \|[ I - \ga (L_K + \la I) ](f-f_\la)\|_\st^2 + \ga^2 \zeta(f) 
\end{equation}
where
\[ \zeta(f):= \E[\|(L_K-L_K^x)f - (L_K f_\rho - g^z)\|_\st^2] \]
since
\[ \E[(L_K-L_K^x)f - (L_K f_\rho - g^z)]=0. \]

Let us now study the two terms in equation (\ref{eq:pierre1}). First,
\begin{eqnarray*}
& & \|[ I - \ga (L_K + \la I) ](f-f_\la)\|_\st^2  \\
& & = (1-\ga \la)^2 \|f-f_\la\|_\st^2 -2 \ga (1-\ga\la) \<L_K (f-f_\la),f-f_\la\>_\st + \ga^2 \|L_K (f-f_\la)\|_\st^2 \\
& & \leq   (1-\ga \la)^2 \|f-f_\la\|_\st^2 - 2 \ga (1-\ga\la) \|L_K^{1/2} (f-f_\la)\|_\st^2 + \ga^2 \ka^2 \|L_K^{1/2} ( f-f_\la)\|_\st^2 
\end{eqnarray*} 

It remains to estimate $\ze(f)$:
\begin{eqnarray*}
\ze(f) & =  & \E [ \|(L_K-L_K^x)(f -f_\la)+ (L_K - L_K^x)f_\la+ (L_K f_\rho - g^z)\|_\st^2 ] \\
& \leq & 3 \E[\|(L_K-L_K^x)(f -f_\la)\|_\st^2+ \|(L_K - L_K^x)f_\la\|_\st^2+ \|(L_K f_\rho - g^z)\|_\st^2] \\
& \leq & 3 \E [\|L_K^x(f-f_\la)\|_\st^2 + \|L_K^x f_\la \|_\st^2 + \|g^z\|_\st^2 ] \\
& \leq & 3 \kappa^2 [\|L_K^{1/2}(f-f_\la)\|_\st^2 + \|L_K^{1/2} f_\la \|_\st^2 + \M^2]  
\end{eqnarray*}
where if $\st=K$, 
\[ r.h.s. = 3 \kappa^2 [\|L_K^{1/2}(f-f_\la)\|_K^2 + \|L_K^{1/2} f_\la \|_K^2 + \M^2] \leq 3 \kappa^2 \|f-f_\la\|_\rho^2  + 15 \ka^2 \M^2 \]
using $\|L_K^{1/2} f_\la\|_K = \|f_\la\|_\rho\leq 2 \M$, and if $\st=\rho$,
\[ r.h.s. = 3 \kappa^2 [\|L_K^{1/2}(f-f_\la)\|_\rho^2 + \|L_K^{1/2} f_\la \|_\rho^2 + \M^2] \leq 3 \kappa^2 \|L_K^{1/2}(f-f_\la)\|_\rho^2  + 3 (4\ka^2+1) \ka^2 \M^2 \]
using $\|L_K^{1/2} f_\la\|_\rho = \ka \|f_\la\|_\rho\leq 2 \ka \M$. Combining the two estimates gives the result. 
\end{proof}

\begin{lem} \label{lem:expan}
Let 
\begin{equation} \pi_k^t=
\left\{
\begin{array}{lr}
\DS \prod_{i=k}^t \left( 1 - \gamma_i \la_i\right), & k\leq t; \\
I, & k>t.
\end{array}
\right.
\end{equation}

If for all $t\in \N$, $\ga_t (\la_t + 2 \ka^2) \leq 1$, then
\[ \E \|f_t - f_{\la_t}\|_\st^2 \leq \pi_1^t \|f_0 - f_{\la_0}\|_\st^2 + 
\sum_{k=1}^t \pi_{k+1}^t \frac{\|f_{\la_k}-f_{\la_{k-1}}\|_\st^2}{\ga_k\la_k} + C_\st \sum_{k=1}^t \ga_k^2\pi_{k+1}^t.\]
where $\st$ stands for either $\rho$ or $K$, and 
\[C_\st = \left\{\begin{array}{rl}
15  \ka^2 \M^2, & \st = K \\
3 (4\ka^2+1) \ka^2 \M^2 , & \st=\rho
\end{array}
\right.
\]
\end{lem}

\begin{proof}
Using Lemma \ref{lem:pie2}, for $\ga_t (\la_t + 2 \ka^2) \leq 1$,
\[ \E \|f_t - f_{\la_t}\|_\st^2 \leq  (1 - \ga_t \la_t )^2 \|f_{t-1}-f_{\la_{t}}\|_\st^2  + C_\st.\]
Note that 
\begin{eqnarray*} 
\|f_{t-1}-f_{\la_{t}}\|_\st & \leq & \|f_{t-1}-f_{\la_{t-1}}\|_\st + \|f_{\la_t} - f_{\la_{t-1}}\|_\st \\
& = &  \|f_{t-1}-f_{\la_{t-1}}\|_\st + \delta_t, \ \ \ \ \ \mbox{define $\DS \delta_t:=\|f_{\la_t} - f_{\la_{t-1}}\|_\st$} \\
& \leq &  \|f_{t-1}-f_{\la_{t-1}}\|_\st^2 ( 1 + \ga_t \la_t) + \delta_t^2 (1+ 1/(\ga_t\la_t))
\end{eqnarray*}
using that, for all $a,b,c\in \R_+$,
\[ (a+b)^2 \leq a^2(1+c)+b^2(1+1/c) \]
with $x:=\|f_{t-1}-f_{\la_{t-1}}\|_\st$, $a:=\delta_t$ and $b:=\ga_t\la_t$. 

This gives the iteration formula,
\[ \E \|f_t - f_{\la_t}\|_\st^2 \leq  (1 - \ga_t \la_t ) \|f_{t-1}-f_{\la_{t-1}}\|_\st^2 + (1-\ga_t \la_t)\frac{\delta_t^2}{\ga_t\la_t}+  \ga_t^2 C_\st,\]
which, by induction, leads to 
\[ \E \|f_t - f_{\la_t}\|_\st^2 \leq \pi_1^t \|f_0 - f_{\la_0}\|_\st^2 + \sum_{k=1}^t \pi_{k+1}^t \frac{\delta_k^2}{\ga_k\la_k} + 
C_\st \sum_{k=1}^t \ga_k^2\pi_{k+1}^t\]
which ends the proof.
\end{proof}



\begin{cor} \label{cor:rvarbnd}
Assume that $L_K^{-r}f_\rho \in \L2$ with $r\in [1/2,1]$ and $\t^\theta\geq \ka^2 +1$. Then 
\[ \E \|f_t - f_{\la_t}\|_\rho^2 \leq 4 \M^2 + 4\|L_K^{-r} f_\rho \|_\rho^2 + 3 (4\ka^2+1) \ka^2 \M^2 \]
\end{cor}

\begin{proof} For $\t^\theta\geq \ka^2 +1$, we have $\ga_t \ka^2 + \ga_t\la_t \leq 1$, whence by Lemma \ref{lem:expan} with $f_0=0$, 
\begin{equation} 
\E \|f_t - f_{\la_t}\|_\rho^2 \leq \pi_1^t \|f_{\la_0}\|_\rho^2 + \sum_{j=1}^t \ga_j \la_j \pi_{j+1}^t 
\frac{\|f_{\la_j}-f_{\la_{j-1}}\|_\rho^2}{\ga_j\la_j} +  C_\rho \sum_{j=1}^t \frac{\ga_j}{\la_j} \ga_j \la_j \pi_{j+1}^t
\end{equation}

The first term is not larger than $2\M$, using $\pi_1^t \leq 1$ and $\|f_{\la}\|_\rho \leq 2 \M$ in Lemma \ref{lem:fla}(B). 

Now consider the second term. By Theorem \ref{thm:pie1}(A) with $r\in [1/2,1]$,  
\[ \|f_{\la_t} - f_{\la_{t-1}} \|_\rho \leq |\la_{t} - \la_{t-1}| \frac{\|L_K^{-r} f_\rho \|_\rho}{r} \leq 4(1-\theta)(t+\t)^{-r(1-\theta)-1} \|L_K^{-r} f_\rho \|_\rho 
\]
where we use 
\begin{eqnarray*}
|\lambda_{t}^{r} - \lambda_{t-1}^{r} | & = & | (t+\t)^{-r(1-\theta)} - (t+\t-1)^{-r(1-\theta)}| \\
& \leq & r(1-\theta) (t+\t-1)^{-r(1-\theta)-1} \leq 4r(1-\theta) (t+\t)^{-r(1-\theta)-1},
\end{eqnarray*}
since $(a+1)/(b+1)\geq a/b$ for $b>a>0$ and $r(1-\theta)+1\leq 2$. This gives for all $t\in \N$, $\t\geq 1$ and $\theta\in [1/2,1]$, 
\[ \frac{\|f_{\la_t} - f_{\la_{t-1}}\|_\rho^2}{\ga_t \la_t} \leq 16(1-\theta)^2 \|L_K^{-r} f_\rho \|_\rho^2 (t+t_0)^{-2r(1-\theta)-1} \leq 
4\|L_K^{-r} f_\rho \|_\rho^2 \]
Using the telescope sum
\begin{equation} \label{eq:tel}
\sum_{j=1}^t \ga_j \la_j \pi_{j+1}^t = 1 - \pi_1^t \leq 1, 
\end{equation}
we have a bound for the second term
\[ \sum_{j=1}^t \ga_j \la_j \pi_{j+1}^t \frac{\|f_{\la_j}-f_{\la_{j-1}}\|_\rho^2}{\ga_j\la_j} \leq 4\|L_K^{-r} f_\rho \|_\rho^2 \]

It remains to bound the third term. Note that for $\theta\in [1/2,1]$,
$$ \frac{\ga_t}{\la_t} = (t+t_0)^{-(2\theta-1)} \leq 1. $$ 
Together with the telescope sum (\ref{eq:tel}), the third term is not larger than $C_\rho$. This completes the proof.
\end{proof}


\section*{Appendix D: Proof of Theorem \ref{thm:convergence}} 
\renewcommand{\thesection}{D}
\setcounter{equation}{0} \setcounter{thm}{0}
\renewcommand{\thethm}{D.\arabic{thm}}
\renewcommand{\theequation}{D-\arabic{equation}}

\begin{proof}[Proof of Theorem \ref{thm:convergence}]
For convenience suppose that 
\begin{equation} \label{eq:finitevar}
\E \|(A_j \w_j - b_j)\|^2\leq C<\infty. 
\end{equation}
By Generalized Finiteness Condition (C) $\amin_t\to 0$ and condition (B) above $\ga_t/\amin_t\to 0$, we have $\ga_t\to 0$. 
Thus for a sufficiently large $\t$, we have for all $t> \t$, $\ga_t\amax + \ga_t \amin_t\leq 1$ and thus
\begin{equation} \label{eq:product}
\|\Pi_j^t\| \leq \prod_{i=j}^t (1 - a_i), \ \ \ \ \ \mbox{where $a_i:=\ga_i \amin_i$ and $t\geq j > \t$}. 
\end{equation}
Note that by condition (A), $\sum_i a_i = \sum_i \ga_i \amin_i =\infty$. 

The first term in (\ref{eq:rmart}) converges, by $\sum_i a_i=\infty$ and thus
\[ \|\Pi_{t_0+1}^t r_{t_0} \|\leq \prod_{i=t_0+1}^t (1 - a_i) \|r_{t_0}\|\leq e^{-\sum_{i=t_0+1}^t a_i}\| r_{t_0}\| \to 0. \]

Now consider the second term in (\ref{eq:rmart}). By independence of $(z_t)$, 
\[ 
\E \left\| \sum_{j=t_0+1}^t \gamma_j \Pi_{j+1}^t (A_j \w_j - b_j) \right \|^2 = \sum_{j=t_0+1}^t \ga_j^2 \E\|\Pi_{j+1}^t (A_j \w_j - b_j)\|^2 \leq 
C \sum_{j=t_0+1}^t \ga_j^2 \prod_{i=j+1}^t (1 - a_i)^2=:I.
\]
using (\ref{eq:finitevar}) and (\ref{eq:product}). 
$I$ can be further split into two parts: for each $\eps>0$, choose a large enough $J_1\in \N$ 
such that for all $j\geq J_1$, $\ga_j/\amin_j \leq \sqrt{\epsilon/2C}$ (by condition (B) $\ga_j/\amin_j \to 0$), whence
\[ I \leq C \sum_{j=t_0+1}^{J_1} \ga_j^2 \prod_{i=j+1}^t(1 - a_i)^2 + 
\frac{\epsilon}{2} \sum_{j=J_1+1}^t a_j^2 \prod_{i=j+1}^t(1 - a_i)^2 =: I_1+I_2.\]
Clearly $I_2 \leq \epsilon/2$, by using telescope sum
\begin{equation} \label{eq:telescope}
 \sum_{j=J}^t a_j \prod_{i=j+1}^t(1 - a_i)
=\sum_{j=J}^t [1-(1-a_j)]\prod_{i=j+1}^t(1 - a_i)= 1 - \prod_{i=t_0+1}^t(1-a_i) \leq 1.
\end{equation}
Also $\DS I_1\leq \epsilon/2$, since there is a large enough $T_2\in \N$ such that for all $t\geq T_2$, 
$\DS \sum_{i=J_1}^t a_i \geq \frac{1}{2} \log \frac{2 C J_1}{\epsilon\amax^2}$
(as condition (A) $\sum_i a_i =\infty$). 
Therefore $I\leq \eps$ which shows the convergence of the second term.  
 

As to the third term in (\ref{eq:rmart}), denoting $\de_t = \|\De_t\|$, by (\ref{eq:product}) we obtain
\[\left \| \sum_{j=t_0+1}^t \Pi_{j}^t \Delta_j \right \| \leq  \sum_{j=t_0+1}^t \de_j  \prod_{i=j+1}^t (1 - a_i). \]
By condition (C) $\de_t/a_t \to 0$, using a similar splitting treatment in the second term, we can obtain the convergence of the third
term.
\end{proof}

\bibliographystyle{chicagoc}
%\bibliography{yao}

\begin{thebibliography}{}

\bibitem[\protect\citeauthoryear{Berlinet and Thomas-Agnan}{Berlinet and
  Thomas-Agnan}{2004}]{BerTho04}
{\sc Berlinet, A.} and {\sc C.~Thomas-Agnan} (2004).
\newblock {\em Reproducing Kernel Hilbert Spaces in Probability and
  Statistics}.
\newblock Kluwer Academic Publishers.

\bibitem[\protect\citeauthoryear{Caponnetto and De{~V}ito}{Caponnetto and
  De{~V}ito}{2005}]{CapDeV05}
{\sc Caponnetto, A.} and {\sc E.~De{~V}ito} (2005).
\newblock Optimal rates for regularized least squares algorithm.
\newblock {\em preprint\/}.

\bibitem[\protect\citeauthoryear{Carmeli, DeVito, and Toigo}{Carmeli
  et~al.}{2005}]{CarDevToi05}
{\sc Carmeli, C.}, {\sc E.~DeVito}, and {\sc A.~Toigo} (2005).
\newblock Reproducing kernel hilbert spaces and mercer theorem.
\newblock {\em preprint\/}.

\bibitem[\protect\citeauthoryear{Cucker and Smale}{Cucker and
  Smale}{2002}]{CucSma02}
{\sc Cucker, F.} and {\sc S.~Smale} (2002).
\newblock On the mathematical foundations of learning.
\newblock {\em Bull. of the Amer. Math. Soc.\/}~{\em 29\/}(1), 1--49.

\bibitem[\protect\citeauthoryear{Daubechies}{Daubechies}{1992}]{Daubechies92}
{\sc Daubechies, I.} (1992).
\newblock {\em Ten Lectures on Wavelets}.
\newblock Philadelphia, PA: Society for Industrial and Applied Mathematics
  (SIAM).

\bibitem[\protect\citeauthoryear{Duflo}{Duflo}{1996}]{Duflo96}
{\sc Duflo, M.} (1996).
\newblock {\em Algorithmes Stochastiques}.
\newblock Berlin, Heidelberg: Springer-Verlag.

\bibitem[\protect\citeauthoryear{Engl, Hanke, and Neubauer}{Engl
  et~al.}{2000}]{EngHanNeu00}
{\sc Engl, H.~W.}, {\sc M.~Hanke}, and {\sc A.~Neubauer} (2000).
\newblock {\em Regularization of Inverse Problems}.
\newblock Kluwer Academic Publishers.

\bibitem[\protect\citeauthoryear{Gy\"{o}rfi, Kohler, Krzy\.{z}ak, and
  Walk}{Gy\"{o}rfi et~al.}{2002}]{GKKW02}
{\sc Gy\"{o}rfi, L.}, {\sc M.~Kohler}, {\sc A.~Krzy\.{z}ak}, and {\sc H.~Walk}
  (2002).
\newblock {\em A Distribution-Free Theory of Nonparametric Regression}.
\newblock Springer-Verlag, New York.

\bibitem[\protect\citeauthoryear{Halmos and Sunder}{Halmos and
  Sunder}{1978}]{HalSun78}
{\sc Halmos, R.~P.} and {\sc V.~S. Sunder} (1978).
\newblock {\em Bounded Integral Operators in $L^2$ Spaces}.
\newblock Vol. 96 of Ergebnisse der Mathematik und ihrer Grenzgebiete (Results
  in Mathematics and Related Areas). Berlin: Springer-Verlag.

\bibitem[\protect\citeauthoryear{Hastie, Rosset, Tibshirani, and Zhu}{Hastie
  et~al.}{2004}]{HasRosTibZhu04}
{\sc Hastie, T.}, {\sc S.~Rosset}, {\sc R.~Tibshirani}, and {\sc J.~Zhu}
  (2004).
\newblock The entire regularization path for the support vector machine.
\newblock {\em Journal of Machine Learning Research\/}~{\em 5}, 1391--1415.

\bibitem[\protect\citeauthoryear{Kivinen, Smola, and Williamson}{Kivinen
  et~al.}{2004}]{KivSmoWil04}
{\sc Kivinen, J.}, {\sc A.~J. Smola}, and {\sc R.~C. Williamson} (2004).
\newblock Online learning with kernels.
\newblock {\em IEEE Transactions on Signal Processing\/}~{\em 52\/}(8),
  2165--2176.

\bibitem[\protect\citeauthoryear{Kushner and Yin}{Kushner and
  Yin}{2003}]{KusYin03}
{\sc Kushner, H.~J.} and {\sc G.~G. Yin} (2003).
\newblock {\em Stochastic Approximations and Recursive Algorithms and
  Applications}.
\newblock Berlin, Heidelberg: Springer-Verlag.

\bibitem[\protect\citeauthoryear{Lo\'{e}ve}{Lo\'{e}ve}{1948}]{Loeve48}
{\sc Lo\'{e}ve, M.} (1948).
\newblock Fonctions al\`{e}atoires du second ordre.
\newblock In P.~L\`{e}vy (Ed.), {\em Processus Stochastiques et Mouvement
  Brownien}, Paris. Gauthier-Villars.

\bibitem[\protect\citeauthoryear{Neveu}{Neveu}{1975}]{Neveu75}
{\sc Neveu, J.} (1975).
\newblock {\em Discrete-Parameter Martingales}.
\newblock North-Holland Publishing Company.

\bibitem[\protect\citeauthoryear{Parzen}{Parzen}{1961}]{Parzen61}
{\sc Parzen, E.} (1961).
\newblock An approach to time series analysis.
\newblock {\em The Annals of Mathematical Statistics\/}~{\em 32\/}(4),
  951--989.

\bibitem[\protect\citeauthoryear{Pinelis}{Pinelis}{1994}]{Pinelis94}
{\sc Pinelis, I.} (1994).
\newblock Optimum bounds for the distributions of martingales in banach spaces.
\newblock {\em The Annals of Probability\/}~{\em 22\/}(4), 1679--1706.

\bibitem[\protect\citeauthoryear{Robbins and Monro}{Robbins and
  Monro}{1951}]{RobMon51}
{\sc Robbins, H.} and {\sc S.~Monro} (1951).
\newblock A stochastic approximation method.
\newblock {\em The Annals of Mathematical Statistics\/}~{\em 22\/}(3),
  400--407.

\bibitem[\protect\citeauthoryear{Smale and Yao}{Smale and Yao}{2005}]{SmaYao05}
{\sc Smale, S.} and {\sc Y.~Yao} (2005).
\newblock Online learning algorithms.
\newblock {\em Foundation of Computational Mathematics\/}.
\newblock preprint.

\bibitem[\protect\citeauthoryear{Smale and Zhou}{Smale and
  Zhou}{2004}]{SmaZho-ShannonI}
{\sc Smale, S.} and {\sc D.-X. Zhou} (2004).
\newblock Shannon sampling and function reconstruction from point values.
\newblock {\em Bull. of the Amer. Math. Soc.\/}~{\em 41\/}(3), 279--305.

\bibitem[\protect\citeauthoryear{Smale and Zhou}{Smale and
  Zhou}{2005}]{SmaZho-ShannonIII}
{\sc Smale, S.} and {\sc D.-X. Zhou} (2005).
\newblock Learning theory estimates via integral operators and their
  approximations.
\newblock {\em preprint\/}.

\bibitem[\protect\citeauthoryear{Wahba}{Wahba}{1990}]{Wahba90}
{\sc Wahba, G.} (1990).
\newblock {\em Spline Models for Observational Data}.
\newblock SIAM.
\newblock CBMS-NSF Regional Conference Series in Applied Mathematics, 59.

\bibitem[\protect\citeauthoryear{Yao}{Yao}{2005}]{Yao05}
{\sc Yao, Y.} (2005).
\newblock On complexity issue of online learning algorithms.
\newblock {\em IEEE Transactions on Information Theory\/}.
\newblock submitted.

\end{thebibliography}

\end{document}
